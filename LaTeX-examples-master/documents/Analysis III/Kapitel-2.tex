\index{Lebesgue-Maß}

In diesem Kapitel sei \(X\) eine Menge, \(X\neq\emptyset\).
\begin{definition}
    \index{Ring}
    Sei \(\emptyset\neq \fr \subseteq \cp(X)\).
    $\fr$ heißt ein \textbf{Ring} auf \(X\), genau dann wenn gilt:
    \begin{enumerate}
        \item[(R1)] \(\emptyset \in \fr\)
        \item[(R2)] \(A,B \in \fr \, \implies \; A\cup B, \, B \setminus A \in \fr\)
    \end{enumerate}
\end{definition}

\textbf{Hinweis}: $(\fr, \cup, \setminus)$ ist kein Ring im Sinne
der linearen Algebra, $(\fr, \cup)$ kein Inverses Element hat und
$(\fr, \cup)$ nicht kommutativ ist.

\begin{definition}
    \index{Elementarvolumen}
    \index{Figuren}
    Sei \(d\in\MdN\).
    \begin{enumerate}
        \item \(\ci_d :=\Set{(a,b] | a,b \in \MdR^{d}, \, a \leq b} (\emptyset \in \ci_d)\).
              Seien \(a=(a_{1},\dots,a_{d}),\,b=(b_{1},\dots,b_{d})\in\MdR^d\)
              und \(I:=(a,b] \in \ci_{d}\)
              \[
              \lambda_{d}(I)= \begin{cases}
                0                                             & \text{falls }I=\emptyset\\
                (b_{1}-a_{1})(b_{2}-a_{2})\dots(b_{d}-a_{d}) & \text{falls }I\neq\emptyset\end{cases}\quad\text{(\textbf{Elementarvolumen})}
              \]
        \item \(\cf_d:=\Set{\bigcup_{j=1}^{n}I_{j} | n\in\MdN,\,I_{1},\dots,I_{n}\in \ci_d}\) (\textbf{Menge der Figuren})
    \end{enumerate}
\end{definition}
Ziel dieses Kapitels: Fortsetzung von \(\lambda_{d}\) auf \(\cf_{d}\)
und dann auf \(\fb_d\) (\(\leadsto\) Lebesgue-Maß)

Beachte: \(\ci_{d}\subseteq\cf_{d}\subseteq\fb_{d}\overset{1.4}{\implies}\fb_{d}=\sigma(\ci_{d})=\sigma(\cf_{d})\)
\begin{lemma}
    \label{Lemma 2.1}
    Seien \(I,I'\in\ci_{d}\) und \(A\in\cf_{d}\). Dann:
    \begin{enumerate}
        \item \(I\cap I'\in\ci_{d}\)
        \item \(I\setminus I'\in\cf_{d}.\)
              Genauer: \(\exists\left\{I_{1}',\dots,I_{l}'\right\}\subseteq\ci_{d}\) disjunkt:
              \(I\setminus I'=\bigcup_{j=1}^{l}{I_{j}'}\) % \bigcupdot
        \item \(\exists\left\{I_{1}',\dots,I_{l}'\right\}\subseteq\ci_{d}\) disjunkt: \(A=\bigcup_{j=1}^{l}{I_{j}'}\)
        \item \(\cf_d\) ist ein Ring.
    \end{enumerate}
\end{lemma}

\begin{beweis}
\begin{enumerate}
    \item Sei \(I=\prod_{k=1}^{d}{(a_{k},b_{k}]},
             \,I'=\prod_{k=1}^{d}{(\alpha_{k},\beta_{k}]};
   \,\alpha_{k}':=\max\{\alpha_{k},a_{k}\},
    \,\beta_{k}':=\min\{\beta_{k},b_{k}\}\)

          \(\exists k\in\Set{1,\dots,d} : \alpha_{k}'\geq\beta_{k}'
            \implies I\cap I'=\emptyset\in\ci_{d}\).\\
          Sei \(\alpha_{k}'<\beta_{k}'\forall k\in\{1,\dots,d\}\), so
          ist \(I\cap I'=\prod_{k=1}^{d}{(\alpha_{k}',\beta_{k}']\in\ci_{d}}\)
    \item Induktion nach \(d\):
          \begin{itemize}
            \item[I.A.] Klar \checkmark % hier fehlt noch eine Graphik
            \item[I.V.] Die Behauptung gelte für ein \(d\geq 1\)
            \item[I.S.] Seien \(I,I'\in\ci_{d+1}\). Es existieren \(I_{1},I_{1}'\in\ci_{1}\) und \(I_{2},I_{2}'\in\ci_{d}\) mit:
                        \(I=I_{1}\times I_{2},\,I'=I_{1}'\times I_{2}'\)
                        % Graphik einfuegen!

                        Nachrechnen:
                        \[
                        I\setminus I'=(I_{1}\setminus I_{1}')\times I_{2}\dot \cup(I_{1}\cap I_{1}')\times(I_{2}\setminus I_{2}')
                        \]
                        I.A.\(\implies\,I_{1}\setminus I_{1}'=\) endliche disjunkte Vereinigung von Elementen aus \(\ci_{1}\)\\
                        I.V.\(\implies\,I_{2}\setminus I_{2}'=\) endliche disjunkte Vereinigung von Elementen aus \(\ci_{d}\)\\
                        Daraus folgt die Behauptung für \(d+1\)
          \end{itemize}
    \item \begin{itemize}
            \item[\underline{Vor.:}] Sei $n \in \mdn$ und
                \(A=\bigcup_{j=1}^{n}{I_{j}}\) mit
                \(I_{1},\dots,I_{d}\in\ci_{d}\)
            \item[\underline{Beh.:}] Es existiert
                \(\{I_{1}',\dots,I_{l}'\}\subseteq\ci_{d}\) disjunkt:
                \(A=\bigcup_{j=1}^{l}{I_{j}'}\)
          \item[\underline{Bew.:}] mit Induktion nach $n$:
          \begin{itemize}
            \item[I.A.] \(n=1:\,A=I_{1}\)\checkmark
            \item[I.V.] Die Behauptung gelte für ein \(n\geq 1\)
            \item[I.S.] Sei \(A=\bigcup_{j=1}^{n+1}{I_{j}}\quad(I_{1},\dots,I_{n+1}\in\ci_{d})\)

                        IV\(\,\implies\,\exists\{I_{1}',\dots,I_{l}'\}\subseteq\ci_{d}\) disjunkt:
                        \(\bigcup_{j=1}^{n}{I_{j}}=\bigcup_{j=1}^{l}{I_{j}'}\)	% \bigcupdot

                        Dann: \(A=I_{n+1}\cup\bigcup_{j=1}^{l}{I_{j}'}=I_{n+1}\cup\bigcup_{j=1}^{l}{(I_{j}'\setminus I_{n+1})}\) % \cupdot

                        Wende (2) auf jedes \(I_{j}'\setminus I_{n+1}\) an \((j=1,\dots,l)\):
                        \(I_{j}'\setminus I_{n+1}=\bigcup_{j=1}^{l_{j}}{I_{j}''}\quad(I_{j}''\in\ci_{d})\)

                        Damit folgt:
                        \[
                        A=I_{n+1}\cup\bigcup_{j=1}^{l}{\left(\bigcup_{j=1}^{l_{j}}{I_{j}''}\right)}
                        \]
                        Daraus folgt die Behauptung für \(n+1\).
            \end{itemize}
        \end{itemize}
    \item \((a,a]=\emptyset\implies\emptyset\in\cf_{d}\)

          Seien \(A,B\in\cf_{d}\). Klar: \(A\cup B\in\cf_{d}\)

          Sei \(A=\bigcup_{j=1}^{n}{I_{j}},\,B=\bigcup_{j=1}^{n}{I_{j}'}\quad(I_{j},I_{j}'\in\ci_{d})\). Zu zeigen: \(B\setminus A\in\cf_{d}\)
          \begin{itemize}
            \item[I.A.] \(n=1:\,A=I_{1}\implies B\setminus A=\bigcup_{j=1}^{n}(\underbrace{I_{j}'\setminus I_{j}}_{\in\cf_{d}})\). Wende
                        (2) auf jedes \(I_{j}'\setminus I_{1}\) an. Aus (2) folgt dann \(B\setminus A\in\cf_{d}\).
            \item[I.V.] Die Behauptung gelte für ein \(n\in\MdN\)
            \item[I.S.] Sei \(A'=A\cup I_{n+1}\quad(I_{n+1}\in\ci_{d})\). Dann:
                        \[
                        B\setminus A'=\underbrace{(B\setminus A)}_{\in\cf_{d}}\setminus\underbrace{I_{n+1}}_{\in\cf_{d}}\in\cf_{d}
                        \text{ (siehe I.A.)}
                        \]
          \end{itemize}
  \end{enumerate}
\end{beweis}
ohne Beweis:
\begin{lemma}[Unabhängigkeit von der Darstellung]
    \label{Lemma 2.2}
    Sei \(A\in\cf_{d}\) und \(\{I_{1},\dots,I_{n}\}\subseteq\ci_{d}\) disjunkt und
    \(\{I_{1}',\dots,I_{m}'\}\subseteq\ci_{d}\) disjunkt mit
    \(\bigcup_{j=1}^{n}{I_{j}}=A=\bigcup_{j=1}^{m}{I_{j}'}\). Dann:
    \[
    \sum_{j=1}^{n}{\lambda_{d}(I_{j})}=\sum_{j=1}^{m}{\lambda_{d}(I_{j}')}
    \]
\end{lemma}
\begin{definition}
    Sei \(A\in\cf_{d}\) und \(A=\bigcup_{j=1}^{n}{I_{j}}\) mit
    \(\{I_{1},\dots,I_{n}\}\subseteq\ci_{d}\)
    disjunkt (beachte Lemma \ref{Lemma 2.1}, Punkt 3).
    \[
    \lambda_{d}(A):=\sum_{j=1}^{n}{\lambda_{d}(I_{j})}
    \]
    \folgtnach{\ref{Lemma 2.2}} \(\lambda_{d}:\cf_{d}\to[0,\infty)\)
    ist wohldefiniert.
\end{definition}
\begin{satz}
    \label{Satz 2.3}
    Seien \(A,B\in\cf_{d}\) und \((B_{n})\) sei eine Folge in \(\cf_{d}\).
    \begin{enumerate}
        \item \(A\cap B=\emptyset\implies\lambda_{d}(A\cup B)=\lambda_{d}(A)+\lambda_{d}(B)\)
        \item \(A\subseteq B\implies\lambda_{d}(A)\leq\lambda_{d}(B)\)
        \item \(\lambda_{d}(A\cup B)\leq\lambda_{d}(A)+\lambda_{d}(B)\)
        \item Sei \(\delta>0\). Es existiert \(C\in\cf_{d}:\overline{C}\subseteq B\)
              und \(\lambda_{d}(B\setminus C)\leq\delta\).
        \item Ist \(B_{n+1}\subseteq B_{n}\forall n\in\mdn\) und
              \(\bigcap B_{n}=\emptyset\), so gilt:
              \(\lambda_{d}(B_{n})\to 0\,(n\to \infty)\)
    \end{enumerate}
\end{satz}

\begin{beweis}
\begin{enumerate}
\item Aus Lemma \ref{Lemma 2.1} folgt: Es existiert
\(\{I_{1},\dots,I_{n}\}\subseteq\ci_{d}\)
disjunkt und es existiert \(\{I_{1}',\dots,I_{m}'\}\subseteq\ci_{d}\) disjunkt:
\(A=\bigcup_{j=1}^{n}{I_{j}},\,B=\bigcup_{j=1}^{m}{I_{j}'}\).

\(J:=\{I_{1},\dots,I_{n},I_{1}',\dots,I_{m}'\}\subseteq\ci_{d}\). Aus
\(A\cap B=\emptyset\) folgt: \(J\) ist disjunkt. Dann:
\(A\cup B=\bigcup_{I\in J}{I}\)	% Hier auch wieder: \bigcupdot

Also:
\begin{align*}
\lambda_{d}(A\cup B)&=\sum_{I\in J}{\lambda_{d}(I)}\\
    &=\sum_{j=1}^{n}{\lambda_{d}(I_{j})}+\sum_{j=1}^{m}{\lambda_{d}(I_{j}')}\\
    &=\lambda_{d}(A)+\lambda_{d}(B)
\end{align*}
\item wie bei Satz \ref{Satz 1.7}
\item \(\lambda_{d}(A\cup B)=\lambda(A \dot{\cup} (B\setminus A))\overset{(1)}{=}\lambda_{d}(A)+\lambda_{d}(B\setminus A)\overset{(2)}{\leq}\lambda_{d}(A)+\lambda_{d}(B)\) % \cupdot
\item Übung (es genügt \(B\in\ci_{d}\) zu betrachten).
\item Sei \(\varepsilon>0\). Aus (4) folgt: Zu jedem \(B_{n}\) existiert ein
\(C_{n}\in\cf_{d}:\overline{C}_{n}\subseteq B_{n}\) und
\begin{equation}
\label{eq: Abschaetzung Mass -- Beweis Satz 2.3.(5)}
\lambda_{d}(B_{n}\setminus C_{n})\leq\frac{\varepsilon}{2^{n}}
\end{equation}
Dann:
\(\bigcap{\overline{C}_{n}}\subseteq\bigcap{B_{n}}=\emptyset\implies\bigcup{\overline{C}_{n}^{c}}=\mdr^{d}\implies\underbrace{\overline{B}_{1}}_{\text{kompakt}}\subseteq\bigcup{\underbrace{\overline{C}_{n}^{c}}_{\text{offen}}}\)

Aus der Definition von Kompaktheit (Analysis II, \S 2) folgt:
\(\exists m\in\mdn:\,\bigcup_{j=1}^{m}{\overline{C}_{j}^{c}}\supseteq\overline{B}_{1}\)
Dann: \(\bigcap_{j=1}^{m}{\overline{C}_{j}}\subseteq\overline{B}_{1}^{c}\).
Andererseits: \(\bigcap_{j=1}^{m}{\overline{C}_{j}}\subseteq\bigcap_{j=1}^{m}{B_{j}}\subseteq B_{1}\subseteq\overline{B}_{1}\).

Also: \(\bigcap_{j=1}^{m}{\overline{C}_{j}}=\emptyset\). Das heißt:
\(\bigcap_{j=1}^{n}{\overline{C}_{j}}=\emptyset \quad \forall n\geq m\)

\(D_{n}:=\bigcap_{j=1}^{n}{C_{j}}\). Dann: \(D_{n}=\emptyset \quad \forall n\geq m\)

\textbf{Behauptung:} \(\lambda_{d}(B_{n}\setminus D_{n})\leq\left(1-\frac{1}{2^{n}}\right)\ep \quad \forall n\in\mdn\)
\begin{beweis} (induktiv)
\begin{itemize}
\item[I.A.] \(\lambda_{d}(B_{1}\setminus D_{1})=\lambda_{d}(B_{1}\setminus C_{1})\overset{\eqref{eq: Abschaetzung Mass -- Beweis Satz 2.3.(5)}}{\leq}\frac{\ep}{2}=\left(1-\frac{1}{2}\right)\ep\) \checkmark
\item[I.V.] Sei \(n\in\mdn\) und es gelte
            $\lambda_{d}(B_{n}\setminus D_{n})\leq\left(1-\frac{1}{2^{n}}\right)\ep$
\item[I.S.] \begin{align*}
    \lambda_{d}(B_{n+1}\setminus D_{n+1})&=\lambda_{d}\left((B_{n+1}\setminus D_{n})\cup(B_{n+1}\setminus C_{n+1})\right)\\
    &\overset{(3)}{\leq}\lambda_{d}(\underbrace{B_{n+1}\setminus D_n}_{\subseteq B_{n}\setminus D_{n}})+\underbrace{\lambda_{d}(B_{n+1}\setminus C_{n+1})}_{\overset{\eqref{eq: Abschaetzung Mass -- Beweis Satz 2.3.(5)}}{\leq}\frac{\ep}{2^{n+1}}}\\
    &\overset{(2)}{\leq}\lambda_{d}(B_{n}\setminus D_{n})+\frac{\ep}{2^{n+1}}\\
    &\overset{\text{I.V.}}{\leq}\left(1-\frac{1}{2^{n}}\right)+\frac{\ep}{2^{n+1}}\\
&=\left(1-\frac{1}{2^{n+1}}\right)\ep
    \end{align*}
\end{itemize}
\end{beweis}

Für \(n\geq m:\,D_{n}=\emptyset\,\implies\,\lambda_{d}(B_{n})=\lambda_{d}(B_{n}\setminus D_{n})\leq\left(1-\frac{1}{2^{n}}\right)\varepsilon\leq\varepsilon\)
\end{enumerate}
\end{beweis}

\begin{definition}
\index{Prämaß}
Es sei \(\fr\) ein Ring auf \(X\). Eine Abbildung \(\mu:\fr\to[0,\infty]\)
heißt ein \textbf{Prämaß} \ auf \(\fr\), wenn gilt:
\begin{enumerate}
\item \(\mu(\emptyset)=0\)
\item Ist \(A_{j}\) eine disjunkte Folge in \(\fr\) und \(\bigcup{A_{j}}\in\fr\), so ist \(\mu\left(\bigcup{A_{j}}\right)=\sum{\mu(A_{j})}\).
\end{enumerate}
\end{definition}

\begin{satz}
\label{Satz 2.4}
\(\lambda_{d}:\cf_{d}\to[0,\infty]\) ist ein Prämaß auf $\cf_{d}$.
\end{satz}
\begin{beweis}
\begin{enumerate}
\item Klar: \(\lambda_{d}(\emptyset)=0\)
\item Sei \(A_{j}\) eine disjunkte Folge in \(\cf_{d}\) und \(A:=\bigcup{A_{j}}\in\cf_{d}\).

\(B_{n}:=\bigcup_{j=n}^{\infty}{A_{j}}\,(n\in\mdn)\); \((B_{n})\) hat die
Eigenschaften aus \ref{Satz 2.3}, Punkt 5. Also: \(\lambda_{d}(B_{n})\to 0\).

Für \(n\geq 2\):
\[
\lambda_{d}(A)=\lambda_{d}(A_{1}\cup\dots\cup A_{n-1}\cup B_{n})\overset{\ref{Satz 2.3}.(1)}{=}\sum_{j=1}^{n-1}{\lambda_{d}(A_{j})}+\lambda_{d}(B_{n})
\]
Daraus folgt:
\[
\sum_{j=1}^{n-1}{\lambda_{d}(A_{j})}=\lambda_{d}(A)-\lambda_{d}(B_{n})\quad\forall n\geq 2
\]
Mit \(n\to\infty\) folgt die Behauptung.
\end{enumerate}
\end{beweis}

Ohne Beweis:
\begin{satz}[Fortsetzungssatz von Carath\'eodory]
\label{Satz 2.5}
Sei \(\fr\) ein Ring auf \(X\) und \(\mu:\fr\to[0,\infty]\) ein Prämaß. Dann
existiert ein Maßraum \((X,\fa(\mu),\overline{\mu})\) mit
\begin{enumerate}
\item \(\sigma(\fr)\subseteq\fa(\mu)\)
\item \(\overline{\mu}(A)=\mu(A) \quad \forall A\in\fr\)
\end{enumerate}
Insbesondere: \(\overline{\mu}\) ist ein Maß\ auf \(\sigma(\fr)\).
\end{satz}

\begin{satz}[Eindeutigkeitssatz]
\label{Satz 2.6}
Sei \(\emptyset\neq\ce\subseteq\cp(X)\), es seien \(\nu,\,\mu\) Maße auf
\(\sigma(\ce)\).

Es gelte:
\begin{enumerate}
    \item \(E,F\in\ce\implies E\cap F\in\ce\quad\text{(durchschnittstabil)}\)
    \item $\exists$ eine Folge \((E_{n})\) in \(\ce\): \(\bigcup{E_{n}}=X\)
          und \(\mu(E_{n})<\infty \quad \forall n\in\mdn\).
    \item \(\mu(E)=\nu(E) \quad \forall E\in\ce\)
\end{enumerate}
Dann: \(\mu=\nu\) auf \(\sigma(\ce)\).
\end{satz}

\begin{satz}
\label{Satz 2.7}
\index{Lebesgue-Maß}
Es gibt genau eine Fortsetzung von \(\lambda_{d}:\cf_{d}\to[0,\infty]\) auf
\(\fb_{d}\) zu einem Maß. Diese Fortsetzung heißt \textbf{Lebesgue-Maß} \ (L-Maß)
und wird ebenfalls mit \(\lambda_{d}\) bezeichnet.
\end{satz}
\begin{beweis}
\folgtnach{(\ref{Lemma 2.1}) und (\ref{Satz 2.4})}: \(\lambda_{d}\) ist ein
Prämaß\ auf \(\fr:=\cf_{d}\); es ist \(\sigma(\fr)=\fb_{d}\).

\folgtnach{\ref{Satz 2.5}}: \(\lambda_{d}\) kann zu einem Maß auf
\(\sigma(\cf_{d}) = \fb_{d}\) fortgesetzt werden. Für diese
Fortsetzung schreiben wir wieder $\lambda_d$, also
$\lambda_d: \fb_{d} \rightarrow [0, +\infty]$

Sei \(\nu\) ein weiteres Maß\ auf \(\fb_{d}\) mit:
\(\nu(A)=\lambda_{d}(A)\,\forall A\in\cf_{d}\). \(\ce:=\ci_{d}\). Dann:
\(\sigma(\ce)\overset{\ref{Satz 1.4}}{=}\fb_{d}\).
\begin{enumerate}
    \item \(E,F\in\ce\overset{\ref{Lemma 2.1}}{\implies}E\cap F\in\ce\)
    \item \(E_{n}:=(-n,n]^{d}\)
          Klar:
          \begin{align*}
            \bigcup E_{n}&=\mdr^{d}\\
            \lambda_{d}(E_{n})&=(2n)^{d}<\infty
          \end{align*}
\end{enumerate}
Klar: \(\nu(E)=\lambda_{d}(E)\,\forall E\in\ce\). Mit Satz \ref{Satz 2.6} folgt
dann: \(\nu=\lambda_{d}\) auf \(\fb_{d}\).
\end{beweis}

\begin{bemerkung}
Sei \(X\in\fb_{d}\). Aus 1.6 folgt: \(\fb(X)=\Set{A\in\fb_{d} | A\subseteq X}\).
Die Einschränkung von \(\lambda_{d}\) auf \(\fb(X)\) heißt ebenfalls
L-Maß\ und wird mit \(\lambda_{d}\) bezeichnet.
\end{bemerkung}

\begin{beispieleX}
\begin{enumerate}
\item Seien \(a=(a_{1},\dots,a_{d}),\,b=(b_{1},\dots,b_{d})\in\mdr^{d},\,a\leq b\) und \(I=[a,b]\).\\
\textbf{Behauptung}\\\(\lambda_{d}([a,b])=(b_{1}-a_{1})\dots(b_{d}-a_{d})\) (Entsprechendes gilt für \((a,b)\) und \([a,b)\))
\begin{beweis}
\(I_{n}:=(a_{1}-\frac{1}{n},b_{1}]\times\dots\times(a_{d}-\frac{1}{n},b_{d}];\,I_{1}\supset I_{2}\supset\dots;\,\bigcap I_{n}=I,\,\lambda_{d}(I_{1})<\infty\)

Aus Satz \ref{Satz 1.7}, Punkt 5, folgt:
\begin{align*}
\lambda_{d}(I)&=\lim_{n\to\infty}{\lambda_{d}(I_{n})}\\
&=\lim_{n\to\infty}{(b_{1}-a_{1}+\frac{1}{n})\dots(b_{d}-a_{d}+\frac{1}{n})}\\
&=(b_{1}-a_{1})\dots(b_{d}-a_{d})
\end{align*}
\end{beweis}
\item Sei \(a\in\mdr^{d},\,\{a\}=[a,a]\in\fb_{d}\). \folgtnach{Bsp (1)} \(\lambda_{d}(\{a\})=0\).
\item \(\mdq^{d}\) ist abzählbar, also: \(\mdq^{d}=\{a_{1},a_{2},\dots\}\)
mit \(a_{j}\neq a_{i}\,(i\neq j)\). Dann: \(\mdq^{d}=\bigcup\{a_{j}\}\) %\bigcupdot

Dann gilt: \(\mdq^{d}\in\fb_{d}\) und \(\lambda_{d}(\mdq^{d})=\sum{\lambda_{d}(\{a_{j}\})}=0\).
\item Wie in Beispiel (3): Ist \(A\subseteq\mdr^{d}\) abzählbar, so ist
\(A\in\fb_{d}\) und \(\lambda_{d}(A)=0\).
\item Sei \(j\in\{1,\dots,d\}\) und \(H_{j}:=\Set{(x_{1},\dots,x_{d})\in\mdr^{d} | x_{j}=0}\). \(H_{j}\) ist abgeschlossen, damit folgt: \(H_{j}\in\fb_{d}\).

Ohne Beschränkung der Allgemeinheit sei \(j=d\). Dann:
\(I_{n}:=\underbrace{[-n,n]\times\dots\times[-n,n]}_{(d-1)-\text{mal}}\times\{0\}\).
% Hier fehlt noch eine Graphik
Aus Beispiel (1) folgt: \(\lambda_{d}(I_{n})=0\).

Aus \(H_{d}=\bigcup{I_{n}}\) folgt: \(\lambda_{d}(H_{d})\leq\sum{\lambda_{d}(I_{n})}=0\). Also: \(\lambda_{d}(H_{j})=0\).
\end{enumerate}
\end{beispieleX}

\begin{definition}
    Sei $x\in\mdr^d, \emptyset \neq A\subseteq\mdr^d$. Definiere:
    \begin{align*}
        x+A          &:= \Set{x+a | a \in A}\\
        x+ \emptyset &:= \emptyset
    \end{align*}
\end{definition}

\begin{beispiel}
Ist $I\in\ci_d$, so gilt $x+I\in\ci_d$ und $\lambda_d(x+I)=\lambda_d(I)$.
\end{beispiel}

\begin{satz}
\label{Satz 2.8}
Sei $x\in\mdr^d, \fa:=\{B\in\fb_d:x+B\in\fb_d\}$ und $\mu:\fa\to[0,\infty]$ sei definiert durch $\mu(A):=\lambda_d(x+A)$. Dann gilt:
\begin{enumerate}
\item $(\mdr^d,\fa,\mu)$ ist ein Maßraum.
\item Es ist $\fa=\fb_d$ und $\mu=\lambda_d$ auf $\fb_d$. D.h. für alle $A\in\fb_d$ ist $x+A\in\fb_d$ und $\lambda_d(x+A)=\lambda_d(A)$ (Translationsinvarianz des Lebesgue-Maßes).
\end{enumerate}
\end{satz}

\begin{beweis}
\begin{enumerate}
\item Leichte Übung!
\item Es ist klar, dass $\fb_d\supseteq\fa$. Nach dem Beispiel von oben gilt:
\[\ci_d\subseteq\fa\subseteq\fb_d=\sigma(\ci_d)\subseteq\sigma(\fa)=\fa\]
Setze $\ce:=\ci_d$, dann ist $\sigma(\ce)=\fb_d$ und es gilt nach dem Beispiel von oben:
\[\forall E\in\ce:\mu(E)=\lambda_d(E)\]
$\ce$ hat die Eigenschaften (1) und (2) aus Satz \ref{Satz 2.6}, daraus folgt dann, dass $\mu=\lambda_d$ auf $\fb_d$ ist.
\end{enumerate}
\end{beweis}

Ohne Beweis:
\begin{satz}
    \label{Satz 2.9}
    Sei $\mu$ ein Maß auf $\fb_d$ mit der Eigenschaft:
    \[\forall x\in\mdr^d, A\in\fb_d:\mu(A)=\mu(x+A)\]
    Weiter sei $c:=\mu((0,1]^d)<\infty$. Dann gilt:
    \[\mu=c\cdot\lambda_d\]
    Falls $c=1$, so ist $\mu$ das Lebesgue-Maß.
\end{satz}

\begin{satz}[Regularität des Lebesgue-Maßes]
\label{Satz 2.10}
Sei $A \in\fb_d$, dann gilt:
\begin{enumerate}
\item
$\lambda_d(A)
 =\inf\Set{\lambda_d(G) | G\subseteq\mdr^d\text{ offen und }A \subseteq G}\\
 =\inf\Set{\lambda_d(V) | V=\bigcup_{j=1}^\infty I_j, I_j\subseteq\mdr^d\text{ offenes Intervall }, A\subseteq V}$
\item $\lambda_d(A)=\sup\Set{\lambda_d(K) | K\subseteq\mdr^d\text{ kompakt }, K\subseteq A}$
\end{enumerate}
\end{satz}

\begin{beweis}
\begin{enumerate}
\item Ohne Beweis.
\item Setze $\beta:=\sup\Set{\lambda_d(K) | K\subseteq\mdr^d\text{ kompakt }, K\subseteq A}$.
      Sei $K$ kompakt und $K\subseteq A$, dann gilt $\lambda_d(K)\le\lambda_d(A)$, also ist auch $\beta\le\lambda_d(A)$.

\textbf{Fall 1:} Sei $A$ zusätzlich beschränkt.\\
Sei $\ep>0$. Es existiert ein $r>0$, sodass $A\subseteq B:=\overline{U_r(0)}\subseteq[-r,r]^d$ ist, dann gilt:
\[\lambda_d(A)\le\lambda_d([-r,r]^d)=(2r)^d<\infty\]
Aus (1) folgt, dass eine offene Menge $G\supseteq B\setminus A$ existiert mit $\lambda_d(G)\le\lambda_d(B\setminus A)+\ep$. Dann gilt nach \ref{Satz 1.7}:
\[\lambda_d(B\setminus A)=\lambda_d(B)-\lambda_d(A)\]
Setze nun $K:=B\setminus G=B\cap G^c$, dann ist $K$ kompakt und $K\subseteq B\setminus(B\setminus A)=A$. Da $B\subseteq G\cup K$ ist, gilt:
\[\lambda_d(B)\le\lambda_d(G\cup K)\le \lambda_d(B)-\lambda_d(A)+\ep+\lambda_d(K)\]
Woraus folgt:
\[\lambda_d(A)\le\lambda_d(K)+\ep\]

\textbf{Fall 2:} Sei $A\in\fb_d$ beliebig.\\
Setze $A_n:=A\cap\overline{U_n(0)}$. Dann ist $A_n$ für alle $n\in\mdn$ beschränkt, $A_n\subseteq A_{n+1}$ und $A=\bigcup_{n\in\mdn} A_n$. Nach \ref{Satz 1.7} gilt:
\[\lambda_d(A)=\lim\lambda_d(A_n)\]
Aus Fall 1 folgt, dass für alle $n\in\mdn$ ein kompaktes $K_n\subseteq A_n$ mit $\lambda_d(A_n)\le\lambda_d(K_n)+\frac1n$ existiert. Dann gilt:
\[\lambda_d(A_n)\le\lambda_d(K_n)+\frac1n\le\lambda_d(A)+\frac1n\]
Also auch:
\[\lambda_d(A)=\lim\lambda(K_n)\le\beta\]
\end{enumerate}
\end{beweis}

\textbf{Auswahlaxiom:}\\
Sei $\emptyset\ne\Omega$ Indexmenge, es sei $\Set{X_\omega | \omega\in\Omega}$
ein disjunktes System von nichtleeren Mengen $X_\omega$. Dann
existiert ein $C\subseteq\bigcup_{\omega\in\Omega}X_\omega$, sodass
$C$ mit jedem $X_j$ genau ein Element gemeinsam hat.

\begin{satz}[Satz von Vitali]
\label{Satz 2.11}
Es existiert ein $C\subseteq\mdr^d$ sodass $C\not\in\fb_d$.
\end{satz}

\begin{beweis}
Wir definieren auf $[0,1]^d$ eine Äquivalenzrelation $\sim$, durch:
\begin{align*}
\forall x,y\in[0,1]^d: x \sim y\iff x-y\in\mdq^d\\
\forall x\in[0,1]^d:[x]:=\Set{y\in[0,1]^d | x\sim y}
\end{align*}
Nach dem Auswahlaxiom existiert ein $C\subseteq[0,1]^d$, sodass $C$ mit jedem $[x]$ genau ein Element gemeinsam hat.
Es ist $\mdq^d\cap[-1,1]^d=\{q_1,q_2,\dots\}$ mit $q_i\ne q_j$ für $(i\ne j)$. Dann gilt:
\begin{align*}
\tag{1} \bigcup_{n=1}^\infty(q_n+C)\subseteq[-1,2]^d\\
\tag{2} [0,1]^d\subseteq\bigcup_{n=1}^\infty(q_n+C)
\end{align*}
\begin{beweis}
Sei $x\in[0,1]^d$. Wähle $y\in C$ mit $y\in[x]$, dann ist $x\sim y$, also $x-y\in\mdq^d\cap[-1,1]^d$. D.h.:
\[\exists n\in\mdn: x-y=q_n\implies x=q_n+y\in q_n+C\]
\end{beweis}
Außerdem ist $\Set{q_n+C | n\in\mdn}$ disjunkt.
\begin{beweis}
Sei $z\in(q_n+C)\cap(q_m+C)$, dann existieren $a,b\in\mdq^d$, sodass gilt:
\begin{align*}
(q_n+a=z=q_m+b) &\implies (b-a=q_m-q_n\in\mdq^d)\\
&\implies (a\sim b) \implies([a]=[b])\\
&\implies (a=b)\implies (q_n=q_m)
\end{align*}
\end{beweis}
\textbf{Annahme:} $C\in\fb_d$, dann gilt nach (1):
\begin{align*}
3^d&=\lambda_d([-2,1]^d)\\
&\ge\lambda_d(\bigcup(q_n+C))\\
&=\sum \lambda_d(q_n+C)\\
&=\sum \lambda_d(C)
\end{align*}
Also ist $\lambda_d(C)=0$. Damit folgt aus (2):
\begin{align*}
1&=\lambda_d([0,1]^d)\\
&\le \lambda_d(\bigcup (q_n+C))\\
&=\sum \lambda_d(C)\\
&=0
\end{align*}
\end{beweis}
