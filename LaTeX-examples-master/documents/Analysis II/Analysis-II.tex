% Original Source: http://mitschriebwiki.nomeata.de/data/SS10/Ana2Bachelor.tex
\documentclass[a4paper,oneside,DIV15,BCOR12mm]{scrbook}
\usepackage{mathe}
\usepackage{saetze-schmoeger}

\lecturer{Dr. C. Schmoeger}
\semester{Sommersemester 2010 und 2012}
\scriptstate{complete}

\author{Die Mitarbeiter von \href{http://mitschriebwiki.nomeata.de/}{mitschriebwiki.nomeata.de} und \href{https://github.com/MartinThoma/LaTeX-examples/tree/master/documents}{GitHub}}
\title{Analysis II}
\makeindex

\hypersetup{
  pdfauthor   = {Die Mitarbeiter von mitschriebwiki.nomeata.de und GitHub},
  pdfkeywords = {Analysis},
  pdftitle    = {Analysis II}
}

\begin{document}
\maketitle

\renewcommand{\thechapter}{\Roman{chapter}}
%\chapter{Inhaltsverzeichnis}
\addcontentsline{toc}{chapter}{Inhaltsverzeichnis}
\tableofcontents

\chapter*{Vorwort}

\section*{Über dieses Skriptum}
Dies ist ein erweiterter Mitschrieb der Vorlesung "`Analysis II"' von Herrn Schmoeger im
Sommersemester 2005 (bis einschließlich §14) und im Sommersemester 2010 (ab §15) an der Universität
Karlsruhe (KIT). Die Mitschriebe der Vorlesung werden mit ausdrücklicher Genehmigung
von Herrn Schmoeger hier veröffentlicht, Herr Schmoeger ist für den Inhalt nicht
verantwortlich.

\section*{Wer}
Gestartet wurde das Projekt von Joachim Breitner. Beteiligt am
Mitschrieb (von 2005) sind außer Joachim
noch Pascal Maillard, Wenzel Jakob und andere.
Beteiligt am Mitschrieb (von 2010) sind Rebecca Schwerdt, Philipp Ost und Manuel Kaiser.

Im September 2012 wurde das Skript mit der Revisionsnummer 7255 von
\href{http://svn.nomeata.de/wsvn/mitschriebwiki/SS10/Ana2Bachelor.tex?op=log&}{mitschriebwiki}
auf \href{https://github.com/MartinThoma/LaTeX-examples/blob/master/documents/Analysis%20II}{GitHub} hochgeladen.

\section*{Wo}
Alle Kapitel inklusive \LaTeX-Quellen können unter
\href{http://mitschriebwiki.nomeata.de}{mitschriebwiki.nomeata.de}
abgerufen werden.
Dort ist ein \emph{Wiki} eingerichtet und von Joachim Breitner um die
\LaTeX-Funktionen erweitert.
Das heißt, jeder kann Fehler nachbessern und sich an der Entwicklung
beteiligen. Auf Wunsch ist auch ein Zugang über \emph{Subversion} möglich.

Oder man geht auf \href{https://github.com/MartinThoma/LaTeX-examples/blob/master/documents/Analysis%20II/}{github},
erstellt einen Fork und kann direkt Änderungen umsetzen.


\renewcommand{\thechapter}{\arabic{chapter}}
\renewcommand{\chaptername}{§}
\setcounter{chapter}{0}

\chapter{Der Raum $\MdR^n$}

Sei $n\in\MdN$. $\MdR^n=\{(x_1, \ldots, x_n) : x_1,\ldots, x_n \in \MdR\}$ ist mit der üblichen Addition und Skalarmultiplikation ein reeller Vektorraum.\\
$e_1 := (1,0,\ldots,0),\ e_2:=(0,1,0,\ldots, 0),\ \ldots,\ e_n:=(0,\ldots,0,1) \in \MdR^n$.

\begin{definition}
Seien $x=(x_1, \ldots, x_n), y=(y_1, \ldots, y_n) \in \MdR^n$
\begin{enumerate}
\item $x\cdot y := xy := x_1y_1+\cdots+x_ny_n$ heißt das \textbf{Skalar}\indexlabel{Skalarprodukt}- oder \begriff{Innenprodukt} von $x$ und $y$.
\item $\|x\|=(x\cdot x)^\frac{1}{2} = (x_1^2 + \cdots + x_n^2)^\frac{1}{2}$ heißt die \begriff{Norm} oder \begriff{Länge} von $x$.
\item \indexlabel{Abstand!zwischen zwei Vektoren}$\|x-y\|$ heißt der \textbf{Abstand} von $x$ und $y$.
\end{enumerate}
\end{definition}

\begin{beispiele}
\item $\|e_j\|=1\ (j=1,\dots,n)$
\item $n=3: \|(1,2,3)\|=(1+4+9)^{\frac{1}{2}}=\sqrt{14}$
\end{beispiele}

\textbf{Beachte: }
\begin{enumerate}
\item $x \cdot y \in \MdR$
\item $\|x\|^2=x \cdot x$
\end{enumerate}

\begin{satz}[Rechenregeln zur Norm]
Seien $x,y,z \in \MdR^n,\ \alpha, \beta \in \MdR,\ x=(x_1, \ldots, x_n),\ y=(y_1, \ldots, y_n)$
\begin{enumerate}
\index{Cauchy-!Schwarzsche Ungleichung}
\item $(\alpha x + \beta y)\cdot z=\alpha(x\cdot z)+\beta(y \cdot z),\ x(\alpha y + \beta z)=\alpha(xy)+\beta(xz)$
\item $\|x\|\ge 0; \|x\|=0\equizu x=0$
\item $\|\alpha x\|=|\alpha|\|x\|$
\item $|x \cdot y|\le\|x\| \|y\|$ \textbf{Cauchy-Schwarzsche Ungleichung} (\begriff{CSU})
\item $\|x+y\|\le\|x\|+\|y\|$
\item ${\left|\|x\|-\|y\|\right|}\le \|x-y\|$
\item $|x_j|\le\|x\|\le |x_1|+|x_2|+\ldots+|x_n|\ (j=1,\ldots,n)$
\end{enumerate}
\end{satz}

\begin{beweise}
\item[(1)], (2), (3)\ nachrechnen.
\item[(6)] Übung.
\item[(4)] O.B.d.A: $y\ne0$ also $\|y\|>0$. $a:=x\cdot x=\|x\|^2,\ b:=xy,\ c:=\|y\|^2=y\cdot y,\ \alpha:=\frac{b}{c}.\ 0\le\sum_{j=1}^n(x_j-\alpha y_j)^2=\sum_{j=1}^n(x_j^2-2\alpha x_jy_j+\alpha^2y_j^2)=a-2\alpha b + \alpha^2 c=a-2\frac{b}{c}b+\frac{b^2}{c^2}c=a-\frac{b^2}{c}\folgt0\le ac-b^2\folgt b^2\le ac\folgt(xy)^2\le\|x\|^2\|y\|^2$.
\item[(5)] $\|x+y\|^2=(x+y)(x+y)\gleichnach{(1)}x\cdot x + 2xy+y \cdot y=\|x\|^2+2xy+\|y\|^2\le\|x\|^2+2|xy|+\|y\|^2\overset{\text{(4)}}{\le}\|x\|^2+2\|x\|\|y\|+\|y\|^2=(\|x\|+\|y\|)^2$.
\item[(7)] $|x_j|^2=x_j^2\le x_1^2 + \ldots + x_n^2 = \|x\|^2\folgt$ 1. Ungleichung; $x=x_1e_1+\ldots+x_ne_n\folgt\|x\|=\|x_1e_1+\ldots+x_ne_n\|\overset{(5)}{\le}\|x_1e_1\|+\ldots+\|x_ne_n\|=|x_1|+\ldots+|x_n|$
\end{beweise}

Seien $p,q,l \in \MdN$. Es sei $A$ eine reelle $p${\tiny x}$q$-Matrix.

\[
    A = \begin{pmatrix}
    \alpha_{11} & \cdots & \alpha_{1q}\\
    \vdots & & \vdots\\
    \alpha_{p1} & \cdots & \alpha_{pq}
    \end{pmatrix}\qquad \|A\|:=\left(\sum_{j=1}^p\sum_{k=1}^q\alpha^2_{jk}\right)^\frac{1}{2} \text{\textbf{Norm} von A}
\]
Sei $B$ eine reelle $q${\tiny x}$l$-Matrix ($\folgt AB$ existiert). \textbf{Übung}: $\|AB\|\le\|A\|\|B\|$\\
Sei $x=(x_1,\ldots,x_q) \in \MdR^q$. $Ax:=A\begin{pmatrix}x_1\\ \vdots \\ x_q\end{pmatrix}$ (\begriff{Matrix-Vektorprodukt}). \\
Es folgt: \[\|Ax\|\le\|A\|\|x\|\]

\begin{definition}
Sei $x_0 \in \MdR^n$, $\delta > 0$, $A, U\subseteq \MdR^n$.
\begin{enumerate}
 \item $U_\delta(x_0) := \{ x \in \MdR^n: \|x-x_0\|<\delta\}$ heißt $\delta$-Umgebung von $x_0$ oder \begriff{offene Kugel} um $x_0$ mit Radius $\delta$.
 \item $U$ ist eine \begriff{Umgebung} von $x_0$ $:\equizu$ $\exists \delta > 0 : U_\delta(x_0) \subseteq U$.
 \item \indexlabel{Beschränktheit!einer Menge}$A$ heißt \textbf{beschränkt} $:\equizu$ $\exists c \ge 0: \|a\|\le c \forall a\in A$.
 \item $x_0\in A$ heißt ein \begriff{innerer Punkt} von A $:\equizu$ $\exists \delta>0: U_\delta(x_0) \subseteq A$. \\
   $A^\circ:=\{ x\in A: x \text{ ist innerer Punkt von }A\}$ heißt das \indexlabel{Inneres einer Menge}\textbf{Innere} von A. Klar: $A^\circ\subseteq A.$
 \item $A$ heißt offen $:\equizu$ $A=A^\circ$. Zur Übung: $A^\circ$ ist offen.
\end{enumerate}
\end{definition}

\begin{beispiele}
 \item offene Kugeln sind offen, $\MdR^n$ ist offen, $\emptyset$ ist offen.
 \item $A=\{x\in\MdR^n: \|x-x_0\|\le \delta\}$, $A^\circ = U_\delta(x_0)$
 \item $n=2$: $A=\{(x_1,x_2)\in\MdR^n: x_2 = x_1^2\}$, $A^\circ=\emptyset$
\end{beispiele}

\begin{definition}
 $A\subseteq \MdR^n$
 \begin{enumerate}
 \item $x_0\in \MdR^n$ heißt ein \begriff{Häufungspunkt} (HP) von $A$ $:\equizu$ $\forall \delta > 0: (U_\delta(x_0) \backslash \{x_0\}) \cap A \ne \emptyset$. $\H(A) := \{ x\in\MdR^n: x \text{ ist Häufungspunkt von } A\}$.
 \item  $x_0\in\MdR^n$ heißt ein \begriff{Berührungspunkt} (BP) von $A$ $:\equizu$ $\forall\delta>0: U_\delta(x_0) \cap A \ne \emptyset$. $\bar{A}:=\{x\in\MdR^n: x \text{ ist ein Berührungspunkt von } A\}$ heißt der \begriff{Abschluss} von $A$.\\
 Klar: $A\subseteq\bar{A}$. Zur Übung: $\bar{A} = A \cup \H(A)$.
 \item \indexlabel{abgeschlossen!Menge}$A$ heißt \textbf{abgeschlossen} $:\equizu$ $A=\bar{A}$. Zur Übung: $\bar{A}$ ist abgeschlossen.
 \item $x_0\in\MdR^n$ heißt ein \begriff{Randpunkt} von $A$ $:\equizu$ $\forall\delta>0: U_\delta(x_0) \cap A \ne \emptyset$ und $U_\delta(x_0) \cap (\MdR^n\backslash A) \ne \emptyset$. $\partial A := \{x\in\MdR^n: x \text{ ist ein Randpunkt von } A \}$ heißt der \begriff{Rand} von $A$. Zur Übung: $\partial A = \bar{A}\backslash A^\circ$.
 \end{enumerate}
\end{definition}

\begin{beispiele}
\index{abgeschlossen!Kugel}
\item $\MdR^n$ ist abgeschlossen, $\emptyset$ ist abgeschlossen; \\
  $\bar{A}=\overline{U_\delta(x_0)} = \{x\in\MdR^n: \|x-x_0\| \le \delta\}$ (\textbf{abgeschlossene Kugel} um $x_0$ mit Radius $\delta$)
\item $\partial U_\delta(x_0) = \{x\in\MdR^n: \|x-x_0\|=\delta\} = \partial \overline{U_\delta(x_0)}$
\item $A = \{(x_1,x_2)\in\MdR^2; x_2 = x_1^2\}$. $A=\bar{A}=\partial A$
\end{beispiele}

\begin{satz}[Offene und abgeschlossene Mengen]
 \begin{enumerate}
 \item Sei $A\subseteq\MdR^n$. $A$ ist abgeschlossen $:\equizu$  $\MdR^n\backslash A$ ist offen.
 \item Die Vereinigung offener Mengen ist offen.
 \item Der Durchschnitt abgeschlossener Mengen ist abgeschlossen.
 \item Sind $A_1,\ldots,A_n\subseteq\MdR^n$ offen $\folgt$ $\bigcap_{j=1}^nA_j$ ist offen
 \item Sind $A_1,\ldots,A_n\subseteq\MdR^n$ abgeschlossen $\folgt$ $\bigcup_{j=1}^nA_j$ ist abgeschlossen
 \end{enumerate}
\end{satz}

\begin{beispiel}
  $(n=1)$. $A_t := (0,1+t)\ (t>0)$. Jedes $A_t$ ist offen. $\bigcap_{t>0}A_t = (0,1]$ ist nicht offen.
\end{beispiel}

\begin{beweise}
 \item "`\folgt"': Sei $x_0\in\MdR^n\backslash A$. Annahme: $\forall \delta>0: U_\delta(x_0) \nsubseteq \MdR^n\backslash A\folgt\forall\delta>0: U_\delta(x_0)\cap A\ne\emptyset \folgt x_0\in\bar{A} \gleichnach{Vor.} A$, Widerspruch \\
 "`$\impliedby$"': Annahme: $\subset\bar{A} \folgt \ \exists x_0\in\bar{A}: x_0\notin A$; also $x_0\in\MdR^n\backslash A$. Voraussetzung $\folgt \ \exists \delta > 0: U_\delta(x_0) \subseteq \MdR^n\backslash A \folgt U_\delta(x_0) \cap A = \emptyset \folgt x_0 \notin \bar{A}$, Widerspruch!
 \item Sei $(A_\lambda)_{\lambda\in M}$ eine Familie offener Mengen und $V := \bigcup_{\lambda\in M} A_\lambda$. Sei $x_0\in V \folgt \exists\lambda_0\in M: x_0 \in A_{\lambda_0}$. $A_{\lambda_0}$ offen $\folgt \ \exists \delta > 0: U_\delta(x_0) \subseteq A_{\lambda_0} \subseteq V$
 \item folgt aus (1) und (2) (Komplemente!)
 \item $D:=\bigcap_{j=1}^mA_j$. Sei $x_0\in D$. $\forall j\in\{1,\ldots,m\}: x_0\in A_j$, also eixistiert $\delta_j>0: U_\delta(x_0)\subseteq A_j$. $\delta := \min\{\delta_j,\ldots,\delta_m\} \folgt U_\delta(x_0) \subseteq D$
 \item folgt aus (1) und (4)
\end{beweise}

\chapter{Konvergenz im $\MdR^n$}

Sei $(a^{(k)})$ eine Folge in $\MdR^n$, also $(a^{(k)}) = ( a^{(1)}, a^{(2)}, \ldots ) $ mit $a^{(k)} = (a_1^{(k)}, \ldots a_n^{(k)}) \in \MdR^n$. Die Begriffe \begriff{Teilfolge} und \begriff{Umordnung} definiert man wie in Analysis I. $(a^{(k)})$ heißt beschränkt $:\equizu$ $\exists c\ge0: \|a^{(k)}\| \le c  \ \forall k\in\MdN$.

\begin{definition*}[Grenzwert und Beschränktheit]
    \indexlabel{Konvergenz}$(a^{(k)})$ heißt \textbf{konvergent}
    $:\equizu$ $\exists a\in\MdR^n: \|a^{(k)} - a\| \to 0 \ (k\to\infty)$ ($\equizu\ \exists a\in\MdR^n: \forall \ep>0\exists k_0 \in\MdN: \|a^{(k)} - a\|<\ep \ \forall k\ge k_0$).
    In diesem Fall heißt $a$ der \begriff{Grenzwert} (GW) oder \begriff{Limes} von $(a^{(k)})$ und man schreibt:
    $a=\lim_{k\to\infty}a^{(k)}$ oder $a^{(k)} \to a \ (k\to\infty)$
\end{definition*}

\begin{beispiel}
$(n=2)$: $a^{(k)} = (\frac{1}{k}, 1+\frac{1}{k^2})$ (Erinnerung: $\frac{1}{n}$ konvergiert gegen 0); $a := (0,1)$; $\|a^{(k)} - a \| = \|(\frac{1}{k} , \frac{1}{k^2})\| = (\frac{1}{k^2} + \frac{1}{k^4})^\frac{1}{2} \to 0 \folgt a^{(k)} \to (0,1)$
\end{beispiel}
%satz 2.1
\begin{satz}[Konvergenz]
Sei $(a^{(k)})$ eine Folge in $\MdR^n$.
\begin{enumerate}
 \item Sei $a^{(k)} = (a_1^{(k)}, \ldots, a_n^{(k)})$ und
       $a = (a_1,\ldots,a_n)\in\MdR^n$. Dann:
       \[ a^{(k)} \to a \ (k\to\infty) \equizu a_1^{(k)} \to a_1, \ldots, a_n^{(k)} \to a_n \ (k\to\infty) \]
 \item Der Grenzwert einer konvergenten Folge ist eindeutig bestimmt.
 \item Ist $(a^{(k)})$ konvergent $\folgt \ a^{(k)}$ ist beschränkt
       und jede Teilfolge und jede Umordnung von $(a^{(k)})$ konvergiert gegen $\lim a^{(k)}$.
 \item Sei $(b^{(k)})$ eine weitere Folge, $a,b\in\MdR^n$ und $\alpha\in\MdR$.
       Es gelte $a^{(k)}\to a$, $b^{(k)} \to b$ Dann:
      \begin{align*}
          \|a^{(k)}\|         &\to \|a\|\\
          a^{(k)} + b ^{(k)}  &\to a+b\\
          \alpha a^{(k)}      &\to \alpha a\\
          a^{(k)}\cdot b^{(k)}&\to a\cdot b
      \end{align*}
 \item \begriff{Bolzano-Weierstraß}: Ist $(a^{(k)})$ beschränkt, so enthält $(a^{(k)})$ eine konvergente Teilfolge.
 \item \indexlabel{Cauchy!-Kriterium}\textbf{Cauchy-Kriterium}:
    $(a^{(k)})$ konvergent $\equizu \ \forall\ep>0\ \exists k_0\in\MdN: \|a^{(k)} - a^{(l)}\| <\ep \ \forall k,l \ge k_0$
\end{enumerate}
\end{satz}

\begin{beweise}
  \item 1.1(7) $\folgt |a_j^{(k)} - a_j| \le \|a^{(k)}-a\| \le \sum_{i=1}^n|a_j^{(k)} - a_j| \folgt $ Behauptung.
  \item und
  \item wie in Analysis I.
  \item folgt aus (1)
  \item Sei $(a^{(k)})$ beschränkt. O.B.d.A: $n=2$. Also $a^{(k)}=(a_1^{(k)},a_2^{(k)})$ 1.1(7) $\folgt |a_1^{(k)}|,|a_2^{(k)}|\le\|a^{(k)}\|\ \forall k\in\MdN \folgt (a_1^{(k)},a_2^{(k)})$ sind beschränkte Folgen in $\MdR$. Analysis 1 $\folgt (a_1^{(k)})$ enthält eine konvergente Teilfolge $(a_1^{(k_j)})$. $(a_2^{(k_j)})$ enthält eine konvergente Teilfolge$ (a_2^{(k_{j_l})})$. Analysis 1 $\folgt (a_1^{(k_{j_l})})$ ist konvergent $\overset{(1)}{\folgt} (a^{(k_{j_l})})$ konvergiert.
  \item "`$\folgt$"': wie in Analysis 1. "`$\impliedby$"': 1.1(7) $\folgt |a_j^{(k)}-a_j^{(l)}| \le \|a^{(k)}-a^{(l)}\|\ (j=1,\ldots,n)\ \folgt$ jede Folge $(a_j^{(k)})$ ist eine Cauchyfolge in $\MdR$, also konvergent $\overset{(1)}{\folgt} (a^{(k)})$ konvergiert.
\end{beweise}

\begin{satz}[Häufungswerte und konvergente Folgen]
Sei $A\subseteq\MdR^n$
\begin{enumerate}
\item $x_0 \in \H(A)\equizu\ \exists$ Folge $(x^{(k)})$ in $A\ \backslash\ \{x_0\}$ mit $x^{(k)}\to x_0$.
\item $x_0 \in \bar A\equizu\ \exists$ Folge $(x^{(k)})$ in $A$ mit $x^{(k)}\to x_0$.
\item $A$ ist abgeschlossen $\equizu$ der Grenzwert jeder konvergenten Folge in $A$ gehört zu $A$.
\item Die folgenden Aussagen sind äquivalent:
    \begin{enumerate}
        \item $A$ ist beschränkt und abgeschlossen
        \item Jede Folge in $A$ enthält eine konvergente Teilfolge, deren Grenzwert zu $A$ gehört.
        \item A ist kompakt
    \end{enumerate}
\end{enumerate}
\end{satz}

\begin{beweise}
\item Wie in Analysis 1
\item Fast wörtlich wie bei (1)
\item[(4)] Wörtlich wie in Analysis 1
\item[(3)] "`$\folgt$"': Sei $(a^{(k)})$ eine konvergente Folge in $A$ und $x_0:=\lim a^{(k)} \overset{(2)}{\folgt} x_0 \in \bar A \overset{\text{Vor.}}{=}A$. "`$\impliedby$"': z.z: $\bar A \subseteq A$. Sei $x_0 \in \bar A \overset{(2)}{\folgt}x_0 \in A$. Also: $A=\bar A$.
\end{beweise}

\begin{satz}[Überdeckungen]
$A \subseteq \MdR^n$ sei abgeschlossen und beschränkt
\begin{enumerate}
\item Ist $\ep>0\folgt\ \exists a^{(1)},\ldots,a^{(m)} \in A: A\subseteq \displaystyle\bigcup_{j=1}^m U_\ep(a^{(j)})$
\item $\exists$ abzählbare Teilmenge $B$ von $A: \bar B=A$.
\item \begriff{Überdeckungssatz von Heine-Borel}: Ist $(G_\lambda)_{\lambda \in M}$ eine Familie offener Mengen mit $A \subseteq \displaystyle\bigcup_{\lambda \in M} G_\lambda$, dann existieren $\lambda_1, \ldots, \lambda_m \in M: A\subseteq \displaystyle\bigcup_{j=1}^m G_{\lambda_j}$.
\end{enumerate}
\end{satz}

\begin{beweise}
\item Sei $\ep>0$. Annahme: Die Behauptung ist falsch. Sei $a^{(1)}\in A$. Dann: $A\nsubseteq U_{\ep}(a^{(1)})\folgt\exists a^{(2)}\in A: a^{(2)}\notin U_\ep(a^{(1)})\folgt\|a^{(2)}-a^{(1)}\|\ge\ep$. $A\nsubseteq U_\ep(a^{(1)})\cup U_\ep(a^{(2)})\folgt\exists a^{(3)} \in A: \|a^{(3)}-a^{(2)}\|\ge\ep,\ \|a^{(3)}-a^{(1)}\|\ge\ep$ etc.. Wir erhalten so eine Folge $(a^{(k)})$ in A: $\|a^{(k)}-a^{(l)}\|\ge\ep$ für $k\ne l$. 2.2(4) $\folgt (a^{(k)})$ enthält eine konvergente Teilfolge $\folgtnach{2.1(6)}\ \exists j_0 \in\MdN:\ \|a^{(k_j)}-a^{(k_l)}\|<\ep\ \forall j,l\ge j_0$, Widerspruch!
\item Sei $j\in\MdN$. $\ep:=\frac{1}{j}$. (1) $\folgt\exists$ endl. Teilmenge $B_j$ von $A$ mit $(*)\ A\subseteq \displaystyle\bigcup_{x \in B_j}U_{\frac{1}{j}}(x)$. $B:=\displaystyle\bigcup_{j\in\MdN}B_j\folgt B\subseteq A$ und $B$ ist abzählbar. Dann: $\bar B\subseteq\bar A\gleichnach{Vor.}A$. Noch zu zeigen: $A\subseteq\bar B$. Sei $x_0\in A$ und $\delta>0$: zu zeigen: $U_\delta(x_0)\cap B\ne\emptyset$. Wähle $j\in\MdN$ so, dass $\frac{1}{j}<\delta\ (*)\folgt\exists x \in B_j\subseteq B:\ x_0\in U_{\frac{1}{j}}(x)\folgt \|x_0-x\|<\frac{1}{j}<\delta\folgt x\in U_\delta(x_0)\folgt x\in U_\delta(x_0)\cap B$.
\item Teil 1: Behauptung: $\exists \ep>0:\ \forall a \in A\ \exists\lambda\in M: U_\ep(a)\subseteq G_\lambda$. Beweis: Annahme: Die Behauptung ist falsch. $\forall k\in\MdN\ \exists a^{(k)}\in A:\ (**) U_{\frac{1}{k}}(a^{(k)})\nsubseteq G_\lambda\ \forall \lambda\in M$. 2.2(4) $\folgt (a^{(k)})$ enthält eine konvergente Teilfolge $(a^{(k_j)})$ und $x_0:=\displaystyle\lim_{j\to\infty}a^{k_j}\in A\folgt\exists \lambda_0\in M: x_0 \in G_{\lambda_0};\ G_{\lambda_0}$ offen $\folgt\exists \delta>0: U_\delta(x_0)\subseteq G_{\lambda_0}.\ a^{(k_j)}\to x_0\ (j\to\infty)\folgt\exists m_0\in\MdN: a^{(m_0)}\in U_{\frac{\delta}{2}}(x_0)$ und $m_0\ge\frac{2}{\delta}$. Sei $x\in U_{\frac{1}{m_0}}(a^{(m_0)})\folgt \|x-x_0\|=\|x-a^{(m_0)}+a^{(m_0)}-x_0\|\le\|x-a^{(m_0)}\|+\|a^{(m_0)}-x_0\|\le\frac{1}{m_0}+\frac{\delta}{2}\le\frac{\delta}{2}+\frac{\delta}{2}=\delta\folgt x\in U_\delta(x_0)\folgt x \in G_{\lambda_0}$. Also: $U_{\frac{1}{m_0}}(a^{(m_0)})\subseteq G_{\lambda_0}$, Widerspruch zu $(**)$!\\
Teil 2: Sei $\ep>0$ wie in Teil 1. (1) $\folgt\exists a^{(1)},\ldots,a^{(m)}\in A: A\subseteq\displaystyle\bigcup_{j=1}^mU_\ep(a^{(j)})$. Teil 1 $\folgt\exists \lambda_j\in M: U_\ep(a^{(j)})\subseteq G_{\lambda_j}\ (j=1,\ldots,m)\folgt A\subseteq \displaystyle\bigcup_{j=1}^m G_{\lambda_j}$
\end{beweise}

\chapter{Grenzwerte bei Funktionen, Stetigkeit}

\begin{vereinbarung}
\indexlabel{vektorwertige Funktion}Stets in dem Paragraphen: Sei $\emptyset \ne D \subseteq \MdR^n$ und $f: D\to\MdR^m$ eine (\textbf{vektorwertige}) Funktion. Für Punkte $(x_1, x_2) \in \MdR^2$ schreiben wir auch $(x,y)$. Für Punkte $(x_1, x_2, x_3) \in\MdR^3$ schreiben wir auch $(x, y, z)$. Mit $x=(x_1,\ldots,x_n)\in D$ hat $f$ die Form $f(x)=f(x_1,\ldots,x_n)=(f_1(x_1,\ldots,x_n),\ldots,f_m(x_1,\ldots,x_n))$, wobei $f_j:D\to\MdR\ (j=1,\ldots,m)$. Kurz: $f=(f_1,\ldots,f_m)$.
\end{vereinbarung}

\begin{beispiele}
\item $n=2,m=3$. $f(x,y)=(x+y,xy,xe^y);\ f_1(x,y)=x+y, f_2(x,y)=xy, f_3(x,y)=xe^y$.
\item $n=3,m=1$. $f(x,y,z)=1+x^2+y^2+z^2$
\end{beispiele}

\begin{definition*}
Sei $x_0\in \H(D)$.

\begin{enumerate}
\item Sei $y_0 \in \MdR^m$. $\displaystyle\lim_{x\to x_0}f(x)=y_0 :\equizu$ für \textbf{jede} Folge $(x^{(k)})$ in $D\ \backslash\ \{x_0\}$ mit $x^{(k)}\to x_0$ gilt: $f(x^{(k)})\to y_0$. In diesem Fall schreibt man: $f(x)\to y_0(x\to x_0)$.
\item $\displaystyle\lim_{x\to x_0}f(x)$ existiert $:\equizu\ \exists y_0 \in \MdR^m: \displaystyle\lim_{x\to x_0}f(x)=y_0$.
\end{enumerate}
\end{definition*}

\begin{beispiele}
\item $f(x,y)=(x+y,xy,xe^y); \displaystyle\lim_{(x,y)\to(1,1)}f(x,y)=(2,1,e)$, denn: ist $((x_k, y_n))$ eine Folge mit $(x_k,y_k)\to(1,1)\folgtnach{2.1}x_k\to 1, y_k\to 1 \folgt x_k+y_k\to 2, x_ky_k\to 1, x_ke^{y_k}\to e\folgtnach{2.1}(x_k,y_k)\to(2,1,e)$.
\item $f(x,y)=\begin{cases}
\frac{xy}{x^2+y^2}&\text{, falls }(x,y)\ne(0,0)\\
0&\text{, falls }(x,y)=(0,0)
\end{cases}$\\
$f(\frac{1}{k},0)=0\to 0\ (k\to \infty), (\frac{1}{k},0)\to(0,0), f(\frac{1}{k},\frac{1}{k})=\frac{1}{2}\to\frac{1}{2}\ (k\to \infty), (\frac{1}{k},\frac{1}{k})\to(0,0)$, d.h $\displaystyle\lim_{(x,y)\to(0,0)}f(x,y)$ existiert nicht! \textbf{Aber}: $\displaystyle\lim_{x\to 0}(\displaystyle\lim_{y\to 0} f(x,y))=0=\displaystyle\lim_{y\to 0}(\displaystyle\lim_{x\to 0} f(x,y))$.
\end{beispiele}

\begin{satz}[Grenzwerte vektorwertiger Funktionen]
\begin{enumerate}
\item Ist $f = (f_1,\ldots,f_m)$ und $y_0 = (y_1,\ldots,y_m) \in \MdR^m$, so gilt: $f(x) \to y_0\ (x \to x_0) \equizu f_j(x) \to y_j\ (x \to x_0)\ (j=1,\ldots,m)$
\item Die Aussagen des Satzes Ana I, 16.1 und die Aussagen (1) und (2) des Satzes Ana I, 16.2 gelten sinngemäß für Funktionen von mehreren Variablen.
\end{enumerate}
\end{satz}

\begin{beweise}
\item folgt aus 2.1
\item wie in Ana I
\end{beweise}

\begin{definition*}[Stetigkeit vektorwertiger Funktionen]
\begin{enumerate}
\item \indexlabel{Stetigkeit}Sei $x_0 \in D$. $f$ heißt \textbf{stetig} in $x_0$ gdw. für jede Folge $(x^{(k)})$ in $D$ mit $(x^{(k)}) \to x_0$ gilt: $f(x^{(k)}) \to f(x_0)$. Wie in Ana I: Ist $x_0 \in D \cap \H(D)$, so gilt: $f$ ist stetig in $x_0 \equizu \displaystyle{\lim_{x \to x_0}} f(x) = f(x_0)$.
\item \indexlabel{Stetigkeit!auf einem Intervall}$f$ heißt auf $D$ stetig gdw. $f$ in jedem $x \in D$ stetig ist. In diesem Fall schreibt man: $f \in C(D,\MdR^m)\ (C(D) = C(D,\MdR)).$
\item \indexlabel{Stetigkeit!gleichmäßige}$f$ heißt auf $D$ \textbf{gleichmäßig} (glm) stetig gdw. gilt:\\
$\forall \ep>0\ \exists \delta>0: \|f(x)-f(y)\| < \ep\ \forall x,y \in D: \|x-y\| < \delta$
\item \indexlabel{Stetigkeit!Lipschitz-}$f$ heißt auf $D$  \textbf{Lipschitzstetig} gdw. gilt:\\
$\exists L\ge0: \|f(x)-f(y)\| \le L\|x-y\|\ \forall x,y \in D.$
\end{enumerate}
\end{definition*}

\begin{satz}[Stetigkeit vektorwertiger Funktionen]
\begin{enumerate}
\item Sei $x_0 \in D$ und $f = (f_1,\ldots,f_m).$ Dann ist $f$ stetig in $x_0$ gdw. alle $f_j$ stetig in $x_0$ sind. Entsprechendes gilt für "`stetig auf $D$"', "`glm stetig auf $D$"', "`Lipschitzstetig auf $D$"'.
\item Die Aussagen des Satzes Ana I, 17.1 gelten sinngemäß für Funktionen von mehreren Variablen.
\item Sei $x_0 \in D$. $f$ ist stetig in $x_0$ gdw. zu jeder Umgebung $V$ von $f(x_0)$ eine Umgebung $U$ von $x_0$ existiert mit $f(U \cap D) \subseteq V$.
\item Sei $\emptyset \ne E \subseteq \MdR^m$, $f(D) \subseteq E$, $g: E \to \MdR^p$ eine Funktion, $f$ stetig in $x_0 \in D$ und $g$ stetig in $f(x_0)$. Dann ist $g \circ f: D \to \MdR^p$ stetig in $x_0$.
\end{enumerate}
\end{satz}

\begin{beweise}
\item folgt aus 2.1
\item wie in Ana 1
\item Übung
\item wie in Ana 1
\end{beweise}

\begin{beispiele}
\item $f(x,y) := \begin{cases}
\frac{xy}{x^2+y^2}, & (x,y) \ne (0,0)\\
0,                  & (x,y) = (0,0)
\end{cases}\quad(D = \MdR^2)$

$f(\frac{1}{k},\frac{1}{k}) = \frac{1}{2} \to \frac{1}{2} \ne 0 = f(0,0) \folgt f$ ist in $(0,0)$ \emph{nicht} stetig.

\item $f(x,y) := \begin{cases}
\frac{1}{y} \sin(xy), & y \ne 0\\
x,                    & y = 0
\end{cases}$

Für $y \ne 0: |f(x,y) - f(0,0)| = \frac{1}{|y|}|\sin(xy)| \le \frac{1}{|y|}|xy| = |x|.$

Also gilt: $|f(x,y) - f(0,0)| \le |x|\ \forall (x,y) \in \MdR^2 \folgt f(x,y) \to f(0,0)\ ((x,y) \to (0,0)) \folgt f$ ist stetig in $(0,0)$.

\item Sei $\Phi \in C^1(\MdR),\ \Phi(0) = 0,\ \Phi'(0) = 2$ und $a \in \MdR$.

$f(x,y) := \begin{cases}
\frac{\Phi(a(x^2+y^2))}{x^2+y^2}, & (x,y) \ne (0,0)\\
\frac{1}{2},                      & (x,y) = (0,0)
\end{cases}$

Für welche $a \in \MdR$ ist $f$ stetig in $(0,0)$?

Fall 1: $a = 0$

$f(x,y) = 0\ \forall(x,y) \in \MdR^2\backslash\{(0,0)\} \folgt f$ ist in $(0,0)$ nicht stetig.

Fall 2: $a \ne 0$

$r := x^2 + y^2.\ (x,y) \to (0,0) \equizu \|(x,y)\| \to 0 \equizu r \to 0$, Sei $(x,y) \ne (0,0)$. Dann gilt:

$f(x,y) = \frac{\Phi(ar)}{r} = \frac{\Phi(ar) - \Phi(0)}{r - 0} = a \frac{\Phi(ar) - \Phi(0)}{ar - 0} \overset{r \to 0}{\to} a \Phi'(0) = 2a$. Das heißt: $f(x,y) \to 2a\ ((x,y)\to(0,0))$.

Daher gilt: $f$ ist stetig in $(0,0) \equizu 2a = \frac{1}{2} \equizu a = \frac{1}{4}$.
\end{beispiele}

\begin{definition*}[Beschränktheit einer Funktion]
\indexlabel{Beschränktheit!einer Funktion}
$f:D \to \MdR^m$ heißt \textbf{beschränkt} (auf $D$) gdw. $f(D)$ beschränkt ist $(\equizu \exists c \ge 0: \|f(x)\| \le c\ \forall x \in D)$.
\end{definition*}

\begin{satz}[Funktionen auf beschränkten und abgeschlossenen Intervallen]
$D$ sei beschränkt und abgeschlossen und es sei $f \in C(D,\MdR^m)$.
\begin{enumerate}
\item $f(D)$ ist beschränkt und abgeschlossen.
\item $f$ ist auf $D$ gleichmäßig stetig.
\item Ist $f$ injektiv auf $D$, so gilt: $f^{-1} \in C(f(D),\MdR^n)$.
\item Ist $m = 1$, so gilt: $\exists a,b \in D: f(a) \le f(x) \le f(b)\ \forall x \in D$.
\end{enumerate}
\end{satz}

\begin{beweis}
wie in Ana I.
\end{beweis}

\begin{satz}[Fortsetzungssatz von Tietze]
Sei $D$ abgeschlossen und $f \in C(D,\MdR^m) \folgt \exists F \in C(\MdR^n,\MdR^m): F=f$ auf $D$.
\end{satz}

\begin{satz}[Lineare Funktionen und Untervektorräume von $\MdR^n$]
\begin{enumerate}
\item Ist $f:\MdR^n \to \MdR^m$ und \emph{linear}, so gilt: $f$ ist Lipschitzstetig auf $\MdR^n$, insbesondere gilt: $f \in C(\MdR^n,\MdR^m)$.
\item Ist $U$ ein Untervektorraum von $\MdR^n$, so ist $U$ abgeschlossen.
\end{enumerate}
\end{satz}

\begin{beweise}
\item Aus der Linearen Algebra ist bekannt: Es gibt eine $(m \times n)$-Matrix $A$ mit $f(x) = Ax$. Für $x,y \in \MdR^n$ gilt: $\|f(x)-f(y)\| = \|Ax - Ay\| = \|A(x-y)\| \le \|A\|\cdot \|x-y\|$

\item Aus der Linearen Algebra ist bekannt: Es gibt einen UVR $V$ von $\MdR^n$ mit: $\MdR^n = U \oplus V$. Definiere $P: \MdR^n \to \MdR^n$ wie folgt: zu $x \in \MdR^n$ existieren eindeutig bestimmte $u \in U,\ v \in V$ mit: $x = u+v;\ P(x) := u$.

Nachrechnen: $P$ ist linear.

$P(\MdR^n) = U$ (Kern $P = V,\ P^2 = P$). Sei $(u^{(k)})$ eine konvergente Folge in $U$ und $x_0 := \lim u^{(k)}$, z.z.: $x_0 \in U$.

Aus (1) folgt: $P$ ist stetig $\folgt P(u^{(k)}) \to P(x_0) \folgt x_0 = \lim u^{(k)} = \lim P(u^{(k)}) = P(x_0) \in P(\MdR^n) = U$.
\end{beweise}

\begin{definition*}[Abstand eines Vektor zu einer Menge]
\indexlabel{Abstand!zwischen Vektor und Menge}
Sei $\emptyset \ne A \subseteq \MdR^n,\ x \in \MdR^n.\ d(x,A) := \inf\{\|x-a\|:a \in A\}$ heißt der \textbf{Abstand} von $x$ und $A$.

Klar: $d(a,A) = 0\ \forall a \in A$.
\end{definition*}

\begin{satz}[Eigenschaften des Abstands zwischen Vektor und Menge]
\begin{enumerate}
\item $|d(x,A) - d(y,A)| \le \|x-y\|\ \forall x,y \in \MdR^n$.
\item $d(x,A) = 0 \equizu x \in \overline{A}$.
\end{enumerate}
\end{satz}

\begin{beweise}
\item Seien $x,y \in \MdR^n$. Sei $a \in A$. $d(x,A) \le \|x-a\| = \|x-y+y-a\| \le \|x-y\|+\|y-a\|\\
\folgt d(x,A)-\|x-y\| \le \|y-a\|\ \forall a \in A\\
\folgt d(x,A) - \|x-y\| \le d(y,A)\\
\folgt d(x,A) - d(y,A) \le \|x-y\|$

Genauso: $d(y,A) - d(x,A) \le \|y-x\| = \|x-y\| \folgt$ Beh.
\item Der Beweis erfolgt duch Implikation in beiden Richtungen:
\begin{itemize}
\item["`$\impliedby$"':] Sei $x \in \overline{A} \folgtnach{2.2} \exists$ Folge $(a^{(k)})$ in $A: a^{(k)} \to x \folgtnach{(1)} d(a^{(k)},A) \to d(x,A) \folgt d(x,A) = 0$.
\item["`$\implies$"':] Sei $d(x,A) = 0.\ \forall k \in \MdN\ \exists a^{(k)} \in A: \|a^{(k)} - x\| < \frac{1}{k} \folgt a^{(k)} \to x \folgtnach{2.2} x \in \overline{A}$.
\end{itemize}
\end{beweise}



\chapter{Partielle Ableitungen}

Stets in diesem Paragraphen: $\emptyset\ne D\subseteq \MdR^n$, $D$ sei offen und $f:D\to\MdR$ eine reellwertige Funktion. $x_0 = (x_1^{(0)}, \ldots, x_n^{(0)}) \in D$. Sei $j\in\{1,\ldots,n\}$ (fest).

Die Gerade durch $x_0$ mit der Richtung $e_j$ ist gegeben durch folgende Menge: $\{x_0+te_j:t\in\MdR\}$. $D$ offen $\folgt$ $\exists\delta>0: U_\delta(x_0)\subseteq D$. $\|x_0+te_j - x_0\| = \|te_j\| = |t| \folgt x_0+e_j \in D $ für $t\in(-\delta,\delta)$. $g(t) := f(x_0+te_j)$ $(t\in(-\delta,\delta))$
 Es ist $g(t) = f(x_1^{(0)}, \ldots, x_{j-1}^{(0)}, x_j^{(0)} + t, x_{j+1}^{(0)}, \ldots, x_n^{(0)} )$

\begin{definition}
$f$ heißt in $x_0$ \textbf{partiell differenzierbar} nach $x_j$ :\equizu es exisitert
der Grenzwert
\[\lim_{t\to0}\frac{f(x_0+te_j) - f(x_0)}t\]
und ist $\in\MdR$. In diesem Fall heißt obiger Grenzwert die \textbf{partielle Ableitung
von $f$ in $x_0$} nach $x_j$ und man schreibt für diesen Grenzwert:
\[f_{x_j}(x_0) \text{ oder }\frac{\partial f}{\partial x_j}(x_0)\]

Im Falle $n=2$ oder $n=3$ schreibt man $f_x$, $f_y$, $f_z$ bzw. $\frac{\partial f}{\partial x}$, $\frac{\partial f}{\partial y}$, $\frac{\partial f}{\partial z}$
\end{definition}


\begin{beispiele}
\item $f(x,y,z) = xy+z^2+e^{x+y}$; $f_x(x,y,z) = y + e^{x+y} = \frac{\partial f}{\partial x}(x,y,z)$. $f_x(1,1,2)=1+e^2$. $f_y(x,y,z) = x+e^{x+y}$. $f_z(x,y,z) = 2z = \frac{\partial f}{\partial z}(x,y,z)$.
\item $f(x) = f(x_1,\ldots, x_n) = \|x\| = \sqrt{x_1^2 + \cdots + x_n^2}$.

Sei $x\ne0$: $f_{x_j}(x) = \frac{1}{2\sqrt{x_1^2 + \cdots + x_n^2}}2x_j = \frac{x_j}{\|x\|} $

Sei $x=0$: $\frac{f(t,0,\ldots,0) - f(0,0,\ldots,0)}{t} = \frac{|t|}{t} = \begin{cases} 1, &t>0\\-1,&t<0\end{cases} \folgt f$ ist in $(0,\ldots,0)$ nicht partiell differenzierbar nach $x_1$. Analog: $f$ ist in $(0,\ldots,0)$ nicht partiell differenzierbar nach $x_2,\ldots,x_n$
\item $f(x,y) = \begin{cases} \frac{xy}{x^2+y^2}, &(x,y)\ne(0,0)\\0,&(x,y) = (0,0) \end{cases}$

$\frac{f(t,0) - f(0,0)}{t} = 0 \to 0 \ (t\to0) \folgt f$ ist in $(0,0)$ partiell differenzierbar nach $x$ und $f_x(0,0) = 0$. Analog: $f$ ist in $(0,0)$ partiell differenzierbar nach $y$ und $f_y(0,0) = 0$. Aber: $f$ ist in $(0,0)$ nicht stetig.
\end{beispiele}

\def\grad{\mathop{\rm grad}\nolimits}
\begin{definition}
\indexlabel{partiell!Differenzierbarkeit}
\indexlabel{partiell!Ableitung}
\indexlabel{Differenzierbarkeit!partielle}
\indexlabel{Ableitung!partielle}
\begin{enumerate}
\item $f$ heißt in $x_0$ \textbf{partiell differenzierbar} $:\equizu$ $f$ ist in $x_0$
partiell differenzierbar nach allen Variablen $x_1,\ldots, x_n$. In diesem Fall heißt
$\grad f(x_0) := \nabla f(x_0) := (f_{x_1}(x_0),\ldots, f_{x_n}(x_0))$ der \textbf{Gradient} von
$f$ in $x_0$. \indexlabel{Gradient}
\item $f$ ist auf $D$ \textbf{partiell differenzierbar} nach $x_j$ oder $f_{x_j}$ ist auf $D$ vorhanden :\equizu $f$ ist in jedem $x\in D$ partiell differenzierbar nach $x_j$. In diesem Fall wird durch $x\mapsto f_{x_j}(x)$ eine Funktion $f_{x_j}: D\to \MdR$ definiert die \textbf{partielle Ableitung} von $f$ auf $D$ nach $x_j$.
\item $f$ heißt \textbf{partiell differenzierbar} auf $D$ :\equizu $f_{x_1},\ldots,f_{x_n}$ sind auf $D$ vorhanden.
\item $f$ heißt auf $D$ \textbf{stetig partiell differenzierbar} :\equizu $f$ ist auf $D$ partiell differenzierbar und $f_{x_1},\ldots,f_{x_n}$ sind auf $D$ stetig. In diesem Fall schreibt man $f\in C^1(D,\MdR)$.
\end{enumerate}
\end{definition}

\begin{beispiele}
\item Sei $f$ wie in obigem Beispiel (3). $f$ ist in $(0,0)$ partiell differenzierbar und $\grad f(0,0) = (0,0)$
\item Sei $f$ wie in obigem Beispiel (2). $f$ ist auf $\MdR^n\backslash\{0\}$ partiell differenzierbar und $\grad f(x) = (\frac{x_1}{\|x_n\|},\ldots,\frac{x_n}{\|x_n\|}) = \frac{1}{\|x\|} x \ (x\ne 0)$
\end{beispiele}

\begin{definition}
Seien $j,k\in\{1,\ldots,n\}$ und $f_{x_j}$ sei auf $D$ vorhanden. Ist $f_{x_j}$ in
$x_0\in D$ partiell differenzierbar nach $x_k$, so heißt
\[f_{x_jx_k}(x_0) := \frac{\partial^2 f}{\partial x_j\partial x_k}(x_0) := \left(f_{x_j}\right)_{x_k}(x_0)\]
die \textbf{partielle Ableitung zweiter Ordnung} von $f$ in $x_0$
nach $x_j$ und $x_k$. Ist $k=j$, so schreibt man:
\[\frac{\partial^2 f}{\partial x_j^2}(x_0) = \frac{\partial^2 f}{\partial x_j\partial x_j}(x_0) \]
Entsprechend definiert man partielle Ableitungen höherer Ordnung (soweit vorhanden).
\end{definition}

\begin{schreibweisen}
$\ds f_{xxyzz} = \frac{\partial^5 f}{\partial x^2\partial y\partial z^2}$, vergleiche: $\ds\frac{\partial^{180} f}{\partial x^{179}\partial y}$
\end{schreibweisen}

\begin{beispiele}
\item $f(x,y) = xy + y^2$, $f_x(x,y)=y$, $f_{xx} = 0$, $f_y = x + 2y$, $f_{yy} = 2$, $f_{xy}=1$, $f_{yx} = 1$.
\item $f(x,y,z) = xy + z^2e^x$, $f_x = y+z^2e^x$, $f_{xy} = 1$, $f_{xyz} = 0$. $f_z=2ze^x$, $f_{zy}=0$, $f_{zyx} = 0$.
\item $f(x,y) = \begin{cases} \frac{xy(x^2-y^2)}{x^2+y^2}, & (x,y) \ne (0,0) \\ 0, &(x,y)=(0,0)\end{cases}$

Übungsblatt: $f_{xy}(0,0)$, $f_{yx}(0,0)$ existieren, aber $f_{xy}(0,0) \ne f_{yx}(0,0)$
\end{beispiele}

\begin{definition}
Sei $m\in\MdN$. $f$ heißt auf $D$ \textbf{$m$-mal stetig partiell differenzierbar} :\equizu alle partiellen Ableitungen  von $f$ der Ordnung $\le m$ sind auf $D$ vorhanden und auf $D$ stetig. In diesem Fall schreibt man: $f\in C^m(D,\MdR)$

\[C^0(D, \MdR) := C(D,\MdR),\qquad C^\infty(D,\MdR) := \bigcap_{k\in\MdN_0}C^{k}(D,\MdR)\]
\end{definition}

\begin{satz}[Satz von Schwarz]
Es sei $f\in C^2(D,\MdR)$, $x_0\in D$ und $j,k\in\{1,\ldots,n\}$. Dann: $f_{x_jx_k}(x_0) = f_{x_kx_j}(x_0)$
\end{satz}

\begin{satz}[Folgerung]
Ist $f\in C^m(D,\MdR)$, so sind die partiellen Ableitungen von $f$
der Ordnung $\le m$ unabhängig von der Reihenfolge der Differentation.
\end{satz}

\begin{beweis}
O.B.d.A: $n=2$ und $x_0=(0,0)$. Zu zeigen: $f_{xy}(0,0)=f_{yx}(0,0)$. $D$
offen $\folgt\exists\delta>0: U_\delta(0,0)\subseteq D$. Sei $(x,y) \in U_\delta(0,0)$ und $x\ne 0\ne y$.
\[\nabla:=f(x,y)-f(x,0)-(f(0,y)-f(0,0)),\quad\varphi(t):=f(t,y)-f(t,0)\]
für $t$ zwischen $0$ und $x$. $\varphi$ ist differenzierbar und
$\varphi'(t)=f_x(t,y)-f_x(t,0)$. $\varphi(x)-\varphi(0)=\nabla$.
MWS, Analysis I $\folgt\exists\xi=\xi(x,y)$ zwischen $0$ und $x$:
$\nabla=x\varphi'(\xi)=x(f_x(\xi,y)-f_x(\xi,0))$. $g(s):=f_x(\xi,s)$
für s zwischen $0$ und $y$; $g$ ist differenzierbar und
$g'(s)=f_{xy}(\xi,s)$. Es ist $\nabla=x(g(y)-g(0))\gleichnach{MWS}xyg'(\eta),\ \eta=\eta(x,y)$
zwischen $0$ und $y$. $\folgt \nabla=xyf_{xy}(\xi,\eta).$ (1)\\
$\psi(t):=f(x,t)-f(0,t)$, $t$ zwischen $0$ und $y$.
$\psi'(t)=f_y(x,t)-f_y(0,t)$. $\nabla=\psi(y)-\psi(0)$.
Analog: $\exists \bar\eta=\bar\eta(x,y)$ und $\bar\xi=\bar\xi(x,y)$, $\bar\eta$
zwischen $0$ und $y$, $\bar\xi$ zwischen $0$ und $x$. $\nabla=xyf_{yx}(\bar\xi,\bar\eta).$ (2)\\
Aus (1), (2) und $xy\ne0$ folgt $f_{xy}(\xi,\eta)=f_{yx}(\bar\xi,\bar\eta)$.
$(x,y)\to(0,0)\folgt\xi,\bar\xi,\eta,\bar\eta\to 0\folgtwegen{f\in C^2}f_{xy}(0,0)=f_{yx}(0,0)$
\end{beweis}

\chapter{Differentiation}
\def\grad{\mathop{\rm grad}\nolimits}

\begin{vereinbarung}
Stets in dem Paragraphen: $\emptyset\ne D\subseteq\MdR^n$, $D$ offen und $f:D\to\MdR^m$ eine Funktion, also $f=(f_1,\ldots,f_m)$
\end{vereinbarung}

\begin{definition*}
\begin{enumerate}
\item Sei $k\in\MdN$. $f\in C^k(D,\MdR^m) :\equizu f_j\in C^k(D,\MdR)\ (j=1,\ldots,m)$
\item Sei $x_0\in D$. $f$ heißt \textbf{partiell differenzierbar} in
$x_0 :\equizu$ jedes $f_j$ ist in $x_0$ partiell differenzierbar.
In diesem Fall heißt
\[\frac{\partial f}{\partial x}(x_0):=\frac{\partial(f_1,\ldots,f_m)}{\partial(x_1,\ldots,x_n)}:=J_f(x_0):=
\begin{pmatrix}
\frac{\partial f_1}{\partial x_1}(x_0) & \cdots & \frac{\partial f_1}{\partial x_n}(x_0) \\
\vdots & & \vdots \\
\frac{\partial f_m}{\partial x_1}(x_0) & \cdots & \frac{\partial f_m}{\partial x_n}(x_0)
\end{pmatrix}\]
\indexlabel{Jacobi-Matrix}
\indexlabel{Funktionalmatrix}
die \textbf{Jacobi-} oder \textbf{Funktionalmatrix} von $f$ in $x_0$.
\end{enumerate}
\textbf{Beachte:}
\begin{enumerate}
\item $J_f(x_0)$ ist eine $(m \times n)$-Matrix.
\item Ist $m=1$ folgt $J_f(x_0)=\grad f(x_0)$.
\end{enumerate}
\end{definition*}

\begin{erinnerung}
Sei $I\subseteq\MdR$ ein Intervall, $\varphi:I\to\MdR$ eine Funktion, $x_0\in I$. $\varphi$ ist in $x_0$ differenzierbar
\[
\overset{\text{ANA 1}}{\equizu}\exists a \in\MdR: \ds\lim_{h\to 0}\frac{\varphi(x_0+h)-\varphi(x_0)}{h}=a
\equizu\exists a\in\MdR:\ds\lim_{h\to 0}\frac{\varphi(x_0+h)-\varphi(x_0)-ah}{h}=0
\equizu\exists a\in\MdR: \ds\lim_{h\to 0}\frac{\varphi(x_0+h)-\varphi(x_0)-ah}{|h|}=0
\]
\end{erinnerung}

\begin{definition*}
\begin{enumerate}
    \index{Differenzierbarkeit}
    \item Sei $x_0\in D$. $f$ heißt \textbf{differenzierbar} (db) in
          $x_0 :\equizu \exists (m \times n)$-Matrix $A$, sodass gilt:
          \begin{align*}
            \ds\lim_{h\to 0}\frac{f(x_0+h)-f(x_0)-Ah}{\|h\|}=0\ \tag{$*$}
          \end{align*}
    \item $f$ heißt differenzierbar auf $D\ :\equizu f$ ist in
          jedem $x\in D$ differenzierbar.
\end{enumerate}
\end{definition*}

\begin{bemerkungen}
    \item $f$ ist differenzierbar in
          $x_0\equizu\exists (m \times n)$-Matrix $A$:
          \[ \ds\lim_{x\to x_0}\frac{f(x)-f(x_0)-A(x-x_0)}{\|x-x_0\|}=0 \]
    \item Ist $m=1$, so gilt: $f$ ist differenzierbar in $x_0$
        \begin{align*}
            \equizu \exists a \in\MdR^n:
            \ds\lim_{h\to 0}\frac{f(x_0+h)-f(x_0)-ah}{\|h\|}=0\ \tag{$**$}
        \end{align*}
    \item Aus 2.1 folgt: $f$ ist differenzierbar in $x_0\equizu$ jedes
          $f_j$ ist differenzierbar in $x_0$.
\end{bemerkungen}

\begin{satz}[Differenzierbarkeit und Stetigkeit]
$f$ sei in $x_0\in D$ differenzierbar
\begin{enumerate}
\item $f$ ist in $x_0$ stetig
\item $f$ ist in $x_0$ partiell differenzierbar und die Matrix A in
      $(*)$ ist eindeutig bestimmt: \\
      $A=J_f(x_0)$. $f'(x_0):=A=J_f(x_0)$ (\begriff{Ableitung} von $f$ in $x_0$).
\item Ist $m=1$, so ist $f'(x_0) = a$ (aus $(**)$), also $f'(x_0) = \grad(f(x_0))$
\end{enumerate}
\end{satz}

\begin{beweis}
Sei A wie in $(*)$, $A=(a_{jk})$, $\varrho(h):=\frac{f(x_0+h)-f(x_0)-Ah}{\|h\|}$,
also: $\varrho(h)\to0\ (h\to 0)$. Sei $\varrho=(\varrho_1,\ldots,\varrho_m)$.
2.1 $\folgt \varrho_j(h)\to 0\ (h\to 0)\ (j=1,\ldots,m)$
\begin{enumerate}
\item $f(x_0+h)=f(x_0)+\underbrace{Ah}_{\overset{\text{3.5}}{\to}0}+\underbrace{\|h\|\varrho(h)}_{\to 0\ (h\to 0)}\to f(x_0)\ (h\to 0)$
\item Sei $j\in\{1,\ldots,m\}$ und $k\in\{1,\ldots,n\}$.
      Zu zeigen: $f_j$ ist partiell differenzierbar und
      $\frac{\partial f_j}{\partial x_k}(x_0)=a_{jk}$.
      $\varrho_j(h)=\frac{1}{\|h\|}(f_j(x_0+h)-f_j(x_0)-(a_{j1},\ldots,a_{jn})\cdot h)\to 0\ (h \to 0)$.
      Für $t\in\MdR$ sei $h=te_k\folgt\varrho(h)=\frac{1}{|t|}(f(x_0+te_k)-a_{jk}t)\to 0\ (t\to 0)\folgt\left|\frac{f(x_0+te_k)-f(x_0)}{t}-a_{jk}\right|\to 0\ (t\to 0)\folgt f_j$ ist in $x_0$ partiell differenzierbar und $\frac{\partial f_j}{\partial x_k}(x_0)=a_{jk}$.
\end{enumerate}
\end{beweis}

\begin{beispiele}
\item $$f(x,y)=\begin{cases}
\frac{xy}{x^2+y^2}&\text{, falls } (x,y)\ne(0,0)\\
0&\text{, falls } (x,y)=(0,0)
\end{cases}$$
Bekannt: $f$ ist in $(0,0)$ \textbf{nicht} stetig, aber partiell
differenzierbar und $\grad f(0,0)=(0,0)$ 5.1 $\folgt f$ ist in
$(0,0)$ \textbf{nicht} differenzierbar.
\item \[
    f(x,y)=
    \begin{cases}
        (x^2+y^2) \underbrace{\sin\frac{1}{\sqrt{x^2+y^2}}}_\text{beschränkt}&\text{, falls } (x,y)\ne(0,0)\\
        0&\text{, falls }(x,y)=(0,0)
    \end{cases}
\]
Für $(x,y)\ne(0,0): \left|f(x,y)\right|=(x^2+y^2)\left|\sin\frac{1}{\sqrt{x^2+y^2}}\right|
\le x^2+y^2\overset{(x,y)\to(0,0)}{\to}0 \folgt f$ ist in $(0,0)$ stetig.
$\frac{f(t,0)-f(0,0)}{t}=\frac{1}{t}t^2\sin\frac{1}{|t|}=t\sin\frac{1}{|t|}\to 0\ (t\to 0)\folgt f$
ist in $(0,0)$ partiell differenzierbar nach $x$ und $f_x(0,0)=0$.
Analog: $f$ ist in $(0,0)$ partiell differenzierbar nach $y$ und
$f_y(0,0)=0$. $\varrho(h)=\frac{1}{\|h\|}f(h)\gleichwegen{h=(h_1,h_2)}\frac{1}{\sqrt{h_1^2+h_2^2}}(h_1^2+h_2^2)\sin\frac{1}{h_1^2+h_2^2}=\sqrt{h_1^2+h_2^2}\underbrace{\sin\frac{1}{\sqrt{h_1^2+h_2^2}}}_{\text{beschränkt}}\to 0\ (h\to 0)\folgt f$
ist differenzierbar in $(0,0)$ und $f'(0,0)=\grad f(0,0)=(0,0)$

\item $$f(x,y) := \begin{cases}
\frac{x^3}{x^2+y^2}&\text{, falls} (x,y) \ne (0,0)\\
0&\text{, falls} (x,y) = (0,0)\end{cases}$$

Übung: $f$ ist in $(0,0)$ stetig.

$\frac{f(t,0) - f(0,0)}{t} = \frac{1}{t} \frac{t^3}{t^2} = 1 \to 1\ (t \to 0).\ \frac{f(0,t) - f(0,0)}{t} = 0 \to 0\ (t \to 0)$.

$\folgt f$ ist in $(0,0)$ partiell db und $\grad f(0,0) = (1,0)$.

Für $h = (h_1,h_2) \ne (0,0): \rho(h) = \frac{1}{\|h\|}(f(h) - f(0,0) - \grad f(0,0)\cdot h) = \frac{1}{\|h\|} (\frac{h_1^3}{h_1^2+h_2^2} - h_1) = \frac{1}{\|h\|} \frac{-h_1 h_2^2}{h_1^2 + h_2^2} =  \frac{-h_1 h_2^2}{(h_1^2 + h_2^2)^{3/2}}.$

Für $h_2 = h_1 > 0: \rho(h) = \frac{-h_1^3}{(\sqrt{2})^3 h_1^3} = - \frac{1}{(\sqrt{2})^3} \folgt \rho(h) \nrightarrow 0\ (h \to 0) \folgt f$ ist in $(0,0)$ \emph{nicht} db.
\end{beispiele}

\begin{satz}[Stetigkeit aller partiellen Ableitungen]
Sei $x_0 \in D$ und \emph{alle} partiellen Ableitungen
$\frac{\partial f_j}{\partial x_k}$ seien auf $D$ vorhanden und in
$x_0$ stetig $(j=1,\ldots,m,\ k=1,\ldots,n)$. Dann ist $f$ in $x_0$
db.
\end{satz}

\begin{beweis}
O.B.d.A: $m=1$ und $x_0=0$. Der Übersicht wegen sei $n=2$.

Für $h = (h_1,h_2) \ne (0,0):$ $$\rho(h) := \frac{1}{\|h\|}(f(h) - f(0,0) - (\underbrace{h_1 f_x(0,0) + h_2 f_y(0,0)}_{= \grad f(0,0)\cdot h}))$$

$f(h) - f(0) = f(h_1,h_2) - f(0,0) = \underbrace{f(h_1,h_2) - f(0,h_2)}_{=:\Delta_1} + \underbrace{f(0,h_2) - f(0,0)}_{=:\Delta_2}$

$\varphi(t) := f(t,h_2),\ t$ zwischen $0$ und $h_1 \folgt \Delta_1 = \varphi(h_1) - \varphi(0),\ \varphi'(t) = f_x(t,h_2)$

Aus dem Mittelwertsatz aus Analysis I folgt:
$\exists \xi = \xi(h)$ mit $0 \leq \xi \leq h_1: \Delta_1 = h_1\varphi(\xi) = h_1 f_x(\xi,h_2)\\
\exists \eta = \eta(h)$ zw. $0$ und $h_2: \Delta_2 = h_2\varphi(\eta) = h_2 f_x(\eta,h_2)$

$\folgt \rho(h) := \frac{1}{\|h\|}(h_1 f_x(\xi,h_2) - h_2 f_y(0,\eta) - (h_1 f_x(0,0) + h_2 f_y(0,0)))\\
= \frac{1}{\|h\|} h(\underbrace{f_x(\xi,h_2) - f_x(0,0),\ f_y(0,\eta) - f_y(0,0)}_{=:v(h)})
= \frac{1}{\|h\|} h\cdot v(h)$

$\folgt |\rho(h)| = \frac{1}{\|h\|} |h\cdot v(h)| \overset{\text{CSU}}{\le} \frac{1}{\|h\|} \|h\| \|v(h)\| = \|v(h)\|$

$f_x,f_y$ sind stetig in $(0,0) \folgt v(h) \to 0\ (h \to 0) \folgt \rho(h) \to 0\ (h \to 0)$
\end{beweis}

\begin{folgerung}
Ist $f \in C^1(D,\MdR^m) \folgt f$ ist auf $D$ db.
\end{folgerung}

\begin{definition*}
Sei $k \in \MdN$ und $f \in C^k(D,\MdR^m)$.
Dann heißt $f$ \textbf{auf $D$ $k$-mal stetig db}.
\end{definition*}

\begin{beispiele}
\item $f(x,y,z) = (x^2+y, xyz).\ J_f(x,y,z) = \begin{pmatrix}
2x & 1 & 0\\
yz & xz & xy\end{pmatrix} \folgt f \in C^1(\MdR^3,\MdR^2)$

$\folgtnach{5.3} f$ ist auf $\MdR^3$ db und $f'(x,y,z) = J_f(x,y,z)\ \forall (x,y,z) \in \MdR^3.$

\item Sei $f:\MdR^n \to \MdR^m$ \emph{linear}, es ex. also eine $(m \times n)$-Matrix $A:f(x) = Ax\ (x \in \MdR^n).$

Für $x_0 \in \MdR^n$ und $h \in \MdR^n \backslash\{0\}$ gilt:\\
$\rho(h) = \frac{1}{\|h\|}(f(x_0+h) - f(x_0) - Ah) = \frac{1}{\|h\|}(f(x_0) + f(h) - f(x_0) - f(h)) = 0.$

Also: $f$ ist auf $\MdR^n$ db und $f'(x) = A\ \forall x \in \MdR^n$. Insbesondere ist $f \in C^1(\MdR^n,\MdR^m).$

\item[(2.1)] $n = m$ und $f(x) = x = Ix$ ($I = (m \times n)$-Einheitsmatrix). Dann: $f'(x) = I\ \forall x \in \MdR^n$.

\item[(2.2)] $m = 1:\ \exists a \in \MdR^n: f(x) = ax\ (x \in \MdR^n)$ (Linearform). $f'(x) = a\ \forall x \in \MdR^n$.

\item $$f(x,y) = \begin{cases}
(x^2+y^2) \sin \frac{1}{\sqrt{x^2+y^2}} & \text{, falls} (x,y) \ne (0,0)\\
0 & \text{, falls} (x,y) = (0,0)\end{cases}$$

Bekannt: $f$ ist in $(0,0)$ db. Übungsblatt: $f_x,f_y$ sind in
$(0,0)$ \emph{nicht} stetig.

\item Sei $I \subseteq \MdR$ ein Intervall und
$g = (g_1,\ldots,g_m): I \to \MdR^m;\ g_1,\ldots,g_m: I \to \MdR.$

$g$ ist in $t_0 \in I$ db $\equizu g_1,\ldots,g_m$ sind in $t_0 \in I$ db. In diesem Fall gilt: $g'(t_0) = (g_1'(t_0),\ldots,g_m'(t_0)).$

\item[(4.1)] $m = 2: g(t) = (\cos t,\sin t),\ t \in [0,2\pi].\ g'(t) = (-\sin t,\cos t).$
\item[(4.2)] Seien $a,b \in \MdR^m,\ g(t) = a+t(b-a),\ t \in [0,1],\ g'(t) = b-a$.
\end{beispiele}

\begin{satz}[Kettenregel]
$f$ sei in $x_0 \in D$ db, $\emptyset \ne E \subseteq \MdR^m,\ E$ sei offen, $f(D) \subseteq E$ und $g:E \to \MdR^p$ sei db in $y_0 := f(x_0)$. Dann ist $g \circ f: D \to \MdR^p$ db in $x_0$ und $$(g \circ f)'(x_0) = g'(f(x_0))\cdot f'(x_0)\text{ (Matrizenprodukt)}$$
\end{satz}

\begin{beweis}
$A := f'(x_0),\ B := g'(y_0) = g'(f(x_0)),\ h := g \circ f.$

$$\tilde{g}(y) = \begin{cases}
\frac{g(y)-g(y_0)-B(y-y_0)}{\|y-y_0\|} & \text{, falls } y \in E\backslash\{y_0\} \\
0                                      & \text{, falls } y = y_0
\end{cases}$$

$g$ ist db in $y_0 \folgt \tilde{g}(y) \to 0\ (y \to y_0).$ Aus Satz 5.1 folgt, dass $f$ stetig ist in $x_0 \folgt f(x) \to f(x_0) = y_0\ (x \to x_0) \folgt \tilde{g}(f(x)) \to 0\ (x \to x_0)$

Es ist $g(y) - g(y_0) = \|y-y_0\| \tilde{g}(y) = B(y-y_0)\ \forall y \in E.$

$\ds{\frac{h(x)-h(x_0)-BA(x-x_0)}{\|x-x_0\|} = \frac{1}{\|x-x_0\|}(g(f(x))-g(f(x_0))-BA(x-x_0))}$\\
$\ds{= \frac{1}{\|x-x_0\|} (\|f(x)-f(x_0)\| \tilde{g}(f(x)) + B(f(x)-f(x_0))-BA(x-x_0))}$\\
$\ds{= \underbrace{\frac{\|f(x)-f(x_0)\|}{\|x-x_0\|}}_{=:D(x)} \underbrace{\tilde{g}(f(x))}_{\to 0} + \underbrace{B(\underbrace{\frac{f(x)-f(x_0)-A(x-x_0)}{\|x-x_0\|}}_{\overset{f\text{ db}}{\to} 0\ (x \to x_0)})}_{\overset{\text{3.5}}{\to} 0\ (x \to x_0)}}$

Noch zu zeigen: $D(x)$ bleibt in der "`Nähe"' von $x_0$ beschränkt.

$0 \le D(x) = \ds{\frac{\|f(x)-f(x_0)-A(x-x_0)+A(x-x_0)\|}{\|x-x_0\|}}$\\
$\ds{= \underbrace{\frac{\|f(x)-f(x_0)-A(x-x_0)\|}{\|x-x_0\|}}_{\to 0\ (x \to x_0)} + \underbrace{\frac{\|A(x-x_0)\|}{\|x-x_0\|}}_{\le \|A\|}}.$
\end{beweis}

\paragraph{Wichtigster Fall}
$g = g(x_1,\ldots,x_m)$ reellwertig,
\begin{align*}
h(x) &= h(x_1,\ldots,x_n) \\
&= g(f_1(x_1,\ldots,x_n),f_2(x_1,\ldots,x_n),\ldots,f_m(x_1,\ldots,x_n)) \\
&= (g \circ f)(x)
\end{align*}

$h_{x_j}(x) = g_{x_1}(f(x))\frac{\partial f_1}{\partial x_j}(x)+g_{x_2}(f(x))\frac{\partial f_2}{\partial x_j}(x)+\cdots+g_{x_m}(f(x))\frac{\partial f_m}{\partial x_j}(x)$
\begin{beispiel}
$g = g(x,y,z),\ h(x,y) = g(xy,x^2+y,x \sin y) = g(f(x,y)).$

$h_x(x,y) = g_x(f(x,y))y + g_y(f(x,y))2x + g_z(f(x,y))\sin y.$\\
$h_y(x,y) = g_x(f(x,y))x + g_y(f(x,y))1 + g_z(f(x,y))x \cos y.$
\end{beispiel}

\begin{hilfssatz}
Es sei $A$ eine $(m \times n)$-Matrix (reell), es sei $B$ eine $(n \times m)$-Matrix (reell) und es gelte
\begin{itemize}
\item[(i)] $BA = I $($= (n \times n)$-Einheitsmatrix) und
\item[(ii)] $AB = \tilde{I} $($= (m \times m)$-Einheitsmatrix)
\end{itemize}
Dann: $m = n$.
\end{hilfssatz}

\begin{beweis}
$\Phi(x):=Ax (x \in \MdR^n). \text{ Lin. Alg.} \folgt \Phi \text{ ist linear, }
\Phi:\MdR^n \to \MdR^m. \folgtnach{(i)} \Phi \text{ ist injektiv, also }
Kern \Phi = {0}. \text{ (ii) Sei }z \in \MdR^m, x:=Bz \folgtnach{(ii)} z = ABz = Ax = \Phi(x) \folgt \Phi \text{ ist surjektiv. Dann: } n = \dim \MdR^n \gleichnach{LA} \dim\kernn\Phi + \text{dim}\Phi(\MdR^n) = m.$
\end{beweis}

\begin{satz}[Injektivität und Dimensionsgleichheit]
$f:D\to \MdR^m$ sei db auf $D$, es sei $f(D)$ offen, $f$ injektiv auf $D$ und $f^{-1}:f(D)\to \MdR^n$ sei db auf $f(D)$. Dann:
\item[(1)] $m = n$
\item[(2)] $\forall x \in D:f'(x)$ ist eine invertierbare Matrix und $f'(x)^{-1} = (f^{-1})'(f(x))$
\end{satz}

\textbf{Beachte:}
\begin{itemize}
\item[(1)] Ist $D$ offen und $f:D\to \MdR^m$ db, so muss i. A. $f(D)$ nicht offen sein. Z.B.: $f(x) = \sin x, D = \MdR, f(D) = [-1,1]$
\item[(2)] Ist $D$ offen, $f:D\to \MdR^m$ db und injektiv, so muss i.A. $f^{-1}$ \underline{nicht} db sein. Z.B.: $f(x) = x^3, D = \MdR, f^{-1}$ ist in 0 \underline{nicht} db.
\end{itemize}

\begin{beweis} von 5.5: $g:=f^{-1}; x_0 \in D, z_0:=f(x_0) (\folgt x_0 = g(z_0))$
Es gilt: $g(f(x)) = x \forall x \in D, f(g(z)) = z \forall z \in f(D) \folgtnach{5.4} g'(f(x))\cdot f'(x) = I \forall x \in D; f'(g(z))\cdot g'(z) = \tilde{I}
\forall z \in f(D) \folgt \underbrace{g'(z_0)}_{=:B}\cdot \underbrace{f'(x_0)}_{=:A} = I, f'(x_0)\cdot g'(z_0) = \tilde{I} \folgtnach{5.5} m = n$ und $f'(x_0)^{-1} = g'(z_0) = (f^{-1})'(f(x_0))$.
\end{beweis}

\theoremstyle{numberbreak}
\newtheorem{spezialfall}[satz]{Spezialfall}
\chapter{Differenzierbarkeitseigenschaften reellwertiger Funktionen}
\def\grad{\mathop{\rm grad}\nolimits}

\begin{definition}
\begin{enumerate}
\index{Konvexität}
\item Seien $a,b \in \MdR^n; S[a,b]:=\{a+t(b-a): t\in [0,1]\}$ heißt
\begriff{Verbindungsstrecke} von $a$ und $b$
\item $M\subseteq \MdR^n$ heißt \textbf{konvex} $:\equizu$\ aus $a,b \in M$ folgt
stets: $S[a,b] \subseteq M$
\item Sei $k \in \MdN$ und $x^{(0)},\ldots,x^{(k)} \in \MdR^n.\ S[x^{(0)},\ldots,x^{(k)}]:=\bigcup_{j=1}^{k}S[x^{(j-1)}, x^{(j)}]$ heißt \begriff{Streckenzug} durch $x^{(0)},\ldots,x^{(k)}$ (in dieser Reihenfolge!)
\item Sei $G \subseteq \MdR^n$. $G$ heißt \begriff{Gebiet}$:\equizu\ G$ ist offen und aus $a,b \in G$ folgt: $\exists x^{(0)},\ldots,x^{(k)} \in G: x^{(0)}=a, x^{(k)}=b$ und $S[x^{(0)},\ldots,x^{(k)}] \subseteq G$.
\end{enumerate}
\end{definition}

\begin{vereinbarung}
Ab jetzt in diesem Paragraphen: $\emptyset \ne D \subseteq \MdR^n$, $D$ offen und
$f:D\to \MdR$ eine Funktion.
\end{vereinbarung}

\begin{satz}[Der Mittelwertsatz]
$f:D\to\MdR$ sei differenzierbar auf $D$, es seien $a,b \in D$ und $S[a,b]\subseteq D$. Dann: $$\exists\ \xi \in S[a,b]: f(b)-f(a)=f'(\xi)\cdot(b-a)$$
$ $%Bug
\end{satz}

\begin{beweis}
Sei $g(t):=a+t\cdot(b-a)$ für $t\in[0,1]$. $g([0,1])=S[a,b]\subseteq D$. $\Phi(t):=f(g(t)) (t \in [0,1])$ 5.4 $\folgt \Phi$ ist differenzierbar auf $[0,1]$ und $\Phi'(t) = f'(g(t))\cdot g'(t) = f'(a+t(b-a))\cdot(b-a)$. $f(b)-f(a)=\Phi(1)-\Phi(0) \folgtnach{MWS, AI} \Phi'(\eta) = f'(\underbrace{a+\eta(b-a)}_{=:\xi \in S})\cdot(b-a), \eta \in [0,1]$
\end{beweis}

\begin{folgerungen}
Sei $D$ ein \textbf{Gebiet} und $f,g:D\to\MdR$ seien differenzierbar auf $D$.
\begin{enumerate}
\item Ist $f'(x)=0\ \forall x \in D \folgt f$ ist auf $D$ konstant.
\item Ist $f'(x)=g'(x) \forall x \in D \folgt \exists c \in \MdR: f=g+c$ auf $D$.
\end{enumerate}
\end{folgerungen}

\begin{beweis}
(2) folgt aus (1). (1) Seien $a,b \in D$. Z.z.: $f(a)=f(b)$.
$\exists x^{(0)},\ldots,x^{(k)} \in D, x^{(0)}=a, x^{(k)}=b: S[x^{(0)},\ldots,x^{(k)}] \subseteq D$
$\forall j \in \{1,\ldots,k\}$ ex. nach 6.1 ein $\xi_j \in S[x^{(j-1)}, x^{(j)}]:
f(x^{(j)})-f(x^{(j-1)}) = \underbrace{f'(\xi_j)}_0\cdot(x^{(j)}-x^{(j-1)})=0
\folgt f(x^{(j)}) = f(x^{(j-1)}) \folgt f(a)=f(x^{(0)})=f(x^{(1)})=f(x^{(2)})=\ldots = f(x^{(k)}) = f(b)$.
\end{beweis}

\begin{satz}[Bedingung für Lipschitzstetigkeit]
$D$ sei konvex und $f:D\to\MdR$ sei differenzierbar auf $D$. Weiter sei $f'$ auf $D$ beschränkt. Dann ist $f$ auf $D$ Lipschitzstetig.
\end{satz}

\begin{beweis}
$\exists L \ge 0: \|f'(x)\| \le L \forall x \in D$. Seien $u,v \in D$. $D$ konvex $\folgt S[u,v]\subseteq D$. 6.1 $\folgt \exists\xi\in S[u,v]:f(u)-f(v)=f'(\xi)\cdot(u-v) \folgt |f(u)-f(v)|=|f'(\xi)\cdot(u-v)|\stackrel{CSU}{\le}\|f'(\xi)\| \|u-v\| \le L\|u-v\|$.
\end{beweis}

\begin{satz}[Linearität]
Sei $\Phi:\MdR^n\to\MdR^m$ eine Funktion.

$\Phi$ ist linear $\equizu \Phi \in C^1(\MdR^n, \MdR^m)$ und $\Phi(\alpha x)=\alpha\Phi(x)\ \forall x \in \MdR^n\ \forall \alpha \in \MdR.$
\end{satz}

\begin{beweis}
``$\folgt$'':
``$\impliedby$'': O.B.d.A.: $m=1$. Z.z.: $\exists a \in \MdR^n:\Phi(x)=a\cdot x \forall x \in \MdR^n$.
$a:=\Phi'(0) \Phi(0)=\Phi(2\cdot0)=2\cdot \Phi(0) \folgt \Phi(0)=0$.
$\forall x \in \MdR^n \forall \alpha \in \MdR: \Phi(\alpha x)=\alpha\Phi(x) \folgtnach{5.4} \alpha \Phi'(\alpha x)=\alpha\Phi'(x)\ \forall x \in \MdR^n\ \forall \alpha \in \MdR
\folgt \Phi'(x)=\Phi'(\alpha x)\ \forall x \in\MdR^n\ \forall\alpha\ne0$.$ \folgtnach{$\alpha\to0, f\in C^1$} \Phi'(x)=\Phi'(0)=a\ \forall x \in\MdR^n$.
$g(x):=(\Phi(x)-ax)^2\ (x \in \MdR^n)$, $ g(0)=(\Phi(0)-a\cdot0)^2=0$.
5.4 $\folgt g$ ist differenzierbar auf $\MdR^n$ und $g'(x)=2(\Phi(x)-ax)(\Phi'(x)-a)=0\ \forall x \in \MdR^n$.
6.2(1) $\folgt g(x)=g(0)=0\ \forall x\in\MdR^n \folgt \Phi(x)=a\cdot x\ \forall x \in \MdR^n.$
\end{beweis}

\paragraph{Die Richtungsableitung}
\indexlabel{Richtung}
\indexlabel{Richtungs-!Vektor}
\indexlabel{Richtungs-!Ableitung}
Sei $\emptyset \ne D \subseteq \MdR^n,\ D$ offen, $f:D \to \MdR$ und $x_0 \in D$. Ist $a \in \MdR^n$ und $\|a\|=1$, so heißt $a$ eine \textbf{Richtung} (oder ein \textbf{Richtungsvektor}).

Sei $a \in \MdR^n$ eine Richtung. $D$ offen $\folgt \exists \delta>0: U_\delta(x_0) \subseteq D$. Gerade durch $x_0$ mit Richtung $a:\{x_0+ta:t\in\MdR\}.\  \|x_0+ta-x_0\| = \|ta\| = |t|$. Also: $x_0+ta \in D$ für $t \in (-\delta,\delta),\ g(t) := f(x_0+ta)\ (t \in (-\delta,\delta))$.

$f$ heißt \textbf{in $x_0$ in Richtung $a$ db}, gdw. der Grenzwert $$\lim_{t\to 0} \frac{f(x_0+ta)-f(x_0)}{t}$$ existiert und $\in \MdR$ ist. In diesem Fall heißt $$\frac{\partial f}{\partial a}(x_0) := \lim_{t\to 0} \frac{f(x_0+ta)-f(x_0)}{t}$$ die \textbf{Richtungsableitung von $f$ in $x_0$ in Richtung $a$}.

\begin{beispiele}
\item $f$ ist in $x_0$ partiell db nach $x_j \equizu f$ ist in $x_0$ db in Richtung $e_j$. In diesem Fall gilt: $\frac{\partial f}{\partial x_j}(x_0) = \frac{\partial f}{\partial e_j}(x_0)$.

\item $$f(x,y) := \begin{cases}
\frac{xy}{x^2+y^2} & \text{, falls } (x,y) \ne (0,0)\\
0 & \text{, falls } (x,y) = (0,0)\end{cases}$$

$x_0 = (0,0).$ Sei $a=(a_1,a_2) \in \MdR^2$ eine Richtung, also $a_1^2+a_2^2=1;\ \frac{f(ta)-f(0,0)}{t} = \frac{1}{t} \frac{t^2a_1a_2}{t^2a_1^2+t^2a_2^2} = \frac{a_1a_2}{t}$. D.h.: $\frac{\partial f}{\partial a}(0,0)$ ex. $\equizu a_1a_2 = 0 \equizu a \in \{(1,0),(-1,0),(0,1),(0,-1)\}$. In diesem Fall: $\frac{\partial f}{\partial a}(0,0) = 0.$

\item $$f(x,y) := \begin{cases}
\frac{xy^2}{x^2+y^4} & \text{, falls } (x,y) \ne (0,0)\\
0 & \text{, falls } (x,y) = (0,0)\end{cases}$$

$x_0 = (0,0)$. Sei $a = (a_1,a_2) \in \MdR$ eine Richtung. $\frac{f(ta)-f(0,0)}{t} = \frac{1}{t} \frac{t^3a_1a_2^2}{t^2a_1^2+t^4a_2^4} = \frac{a_1a_2^2}{a_1^2+t^2a_2^4} \overset{t \to 0}{\to} \begin{cases}
0 & \text{, falls } a_1=0\\
\frac{a_2^2}{a_1} & \text{, falls } a_1 \ne 0 \end{cases}$

D.h. $\frac{\partial f}{\partial a}(0,0)$ existiert für \emph{jede} Richtung $a \in \MdR^2$. Z.B.: $a = \frac{1}{\sqrt{2}}(1,1): \frac{\partial f}{\partial a}(0,0) = \frac{1}{\sqrt{2}}.$

$f(x,\sqrt{x}) = \frac{x^2}{2x^2} = \frac{1}{2}\ \forall x>0 \folgt f$ ist in $(0,0)$ \emph{nicht} stetig.
\end{beispiele}

\begin{satz}[Richtungsableitungen]
Sei $x_0 \in D,\ a \in \MdR^n$ eine Richtung, $f:D \to \MdR$.
\begin{enumerate}
\item $\frac{\partial f}{\partial a}(x_0)$ existiert $\equizu \frac{\partial f}{\partial (-a)}(x_0)$ existiert. In diesem Fall ist: $$\frac{\partial f}{\partial (-a)}(x_0) = -\frac{\partial f}{\partial a}(x_0)$$
\item $f$ sei in $x_0$ db. Dann:
\begin{enumerate}
\item[(i)] $\frac{\partial f}{\partial a}(x_0)$ existiert und $$\frac{\partial f}{\partial a}(x_0) = a\cdot \grad f(x_0).$$
\item[(ii)] Sei $\grad f(x_0) \ne 0$ und $a_0 := \|\grad f(x_0)\|^{-1}\cdot \grad f(x_0)$. Dann: $$\frac{\partial f}{\partial (-a_0)}(x_0) \le \frac{\partial f}{\partial a}(x_0) \le \frac{\partial f}{\partial a_0}(x_0) = \|\grad f(x_0)\|.$$ Weiter gilt: $\frac{\partial f}{\partial a}(x_0) < \frac{\partial f}{\partial a_0}(x_0)$, falls $a \ne a_0$; $\frac{\partial f}{\partial (-a_0)}(x_0) < \frac{\partial f}{\partial a}(x_0)$, falls $a \ne -a_0$.
\end{enumerate}
\end{enumerate}
\end{satz}

\begin{beweis}
\begin{enumerate}
\item $\frac{(f(x_0+t(-a))-f(x_0))}{t} = -\frac{(f(x_0+(-t)a)-f(x_0))}{-t} \folgt$ Beh.
\item \begin{enumerate}
\item[(i)] $g(t) := f(x_0+ta)$ ($|t|$ hinreichend klein). Aus Satz 5.4 folgt: $g$ ist db in $t=0$ und $g'(0) = f'(x_0) \cdot a \folgt \frac{\partial f}{\partial a}(x_0)$ existiert und ist $= g'(0) = \grad f(x_0)\cdot a$
\item[(ii)] $\uwave{\left| \frac{\partial f}{\partial a}(x_0) \right|} \gleichnach{(i)} |a\cdot \grad f(x_0)| \overset{\text{CSU}}{\le} \|a\|\cdot \|\grad f(x_0)\| = \|\grad f(x_0)\| = \frac{1}{\|\grad f(x_0)\|} \grad f(x_0) \cdot \grad f(x_0) = a_0\cdot \grad f(x_0) \gleichnach{(i)} \uwave{\frac{\partial f}{\partial a_0}(x_0)}$

$\folgt \frac{\partial f}{\partial (-a_0)}(x_0) \gleichnach{(1)} -\frac{\partial f}{\partial a_0}(x_0) \le \frac{\partial f}{\partial a}(x_0) \le \frac{\partial f}{\partial a_0}(x_0) = \|\grad f(x_0)\|$

Sei $\frac{\partial f}{\partial a}(x_0) = \frac{\partial f}{\partial a_0}(x_0) \folgtnach{(i),(ii)} a\cdot\grad f(x_0) = \|\grad f(x_0)\| \folgt a\cdot a_0 = 1 \folgt \|a-a_0\|^2 = (a-a_0)(a-a_0) = a\cdot a - 2a\cdot a_0 + a_0\cdot a_0 = 1-2+1 = 0 \folgt a=a_0.$
\end{enumerate}
\end{enumerate}
\end{beweis}

\paragraph{Der Satz von Taylor}
Im Folgenden sei $f:D \to \MdR$ zunächst "`genügend oft partiell db"', $x_0 \in D$ und $h=(h_1,\ldots,h_n) \in \MdR^n$. Wir führen folgenden Formalismus ein.

$$\nabla := \left( \frac{\partial}{\partial x_1},\ldots,\frac{\partial}{\partial x_n}\right)\ \text{("`Nabla"')};\ \nabla f:= \left( \frac{\partial f}{\partial x_1},\ldots,\frac{\partial f}{\partial x_n}\right) = \grad f;\ \nabla f(x_0) := \grad f(x_0)$$

$$(h\cdot\nabla) := h_1 \frac{\partial}{\partial x_1} + \ldots + h_n \frac{\partial}{\partial x_n};\ (h\cdot\nabla) f:= h_1 \frac{\partial f}{\partial x_1} + \ldots + h_n \frac{\partial f}{\partial x_n} = h \grad f;\ (h\cdot\nabla) f(x_0) := h\cdot\grad f(x_0)$$

$(h\cdot\nabla)^{(0)} f(x_0) := f(x_0)$. Für $k\in\MdN: (h\cdot\nabla)^{(k)} := \left( h_1 \frac{\partial}{\partial x_1} + \ldots + h_n \frac{\partial}{\partial x_n} \right)^k$

$(h\cdot\nabla)^{(2)} f(x_0) = \sum_{j=1}^n \sum_{k=1}^n h_jh_k\frac{\partial^2 f}{\partial x_j \partial x_k} (x_0)$

$(h\cdot\nabla)^{(3)} f(x_0) = \sum_{j=1}^n \sum_{k=1}^n \sum_{l=1}^n h_jh_kh_l\frac{\partial^3 f}{\partial x_j \partial x_k \partial x_l} (x_0)$

\begin{beispiel}
$(n=2): h = (h_1,h_2).$

$(h\cdot\nabla)^{(0)} f(x_0) = f(x_0),\ (h\cdot\nabla)^{(1)} f(x_0) = h\cdot \grad f(x_0) = h_1 f_x(x_0) + h_2 f_y(x_0)$.

$(h\cdot\nabla)^{(2)} f(x_0) = \left( h_1 \frac{\partial f}{\partial x} + h_2 \frac{\partial f}{\partial y}\right)^2 (x_0) = h_1^2 \frac{\partial^2 f}{\partial^2 x} (x_0) + h_1h_2 \frac{\partial^2 f}{\partial x \partial y} (x_0) + h_2h_1 \frac{\partial^2 f}{\partial y \partial x} (x_0) + h_2^2 \frac{\partial^2 f}{\partial^2 y} (x_0).$
\end{beispiel}

\begin{satz}[Der Satz von Taylor]
Sei $k\in\MdN, f\in C^{k+1}(D,\MdR),x_0 \in D, h\in\MdR^n$ und $S[x_0,x_0+h]\subseteq D$. Dann:
$$f(x_0+h)=\sum_{j=0}^k\frac{(h\cdot\nabla)^{(j)}f(x_0)}{j!}+\frac{(h\cdot\nabla)^{(k+1)}f(\xi)}{(k+1)!}$$
wobei $\xi \in S[x_0, x_0+h]$
\end{satz}

\begin{beweis}
$\Phi(t):=f(x_0+th)$ für $t\in[0,1]$. 5.4$\folgt \Phi \in C^{k+1}[0,1],\ \Phi'(t)=f'(x_0+th) \cdot h=(h \cdot \nabla)f(x_0+th)$\\
Induktiv: $\Phi^{(j)}(t)=(h\cdot\nabla)^{(j)}f(x_0+th)\ (j=0,\ldots,k+1, t\in[0,1]).\ \Phi(0)=f(x_0), \Phi(1)=f(x_0+h);\ \Phi^{(j)}(0)=(h\cdot\nabla)^{(j)}f(x_0)$. Analysis 1 (22.2) $\folgt \Phi(1)=\ds\sum_{j=0}^k\frac{\Phi^{(j)}(0)f(x_0)}{j!}+\frac{\Phi^{(k+1)}f(\eta)}{(k+1)!}$, wobei $\eta\in[0,1]\folgt f(x_0+h)=\ds\sum_{j=1}^k\frac{(h\cdot\nabla)^{(j)}f(x_0)}{j!}+\frac{(h\cdot\nabla)^{(k+1)}f(x_0+\eta h)}{(k+1)!},\ \xi:=x_0+\eta h$
\end{beweis}

\begin{spezialfall}
Sei $f\in C^2(D,\MdR),x_0\in D, h\in \MdR^n, S[x_0,x_0+h]\subseteq D$. Dann:
$$f(x_0+h)=f(x_0)+\grad f(x_0)\cdot h+\frac{1}{2}\sum_{j,k=1}^nh_jh_k\frac{\partial^2 f}{\partial x_j\partial x_k}(x_0+\eta h)$$
\end{spezialfall}

\chapter{Quadratische Formen}
\def\grad{\mathop{\rm grad}\nolimits}

\begin{vereinbarung}
In diesem Paragraphen sei $A$ stets eine reelle und symmetrische $(n\times n)$-Matrix, $(A=A^\top)$. Also: $A=(a_{jk})$, dann $a_{jk}=a_{kj}\ (k,j=1,\ldots,n)$
\end{vereinbarung}

\begin{definition*}
$Q_A:\MdR^n\to\MdR$ durch $Q_A(x):=x(Ax)$. $Q_A$ heißt die zu $A$ gehörende \begriff{quadratische Form}. Für $x=(x_1,\ldots,x_n):$
$$Q_A(x)=\ds\sum_{j,k=1}^na_{jk}x_jx_k$$
\end{definition*}

\begin{beispiel}
Sei $f\in C^2(D,\MdR),x_0\in D, h\in \MdR^n, S[x_0,x_0+h]\subseteq D$.
$$H_f(x_0):=\begin{pmatrix}
f_{x_1x_1}(x_0)&\cdots&f_{x_1x_n}(x_0)\\
f_{x_2x_1}(x_0)&\cdots&f_{x_2x_n}(x_0)\\
\vdots& &\vdots\\
f_{x_nx_1}(x_0)&\cdots&f_{x_nx_n}(x_0)\\
\end{pmatrix}$$
heißt die \begriff{Hesse-Matrix} von $f$ in $x_0$. 4.1$\folgt H_f(x_0)$ ist symmetrisch. Aus 6.7 folgt:
$$f(x_0+h)=f(x_0)+\grad f(x_0)\cdot h + \frac{1}{2}Q_B(h)\text{ mit }B=H_f(x_0+\eta h)$$
\end{beispiel}
\begin{definition*}
\begin{tabular}{ll}
\\ % Bug!
\\
\index{Positivdefinitheit}
\index{Indefinitheit}
\index{Negativdefinitheit}
$A$ heißt $\textbf{positiv definit}$ (pd) & $:\equizu$ $Q_A(x)>0\ \forall x\in\MdR^n\ \backslash\ \{0\}$\\
$A$ heißt $\textbf{negativ definit}$ (nd) & $:\equizu$ $Q_A(x)<0\ \forall x\in\MdR^n\ \backslash\ \{0\}$\\
$A$ heißt $\textbf{indefinit}$ (id) & $:\equizu \exists u,v\in\MdR^n: Q_A(u)>0, Q_A(v)<0$
\end{tabular}
\end{definition*}

\begin{beispiele}
\item $(n=2),\ A=\left(\begin{smallmatrix}a&b\\b&c\end{smallmatrix}\right)$\\
$Q_A(x,y):=ax^2+2bxy+cy^2\ \left((x,y)\in\MdR^2\right)$. Nachrechnen:\\
$$aQ_A(x,y)=(ax+by)^2+(\det A)y^2\ \forall (x,y)\in\MdR^2$$ Übung:\\
\begin{tabular}{ll}
A ist positiv definit & $\equizu a>0, \det A>0$\\
A ist negativ definit & $\equizu a<0, \det A>0$\\
A ist indefinit& $\equizu \det A<0$
\end{tabular}
\item $(n=3),\ A=\left(\begin{smallmatrix}1&0&1\\0&0&0\\1&0&1\end{smallmatrix}\right)$\\
$Q_A(x,y,z)=(x+z)^2\ \forall\ (x,y,z)\in\MdR^3.\ Q_A(0,1,0)=0.\ A$ ist weder pd, noch id, noch nd.
\item ohne Beweis ($\to$ Lineare Algebra). $A$ symmetrisch $\folgt$ alle \begriff{Eigenwerte} (EW) von $A$ sind $\in\MdR$.\\
\begin{tabular}{ll}
A ist positiv definit & $\equizu$ Alle Eigenwerte von $A$ sind $>0$\\
A ist negativ definit & $\equizu$ Alle Eigenwerte von $A$ sind $<0$\\
A ist indefinit& $\equizu \exists$ Eigenwerte $\lambda, \mu$ von $A$ mit $\lambda>0,\ \mu<0$
\end{tabular}
\end{beispiele}

\begin{satz}[Regeln zu definiten Matrizen und quadratischen Formen]
\begin{enumerate}
\item $A$ ist positiv definit $\equizu$ $-A$ ist negativ definit
\item $Q_A(\alpha x)=\alpha^2Q_A(x)\ \forall x\in\MdR^n\ \forall \alpha\in\MdR$
\item \begin{tabular}{ll}
A ist positiv definit & $\equizu \exists c>0: Q_A(x)\ge c\|x\|^2\ \forall x\in\MdR^n$\\
A ist negativ definit & $\equizu \exists c>0: Q_A(x)\le -c\|x\|^2\ \forall x\in\MdR^n$
\end{tabular}
\end{enumerate}
\end{satz}

\begin{beweise}
\item Klar
\item $Q_A(\alpha x)=(\alpha x)(A(\alpha x))=\alpha^2x(Ax)=\alpha^2Q_A(x)$
\item "`$\impliedby$"': Klar. "`$\folgt$"': $K:=\{x\in\MdR^n: \|x\|=1\}=\partial U_1(0)$ ist beschränkt und abgeschlossen. $Q_A$ ist stetig auf $K$. 3.3 $\folgt\exists x_0\in K: Q_A(x_0)\le Q_A(x)\ \forall x\in K$. $c:=Q_A(x_0).\ A$ positiv definit, $x_0\ne 0\folgt Q_A(x_0)=c>0$. Sei $x\in\MdR^n\ \backslash\ \{0\};\ z:=\frac{1}{\|x\|}x\folgt z\in K\folgt Q_A(z)\ge c\folgt c \le Q_A\left(\frac{1}{\|x\|}x\right)\gleichnach{(2)}\frac{1}{\|x\|}^2Q_A(x)\folgt Q_A(x)\ge c\|x\|^2$
\end{beweise}

\begin{satz}[Störung von definiten Matrizen]
\begin{enumerate}
\item $A$ sei positiv definit \alt{negativ definit}. Dann existiert ein $\ep>0$ mit: Ist $B=(b_{jk})$ eine weitere symmetrische $(n\times n)$-Matrix und gilt: $(*)\ |a_{jk}-b_{jk}|\le\ep\ (j,k=1,\ldots, n)$, so ist B positiv definit \alt{negativ definit}.
\item $A$ sei indefinit. Dann existieren $u,v\in\MdR^n$ und $\ep>0$ mit: ist $B=(b_{jk})$ eine weitere symmetrische $(n\times n)$-Matrix und gilt: $(*)\ |a_{jk}-b_{jk}|\le\ep\ (j,k=1,\ldots,n)$, so ist $Q_B(u)>0, Q_B(v)<0$. Insbesondere: $B$ ist indefinit.
\end{enumerate}
\end{satz}

\begin{beweise}
\item $A$ sei positiv definit $\folgtnach{7.1}\exists c>0: Q_A(x)\ge c\|x\|^2\ \forall x\in\MdR^n$. $\ep:=\frac{c}{2n^2}$. Sei $B=(b_{jk})$ eine symmetrische Matrix mit $(*)$. Für $x=(x_1,\ldots,x_n)\in\MdR^n:\ Q_A(x)-Q_B(x)\le|Q_A(x)-Q_B(x)|=\left|\ds\sum_{j,k=1}^m(a_{jk}-b_{jk})x_jx_k\right|\le\ds\sum_{j,k=1}^n\underbrace{|a_{jk}-b_{jk}|}_{\le\ep}\underbrace{|x_j|}_{\le\|x\|}\underbrace{|x_k|}_{\le\|x\|}\le\ep\|x\|^2n^2=\frac{c}{2n^2}\|x\|^2n^2=\frac{c}{2}\|x\|^2$
\item $A$ sei indefinit. $\exists u,v\in\MdR^n:\ Q_A(u)>0, Q_A(v)<0$. $\alpha:=\min\left\{\frac{Q_A(u)}{\|u\|^2},\ -\frac{Q_A(v)}{\|v\|^2}\right\}\folgt\alpha>0$. $\ep:=\frac{\alpha}{2n^2}$. Sei $B=(b_{jk})$ eine symmetrische Matrix mit $(*)$.\\
$Q_A(u)-Q_B(u)\overset{\text{Wie bei (1)}}{\le}\ep u^2\|u\|^2=\frac{\alpha}{2n^2}n^2\|u\|^2=\frac{\alpha}{2}\|u\|^2\le\frac{1}{2}\frac{Q_A(u)}{\|u\|^2}\|u\|^2=\frac{1}{2}Q_A(u) \folgt Q_B(u)\ge\frac{1}{2}Q_A(u)>0$. Analog: $Q_B(v)<0$.
\end{beweise}

\chapter{Extremwerte}
\def\grad{\mathop{\rm grad}\nolimits}

\begin{vereinbarung}
In diesem Paragraphen sei $\emptyset\ne D \subseteq\MdR^n, f:D\to\MdR$ und $x_0\in D$
\end{vereinbarung}

\begin{definition*}
\indexlabel{lokal!Maximum}
\indexlabel{lokal!Minimum}
\indexlabel{lokal!Extremum}
\indexlabel{stationärer Punkt}
\begin{enumerate}
\item
$f$ hat in $x_0$ ein \textbf{lokales Maximum} $:\equizu \exists \delta>0:\ f(x)\le f(x_0)\ \forall x\in D \cap U_\delta(x_0)$.\\
$f$ hat in $x_0$ ein \textbf{lokales Minimum} $:\equizu \exists \delta>0:\ f(x)\ge f(x_0)\ \forall x\in D \cap U_\delta(x_0)$.\\
\textbf{lokales Extremum} = lokales Maximum oder lokales Minimum
\item Ist $D$ offen, $f$ in $x_0$ partiell differenzierbar und $\grad f(x_0)=0$, so heißt $x_0$ ein stationärer Punkt.
\end{enumerate}
\end{definition*}

\begin{satz}[Nullstelle des Gradienten]
Ist $D$ offen und hat $f$ in $x_0$ ein lokales Extremum und ist $f$ in $x_0$ partiell differenzierbar, dann ist $\grad f(x_0)=0$.
\end{satz}

\begin{beweis}
$f$ habe in $x_0$ ein lokales Maximum. Also $\exists \delta>0: U_\delta(x_0)\subseteq D$ und $f(x)\le f(x_0)\ \forall x\in U_\delta(x_0)$. Sei $j \in \{1,\ldots,n\}$. Dann: $x_0 + te_j \in U_\delta(x_0)$ für $t\in (-\delta, \delta)$. $g(t):=f(x_0 + te_j)\ (t\in (-\delta, \delta))$. $g$ ist differenzierbar in $t=0$ und $g'(0)=f_{x_j}(x_0)$. $g(t)=f(x_0+te_j)\le f(x_0)=g(0)\ \forall t\in(-\delta,\delta)$. Analysis 1, 21.5 $\folgt g'(0)=0\folgt f_{x_j}(x_0)=0$
\end{beweis}

\begin{satz}[Definitheit und Extremwerte]
Sei $D$ offen, $f\in C^2(D,\MdR)$ und $\grad f(x_0)=0$.
\begin{enumerate}
\item[(i)]
Ist $H_f(x_0)$ positiv definit $\folgt f$ hat in $x_0$ ein lokales Minimum.
\item[(ii)]
Ist $H_f(x_0)$ negativ definit $\folgt f$ hat in $x_0$ ein lokales Maximum.
\item[(iii)]
Ist $H_f(x_0)$ indefinit $\folgt f$ hat in $x_0$ \underline{kein} lokales Extremum.
\end{enumerate}
\end{satz}

\begin{beweis}
\begin{enumerate}
\item[(i),]
(ii) $A:=H_f(x_0)$ sei positiv definit oder negativ definit oder indefinit. Sei $\ep>0$ wie in 7.2. $f\in C^2(D,\MdR)\folgt \exists \delta>0: U_\delta(x_0)\subseteq D$ und $(*)\ |f_{x_jx_k}(x)-f_{x_jx_k}(x_0)|\le\ep\ \forall x\in U_\delta(x_0)\ (j,k=1,\ldots,n)$. Sei $x\in U_\delta(x_0) \ \backslash\ \{x_0\}, h:=x-x_0\folgt x=x_0+h, h\ne 0$ und $S[x_0,x_0+h] \subseteq U_\delta(x_0)$ 6.7$\folgt\exists \eta\in [0,1]:\ f(x)=f(x_0+h)=f(x_0) + \underbrace{h\cdot \grad f(x_0)}_{=0}+\frac{1}{2}Q_B(h)$, wobei $B=H_f(x_0 + \eta h)$. Also: $(**)\ f(x)=f(x_0)+\frac{1}{2}Q_B(h)$. $A$ sei positiv definit \alt{negativ definit} $\folgtnach{7.2} B$ ist positiv definit \alt{negativ definit}. $\folgtwegen{h\ne 0}Q_B(h)\stackrel{(<)}{>}0 \folgtwegen{(**)}f(x)\stackrel{(<)}{>}f(x_0)\folgt f$ hat in $x_0$ ein lokales Minimum \alt{Maximum}.
\item[(iii)]$A$ sei indefinit und es seien $u, v\in\MdR^n$ wie in 7.2. Wegen 7.1 OBdA: $\|u\|=\|v\|=1$. Dann: $x_0+tu, x_0+tv \in U_\delta(x_0)$ für $t\in(-\delta, \delta)$. Sei $t\in(-\delta, \delta), t\ne 0$. Mit $h:=t\stackrel{(v)}{u}$ folgt aus 7.2 und $(**):\ f(x_0+t\stackrel{(v)}{u})=f(x_0)+\frac{1}{2}Q_B(t\stackrel{(v)}{u})=f(x_0)+\frac{t^2}{2}\underbrace{Q_B(\stackrel{(v)}{u})}_{>0\text{/}<0\text{ (7.2)}}\stackrel{(>)}{<}f(x_0)\folgt f$ hat in $x_0$ kein lokales Extremum.
\end{enumerate}
\end{beweis}

\begin{beispiele}
\item $D=\MdR^2, f(x,y)=x^2+y^2-2xy-5$. $f_x=2x-2y, f_y=2y-2x;\ \grad f(x,y)=(0,0)\equizu x=y$. Stationäre Punkte: $(x,x)\ (x\in\MdR)$.\\
$$f_{xx}=2,\ f_{xy}=-2=f_{yx},\ f_{yy}=2\folgt H_f(x,x)=\begin{pmatrix}2&-2\\-2&2\end{pmatrix}$$
$\det H_f(x,x)=0\folgt H_f(x,x)$ ist weder pd, noch nd, noch id.\\
Es ist $f(x,y)=(x-y)^2-5\ge -5\ \forall\ (x,y)\in\MdR^2$ und $f(x,x)=-5\ \forall x\in\MdR$.
\item $D=\MdR^2, f(x,y)=x^3-12xy+8y^3$.\\
$f_x=3x^2-12y=3(x^2-4y),\ f_y=-12x+24y^2=12(-x+2y^2)$. $\grad f(x,y)=(0,0)\equizu x^2=4y, x=2y^2\folgt 4y^4=4y\folgt y=0$ oder $y=1\folgt (x,y)=(0,0)$ oder $(x,y)=(2,1)$\\
$$f_{xx}=6x,\ f_{xy}=-12=f_{yx},\ f_{yy}=48y.\ H_f(0,0)=\begin{pmatrix}0&-12&\\-12&0\end{pmatrix}$$
$\det H_f(0,0)=-144<0\folgt H_f(0,0)$ ist indefinit $\folgt f$ hat in $(0,0)$ kein lokales Extremum.
$$H_f(2,1)=\begin{pmatrix}12&-12\\-12&48\end{pmatrix}$$
$12>0, \det H_f(2,1)>0\folgt H_f(2,1)$ ist positiv definit $\folgt f$ hat in $(2,1)$ ein lokales Minimum.
\item $K:=\{(x,y)\in\MdR^2: x,y\ge 0, y\le -x+3\}, f(x,y)=3xy-x^2y-xy^2$. Bestimme $\max f(K), \min f(K)$. $f(x,y)=xy(3-x-y).\ K=\partial K \cup K^\circ$. $K$ ist beschränkt und abgeschlossen $\folgtnach{3.3}\exists\ (x_1,y_1), (x_2,y_2)\in K: \max f(K)=f(x_1, y_1), \min f(K)=f(x_2,y_2)$. $f\ge 0$ auf $K$, $f=0$ auf $\partial K$, also $\min f(K)=0$. $f$ ist nicht konstant $\folgt f(x_2,y_2)>0\folgt (x_2,y_2)\in K^\circ\folgtnach{8.1}\grad f(x_1,x_2)=0$. Nachrechnen: $(x_2,y_2)=(1,1); f(1,1)=1=\max f(K)$.
\end{beispiele}

\chapter{Der Umkehrsatz}
\def\grad{\mathop{\rm grad}\nolimits}

\begin{erinnerung}
Sei $x_0\in\MdR^n$ und $U\subseteq\MdR^n$. $U$ ist eine Umgebung von $x_0\equizu\exists\delta>0:U_\delta(x_0)\subseteq U$
\end{erinnerung}

\begin{wichtigerhilfssatz}[Offenheit des Bildes]
Sei $\delta>0, f:U_\delta(0)\subseteq\MdR^n\to\MdR^n$ stetig, $f(0)=0$ und $V$ sei eine offene Umgebung von $f(0)\ (=0)$. $U:=\{x\in U_\delta(0):f(x)\in V\}$. Dann ist $U$ eine offene Umgebung von $0$.
\end{wichtigerhilfssatz}
\begin{beweis}
Übung
\end{beweis}

\begin{erinnerung}
\begriff{Cramersche Regel}: Sei $A$ eine reelle $(n\times n)$-Matrix, $\det A\ne 0$, und $b\in\MdR^n$. Das lineare Gleichungssystem $Ax=b$ hat genau eine Lösung: $x=(x_1,\ldots,x_n)=A^{-1}b$. Ersetze in $A$ die $j$-te Spalte durch $b^\top$. Es entsteht eine Matrix $A_j$. Dann: $x_j=\frac{\det A_j}{\det A}$.
\end{erinnerung}

\begin{satz}[Stetigkeit der Umkehrfunktion]
Sei $\emptyset\ne D\subseteq \MdR^n, D$ offen, $f\in C^1(D,\MdR^n)$. $f$ sei auf $D$ injektiv und es sei $f(D)$ offen. Weiter sei $\det f'(x)\ne 0\ \forall x\in D$ und $f^{-1}$ sei auf $f(D)$ differenzierbar. Dann: $f^{-1}\in C^1(f(D),\MdR^n)$.
\end{satz}

\begin{beweis}
Sei $f^{-1}=g=(g_1,\ldots,g_n), g=g(y)$. Zu zeigen: $\frac{\partial g_j}{\partial y_k}$ sind stetig auf $f(D)$. 5.6\folgt $g'(y)\cdot f'(x)=I$ $(n\times n\text{-Einheitsmatrix})$, wobei $y=f(x)\in f(D)\folgt$
$$
\begin{pmatrix}
g_1'(y)\\
\vdots\\
g_n'(y)
\end{pmatrix}\cdot f'(x)=
\begin{pmatrix}
1 & & 0 \\
& \ddots &\\
0 & & 1
\end{pmatrix}$$
$\folgt \grad g_j(y)\cdot f'(x)=e_j\folgt f'(x)^\top\cdot \grad g_j(y)^\top=e_j^\top$. Ersetze in $f'(x)^\top$ die $k$-te Spalte durch $e_j^\top$. Es entsteht die Matrix $A_k(x)=A_k(f^{-1}(y))$. Cramersche Regel $\folgt \frac{\partial g_j}{\partial y_k}(y)=\frac{\det A_k(f^{-1}(y))}{\det f'(x)}=\frac{\det A_k(f^{-1}(y))}{\det f'(f^{-1}(y))}$. $f\in C^1(D,\MdR), f^{-1}$ stetig $\folgt$ obige Definitionen hängen stetig von y ab $\folgt \frac{\partial g_j}{\partial y_k}\in C(f(D),\MdR)$.
\end{beweis}

\begin{satz}[Der Umkehrsatz]
    Sei $\emptyset \ne D \subseteq \MdR ^n$, $D$ sei offen,
    $f\in C^1(D, \MdR^n)$, $x_0\in D$ und $\det f'(x_0) \ne 0$.\\
    Dann existiert eine offene Umgebung $U$ von $x_0$ und eine offene
    Umgebung $V$ von $f(x_0)$ mit:
    \begin{enumerate}[(a)]
        \item $f$ ist auf $U$ injektiv, $f(U)=V$ und
              $\det f'(x) \ne 0 \ \forall x\in U$
        \item Für $f^{-1}: V\to U$ gilt: $f^{-1}$ ist stetig
              differenzierbar auf $V$ und
              \[ (f^{-1})'(f(x)) = (f'(x))^{-1}\ \forall x\in U \]
    \end{enumerate}
\end{satz}

\begin{folgerung}[Satz von der offenen Abbildung]
$D$ und $f$ seien wie in 9.3 und es gelte: $\det f'(x) \ne 0 \ \forall x\in D$. Dann ist $f(D)$ offen.
\end{folgerung}

\begin{beweis}
O.B.d.A: $x_0 = 0$, $f(x_0) = f(0) = 0$ und $f'(0) = I$ (=$(n\times n)$-Einheitsmatrix)

Die Abbildungen $x \mapsto \det f'(x)$ und $x\mapsto \|f'(x) - I\|$ sind auf D stetig, $\det f'(0) \ne 0$, $\| f'(0)- I \| = 0$. Dann existiert ein $\delta > 0$: $K := U_\delta(0) \subseteq D$, $\overline{K} = \overline{U_\delta(0)} \subseteq D$ und
\begin{enumerate}
\item $\det f'(x) \ne 0 \ \forall x\in\overline{K}$ und
\item $\|f'(x) - I \| \le \frac{1}{2n} \ \forall x\in\overline{K}$

\item \textbf{Behauptung:} $\frac{1}{2} \|u-v\| \le \|f(u) - f(v)\| \ \forall u,v\in\overline{K}$, insbesondere ist $f$ injektiv auf $\overline{K}$

\item $f^{-1}$ ist stetig auf $f(\overline{K})$: Seien $\xi, \eta \in f(\overline{K})$, $u:=f^{-1}(\xi)$, $v:= f^{-1}(\eta) \folgt u,v \in \overline{K}$ und $\|f^{-1}(\xi) - f^{-1}(\eta)\| = \|u-v\| \stackrel{\text{(3)}}{\le} 2\|f(u) - f(v)\| = 2\|\xi - \eta\|$
\end{enumerate}

Beweis zu (3): $h(x) := f(x) - x \ (x\in D) \folgt h\in C^1(D,\MdR^n)$ und $h'(x) = f'(x) - I $. Sei $h=(h1,\ldots,h_n)$. Also: $h' = \begin{pmatrix} h_1' \\ \vdots \\ h_n' \end{pmatrix}$. Seien $u,v\in \overline{K}$ und $j\in \{1,\ldots,n\}$.

$|h_j(u) - h_j(v)| \gleichnach{6.1} |h_j'(\xi) \cdot (u-v)| \stackrel{\text{CSU}}{\le} \|h_j'(\xi)\| \|u-v\| \le \|h'(\xi)\| \|u-v\|$, $\xi \in S[u,v] \in \overline{K}$. (2) $\folgt \le \frac{1}{2n}\|u-v\|$ \\
$\folgt \|h(u) - h(v)\| = \left(\sum_{j=1}^{n}(h_j(n) - h_j(v))^2\right)^{\frac{1}{2}} \le \left( \sum_{j=1}^n \frac{1}{4n^2}\|u-v\|^2\right)^{\frac{1}{2}} = \frac{1}{2n}\|u-v\|\sqrt{n} \le \frac{1}{2}\|u-v\| \folgt \|u-v\| - \|f(u)-f(v)\| \le \|f(u) - f(v) - (u-v)\| = \|h(u) - h(v)\| \le \frac{1}{2}\|u-v\| \folgt$ (3)

$V:=U_{\frac{\delta}{4}}(0)$ ist eine offene Umgebung von $f(0) \ (=0)$. $U:=\{x\in K: f(x) \in V\}$ Klar: $U\subseteq K \subseteq \overline{K}$, $0\in U$, 9.1 $\folgt$ $U$ ist eine offene Umgebung von 0. (3) $\folgt$ $f$ ist auf $U$ injektiv. (1) $\folgt \det f'(x) \ne 0 \ \forall x\in U$. (4) $\folgt$ $f^{-1}$ ist stetig auf $f(U)$. Klar: $f(U) \subseteq V$. Für (a) ist noch zu zeigen: $V\subseteq f(U)$.

Sei $y\in V$. $w(x) := \| f(x) - y\|^2 = (f(x) - y)\cdot(f(x)-y) \folgt w\in C^1(D,\MdR)$ und (nachzurechnen) $w'(x) = 2(f(x)-y)\cdot f'(x)$. $\overline K$ ist beschränkt und abgeschlossen $\folgtnach{3.3} \exists x_1 \in \overline K: \text{ (5) } w(x_1) \le w(x) \ \forall x\in\overline K$.

\textbf{Behauptung:} $x_1 \in K$. \\
Annahme: $x_1\ne K \folgt x_1 \in \partial K \folgt \| x_1 \| = \delta$. $2\sqrt {w(0)} =  2\|f(0) - y\| = 2\|y\|\le 2 \frac{\delta} 4 = \frac \delta 2 = \frac{\|x_1\|} 2 = \frac 1 2 \|x_1 - 0 \| \stackrel{\text{(3)}}{\le} \|f(x_1) - f(0)\| = \|f(x_1) - y + y - f(0)\| \le \|f(x_1)-y\| -\|f(0) - y\| = \sqrt{w(x_1)} + \sqrt{w(0)} \folgt \sqrt{w(0)} < \sqrt{w(x_1)} \folgt w(0) < w(x_1) \overset{\text{(5)}}{\le} w(0)$, Widerspruch. Also: $x_1\in K$

(5) $\folgt w(x_1) \le w(x) \ \forall x\in K$. 8.1 $\folgt w'(x_1) = 0 \folgt \left( f(x_1) - y \right) \cdot f'(x_1) = 0$; (1) $\folgt f'(x_1)$ ist invertierbar $\folgt y = f(x_1) \folgt x_1 \in U \folgt y=f(x_1) \in f(U)$. Also: $f(U) = V$. Damit ist (a) gezeigt.

% Laut Schmöger 5.6, bei uns 5.5. Wessen Zählung ist falsch? Wer Lust hat, mal überprüfen
(b): Wegen 5.5 und 9.2 ist nur zu zeigen: $f^{-1}$ ist differenzierbar auf $V$. Sei $y_1 \in V$, $y \in V\backslash\{y_1\}$, $x_1 := f^{-1}(y_1)$, $x := f^{-1}(y)$; $L(y) := \frac{f^{-1}(y) - f^{-1}(y_1) - f'(x_0)^{-1}(y-y_1)}{\|y-y_1\|}$. zu zeigen: $L(y) \to 0 \ (y-y_1)$. $\varrho(x) := f(x)-f(x_1)-f'(x_1)(x-x_1)$. $f$ ist differenzierbar in $x_1$ $\folgt \frac{\varrho(x)}{\|x-x_1\|} \to 0 \ (x\to x_1)$.

$$f'(x_1)^{-1}\varrho(x) = f'(x_1)^{-1}(y-y_1) - (f^{-1}(y) - f^{-1}(y_1)) = -\|y-y_1\| L(y)$$
$$\folgt L(y) = -f'(x_1)^{-1} \frac{\varrho(x)}{\|y-y_1\|} = - f'(x_1)^{-1} \underbrace{\frac{\varrho(x)}{\|x-x_1\|}}_{\to 0\ (x\to x_1)} \cdot \underbrace{\frac{\|x-x_1\|}{\|f(x)-f(x_1)}}_{\le 2, \text{ nach (3)}}$$
Für $y\to y_1$, gilt (wegen (4)) $x\to x_1 \folgt L(y) \to 0$.

\end{beweis}

\begin{beispiel}

$$f(x,y) = (x \cos y, x \sin y)$$

$$f'(x,y) = \begin{pmatrix} \cos y & -x \sin y \\ \sin y & x \cos y \end{pmatrix}, \det f'(x,y) = x \cos^2 y + x \sin^2 y = x$$

$D:=\{(x,y) \in \MdR^2: x\ne 0\}$. Sei $(\xi, \eta)\in D$ 9.3 $\folgt \exists$ Umgebung $U$ von $(\xi, \eta)$ mit: $f$ ist auf $U$ injektiv $(*)$. z.B. $(\xi, \eta) = (1, \frac{\pi}{2}) \folgt f(1,\frac{\pi}2) = (0,1)$. $f'(1,\frac{\pi}2) = \begin{pmatrix}0 & -1 \\ 1 & 0 \end{pmatrix}$, $(f^{-1})(0,1) = f'(1,\frac{\pi}{2})^{-1} = \begin{pmatrix}0 & 1 \\ -1 & 0\end{pmatrix}$.

\end{beispiel}

\paragraph{Beachte:} $f$ ist auf $D$ "`lokal"'   injektiv (im Sinne von $(*)$), aber $f$ ist auf $D$ \emph{nicht} injektiv, da $f(x,y) = f(x,y+2 k\pi) \ \forall x,y\in\MdR \ \forall k\in\MdZ$.

\chapter{Implizit definierte Funktionen}
\def\grad{\mathop{\rm grad}\nolimits}
\def\MdU{\ensuremath{\mathbb{U}}}

\begin{beispiele}
\item $f(x,y)=x^2+y^2-1$. $f(x,y)=0\equizu y^2=1-x^2\equizu y=\pm\sqrt{1-x^2}$. \\
Sei $(x_0, y_0)\in\MdR^2$ mit $f(x_0, y_0)=0$ und $y_0\overset{(<)}{>}0$. Dann existiert eine Umgebung $U$ von $x_0$ und genau eine Funktion $g:U\to\MdR$ mit $g(x_0)=y_0$ und $f(x,g(x))=0\ \forall x \in U$, nämlich $g(x)=\overset{(-\sqrt{\cdots})}{\sqrt{1-x^2}}$

\textbf{Sprechweisen}: "`$g$ ist implizit durch die Gleichung $f(x,y)=0$ definiert"' oder "`die Gleichung $f(x,y)=0$ kann in der Form $y=g(x)$ aufgelöst werden"'

\item $f(x,y,z)=y+z+\log(x+z)$. Wir werden sehen: $\exists$ Umgebung $U\subseteq \MdR^2$ von $(0,1)$ und genau eine Funktion $g:U\to\MdR$ mit $g(0,-1)=1$ und $f(x,y,g(x,y))=0\ \forall\ (x,y)\in U$.
\end{beispiele}

\textbf{Der allgemeine Fall}:
Es seien $p,n\in\MdN,\ \emptyset\ne D\subseteq\MdR^{n+p},\ D$ offen, $f=(f_1,\ldots, f_p) \in C^1(D,\MdR^p)$. Punkte in $D$ (bzw. $\MdR^{n+p}$) bezeichnen wir mit $(x,y)$, wobei $x=(x_1,\ldots, x_n)\in\MdR^n$ und $y=(y_1,\ldots, y_p)\in\MdR^p$, also $(x,y)=(x_1,\dots,x_n,y_1,\ldots,y_p)$. Damit:
$$ f'=
\underbrace{
\left(
\begin{array}{ccc|}
\frac{\partial f_1}{\partial x_1} & \cdots & \frac{\partial f_1}{\partial x_n} \\
\vdots & & \vdots \\
\frac{\partial f_p}{\partial x_1} & \cdots & \frac{\partial f_p}{\partial x_n} \\
\end{array}
\right.
}_{=:\frac{\partial f}{\partial x}\ (p \times n)\text{-Matrix}}
\underbrace{
\left.
\begin{array}{ccc}
\frac{\partial f_1}{\partial y_1} & \cdots & \frac{\partial f_1}{\partial y_p} \\
\vdots & & \vdots \\
\frac{\partial f_p}{\partial y_1} & \cdots & \frac{\partial f_p}{\partial y_p} \\
\end{array}
\right)
}_{=:\frac{\partial f}{\partial y}\ (p\times p)\text{-Matrix}}
\text{; also } f'(x,y)=\left(\frac{\partial f}{\partial x}(x,y),\ \frac{\partial f}{\partial y}(x,y)\right)$$

\begin{satz}[Satz über implizit definierte Funktionen]
Sei $f:D \rightarrow \MdR^p,\ f \in C^1(D, \MdR^p),\ (x_0, y_0) \in D,\ f(x_0, y_0)=0$ und
$\det\frac{\partial f}{\partial y}(x_0, y_0)\ne 0$. \\
Dann existiert eine offene Umgebung $U\subseteq \MdR^n$ von $x_0$ und
genau eine Funktion $g:U\to D \subseteq \MdR^p$ mit:
    \begin{enumerate}
        \item $(x, g(x))\in D\ \forall x\in U$
        \item $g(x_0)=y_0$
        \item $f(x,g(x))=0\ \forall x\in U$, mit $V = g(U)$ gilt:
              $V$ ist offen und für $(a, b) \in U \times V$ mit
              $f(a,b) = 0$ gilt: $b = g(a)$
        \item $g \in C^1(U,\MdR^p)$
        \item $\det\frac{\partial f}{\partial y}(x, g(x))\ne0\ \forall x\in U$
        \item $g'(x)=-\left(\frac{\partial f}{\partial y}(x, g(x))\right)^{-1} \cdot \frac{\partial f}{\partial x}(x, g(x))\ \forall x\in U$
    \end{enumerate}
\end{satz}

\begin{beweis}
Definition: $F:D\to\MdR^{n+p}$ durch $F(x,y):=(x,f(x,y))$. Dann: $F\in C^1(D,\MdR^{n+p})$ und
$$
F'(x,y)=\left(\begin{array}{c|c}
\begin{array}{ccc}
1  &        & 0 \\
   & \ddots &   \\
0  &        & 1 \\
\end{array} &
\begin{array}{ccc}
0  & \cdots & 0 \\
\vdots   &  & \vdots  \\
0  & \cdots & 0 \\
\end{array} \\
\hline\\
\ds\frac{\partial f}{\partial x}(x,y)&
\ds\frac{\partial f}{\partial y}(x,y)
\end{array}
\right)$$
Dann: \begin{enumerate}
\item[(I)] $\det F'(x,y)\gleichnach{LA}\det\frac{\partial f}{\partial y}(x,y)\ ((x, y) \in D)$, insbesondere: $\det F'(x_0, y_0)\ne 0$. Es ist $F(x_0, y_0)=(x_0, 0)$. 9.3$\folgt\exists$ eine offene Umgebung $\MdU$ von $(x_0, y_0)$ mit: $\MdU\subseteq D, f(\MdU)=\vartheta$. $F$ ist auf $\MdU$ injektiv, $F^{-1}:\vartheta\to\MdU$ ist stetig differenzierbar und
\item[(II)] $\det F'(x,y)\gleichnach{(I)}\det\frac{\partial f}{\partial y}(x,y)\ne 0\ \forall\ (x,y)\in\MdU$
\end{enumerate}
\textbf{Bezeichnungen}: Sei $(s,t)\in\vartheta\ (s\in\MdR^n, t\in\MdR^p)$, $F^{-1}(s,t)=:(u(s,t),v(s,t))$, also $u:\vartheta\to\MdR^n$ stetig differenzierbar, $v:\vartheta\to\MdR^p$ stetig differenzierbar. Dann: $(s,t)=F(F^{-1}(s,t))=(u(s,t),f(u(s,t),v(s,t)))\folgt u(s,t)=s\folgt F^{-1}(s,t)=(s,v(s,t))$. Für $(x,y)\in\MdU: f(x,y)=0\equizu F(x,y)=(x,0)\equizu(x,y)=F^{-1}(x,0)=(x,v(x,0))\equizu y=v(x,0)$, insbesondere: $y_0=v(x_0,0)$. $U:=\{x\in\MdR^n: (x,0)\in\vartheta\}$. Es gilt: $x_0\in U$. Übung: $U$ ist eine offene Umgebung von $x_0$.

\textbf{Definition}: $g:U\to\MdR^p$ durch $g(x):=v(x,0)$, für $x\in U$ gilt: $(x,0)\in\vartheta\folgt F^{-1}(x,0)=(x,v(x,0))=(x,g(x))\in \MdU$. Dann gelten: (1), (2), (3) und (4). (5) folgt aus (II).

Zu (6): Definition für $x\in U: \psi(x):=(x,g(x)), \psi\in C^1(U,\MdR^{n+p}),$
$$\psi'(x)=\left(\begin{array}{c}
\begin{array}{ccc}
1 & & 0 \\
& \ddots & \\
0 & & 1 \\
\end{array}\\
\hline \\
\ds{g'(x)}
\end{array}\right)$$
(3)$\folgt 0=f(\psi(x))\ \forall x\in U$. 5.4$\folgt 0=f'(\psi(x))\cdot\psi'(x)=\left(\frac{\partial f}{\partial x}(x, g(x))\ \vline \frac{\partial f}{\partial y}(x, g(x))\right)\cdot\psi'(x)\gleichnach{LA}\frac{\partial f}{\partial x}(x, g(x)) + \frac{\partial f}{\partial y}(x, g(x))\cdot g'(x)\ \forall x\in U$. (5) $\folgt \frac{\partial f}{\partial y}(x, g(x))$ invertierbar, Multiplikation von links mit $\frac{\partial f}{\partial y}(x, g(x))^{-1}$ liefert (6).
\end{beweis}

\begin{beispiel}
$f(x,y,z)=y+z+\log(x+z)$. Zeige: $\exists$ offene Umgebung $U$ von $(0,1)$ und genau eine stetig differenzierbare Funktion $g:U\to\MdR$ mit $g(0,-1)=1$ und $f(x,y,g(x,y))=0\ \forall (x,y)\in U$. Berechne $g'$ an der Stelle $(0,-1)$.\\
$f(0,-1,1)=0$, $f_z=1+\frac{1}{x+z}$; $f_z(0,-1,1)=2\ne 0$. Die Behauptung folgt aus dem Satz über impliziert definierte Funktionen. Also: $0=y+g(x,y)+\log(x+g(x,y))\ \forall (x,y)\in U$.\\
Differentiation nach $x$: $0=g_x(x,y)+\frac{1}{x+g(x,y)}(1+g_x(x,y))\ \forall (x,y)\in U\overset{(x,y)=(0,-1)}{\folgt}0=g_x(0,-1)+\frac{1}{1}(g_x(0,-1)+1)\folgt g_x(0,-1)=-\frac{1}{2}$.\\
Differentiation nach $y$: $0=1+g_y(x,y)+\frac{1}{x+g(x,y)}g_y(x,y)\ \forall (x,y)\in U \folgtnach{(x,y)=(0,-1)}g_y(0,-1)=-\frac{1}{2}$. Also: $g'(0,-1)=(-\frac{1}{2},-\frac{1}{2})$.
\end{beispiel}

\chapter{Extremwerte unter Nebenbedingungen}

\begin{definition}
\indexlabel{Einschränkung einer Funktion}
Seien $M,N$ Mengen $\ne \emptyset,\ f:M\to N$ eine Funktion und
$\emptyset \ne T \subseteq M$. Die Funktion
$f_{|_T}: T \to N,\ f_{|_T}(x) := f(x)\ \forall x \in T$ heißt die
\textbf{Einschränkung} von $f$ auf $T$.
\end{definition}

In diesem Paragraphen gelte stets: $\emptyset \ne D \subseteq \MdR^n,\ D$ offen, $f \in C^1(D,\MdR),\ p \in \MdN,\ p<n$ und $\varphi = (\varphi_1,\ldots,\varphi_p) \in C^1(D,\MdR^p)$. Es sei $T:=\{x\in D: \varphi(x) = 0\} \ne \emptyset.$

\begin{definition}
    \indexlabel{lokal!Extremum unter einer Nebenbedingung}
    $f$ hat in $x_0\in D$ ein \textbf{lokales Extremum unter der Nebenbedingung
    $\varphi = 0$} $:\equizu x_0 \in T$ und $f_{|_T}$ hat in $x_0$ ein lokales Extremum.
\end{definition}

Wir führen folgende Hilfsfunktion ein: Für $x=(x_1,\ldots,x_n) \in D$ und
$\lambda = (\lambda_1,\ldots,\lambda_p) \in \MdR^p$ gilt:
\[ H(x,\lambda) := f(x) + \lambda\cdot\varphi(x) = f(x) + \lambda_1\varphi_1(x) + \cdots + \lambda_p\varphi_p(x)\]

Es ist
\[ H_{x_j} = f_{x_j} + \lambda_1\frac{\partial\varphi_1}{\partial x_j} + \cdots +\lambda_p\frac{\partial\varphi_p}{\partial x_j}\ (j=1,\ldots,n),\ H_{\lambda_j} = \varphi_j ~~~ (j = 1, \dots, p)\]

Für $x_0 \in D$ und $\lambda_0 \in \MdR^p$ gilt:

$H'(x_0,\lambda_0) = 0 \equizu f'(x_0) + \lambda_0\varphi'(x_0) = 0$. Außerdem gilt:\\
$\varphi(x_0) = 0 \equizu f'(x_0) + \lambda_0\varphi'(x_0) = 0$ und $x_0 \in T$ (I)

\begin{satz}[Multiplikationenregel von Lagrange]
    \indexlabel{Multiplikator}
    $f$ habe in $x_0\in D$ eine lokales Extremum unter der Nebenbedingung
    $\varphi=0$ und es sei Rang $\varphi'(x_0) = p$. Dann existiert ein
    $\lambda_0 \in \MdR^p$ mit: $H'(x_0,\lambda_0) = 0$ ($\lambda_0$ heißt \textbf{Multiplikator}).
\end{satz}

\begin{beispiel}
    \begin{align*}
        f(x, y)       &:= x^2 + y^2\\
        \varphi(x, y) &:= x + y -1\\
    \end{align*}
    (Also $n=2, p=1$), $\varphi(x,y)' = (1,1)$, $\text{Rang } \varphi(x,y)' = 1$\\
    $H(\uwave{x,y}, \lambda) = x^2 + y^2 + \lambda (x+y-1)$\\
    \begin{align*}
        H_x       &= 2x + \lambda \stackrel{!}{=} 0\\
        H_y       &= 2y + \lambda \stackrel{!}{=} 0\\
        H_\lambda &= x+y-1 \stackrel{!}{=} 0\\
    \end{align*}
    $\Rightarrow x=y \Rightarrow 2x-1 = 0 \Rightarrow x = y = \frac{1}{2}$.\\
    Extremwertverdächtig: $(\frac{1}{2}, \frac{1}{2})$
\end{beispiel}

\begin{folgerung}
$T$ sei kompakt \folgtwegen{3.3} $\exists a,b \in T: f(a) = \max f(T),\ f(b) = \min f(T).$
Ist Rang $\varphi'(\underset{b}{a}) = p \folgt \exists \lambda_0 \in \MdR^p: H'(\underset{b}{a},\lambda_0) = 0$.
\end{folgerung}

\begin{beweis}
Es ist $x_0 \in T$ und

$$\varphi'(x_0) =
\underbrace{
\left(
\begin{array}{ccc|}
\frac{\partial \varphi_1}{\partial x_1}(x_0) & \cdots & \frac{\partial \varphi_1}{\partial x_p}(x_0)\\
\vdots &  & \vdots\\
\frac{\partial \varphi_p}{\partial x_1}(x_0) & \cdots & \frac{\partial \varphi_p}{\partial x_p}(x_0)\\
\end{array}
\right.
}_{=:A}
\left.
\begin{array}{cc}
\cdots & \frac{\partial \varphi_1}{\partial x_n}(x_0)\\
 & \vdots\\
\cdots & \frac{\partial \varphi_p}{\partial x_n}(x_0)\\
\end{array}
\right)$$

Rang $\varphi'(x_0) = p \folgt$ o.B.d.A.: $\det A \ne 0.$

Für $x=(x_1,\ldots,x_n) \in D$ schreiben wir $x=(y,z)$, wobei $y=(x_1,\dots,x_p),\ z=(x_{p+1},\ldots,x_n).$ Insbesondere ist $x_0=(y_0,z_0)$. Damit gilt: $\varphi(y_0,z_0) = 0$ und $\det \frac{\partial \varphi}{\partial y}(y_0,z_0) \ne 0$.

Aus 10.1 folgt: $\exists$ offene Umgebung $U \subseteq \MdR^{n-p}$ von $z_0,\ \exists$ offene Umgebung $V \subseteq \MdR^p$ von $y_0$ und es existiert $g \in C^1(U,\MdR^p)$ mit:

\begin{itemize}
\item[(II)] $g(z_0)=y_0$
\item[(III)] $\varphi(g(z),z) = 0\ \forall z \in U$
\item[(IV)] $g'(z_0) = -(\frac{\partial \varphi}{\partial y}\underbrace{(g(z_0),z_0)}_{=x_0})^{-1}\frac{\partial \varphi}{\partial z}\underbrace{(g(z_0),z_0)}_{=x_0}$
\end{itemize}

(III) $\folgt (g(z),z) \in T\ \forall z \in U$. Wir definieren $h(z)$ durch
$$h(z) := f(g(z),z)\ (z \in U)$$

Dann hat $h$ in $z_0$ ein lokales Extremum (\emph{ohne} Nebenbedingung). Damit gilt nach 8.1:

$$0=h'(z_0) \gleichnach{5.4} f'(g(z_0),z_0)\cdot\left(
\begin{array}{c}
g'(z_0)\\
I
\end{array}\right) = \left(
\begin{array}{c|c}
\frac{\partial f}{\partial y}(x_0) & \frac{\partial f}{\partial z}(x_0)
\end{array}\right) \cdot \left(
\begin{array}{c}
g'(z_0)\\
I
\end{array}\right) = \frac{\partial f}{\partial y}(x_0) g'(z_0) + \frac{\partial f}{\partial z}(x_0)$$

$$\gleichnach{(IV)} \underbrace{\frac{\partial f}{\partial y}(x_0) \left(-\frac{\partial \varphi}{\partial y}(x_0)\right)^{-1}}_{=: \lambda_0 \in \MdR^p} \frac{\partial \varphi}{\partial z}(x_0) + \frac{\partial f}{\partial z}(x_0) \folgt \frac{\partial f}{\partial z}(x_0) + \lambda_0 \frac{\partial \varphi}{\partial z}(x_0) = 0\text{ (V)}$$

$\ds{\lambda_0 = \frac{\partial f}{\partial y}(x_0) \left(-\frac{\partial \varphi}{\partial y}(x_0)\right)^{-1} \folgt \frac{\partial f}{\partial y}(x_0) + \lambda_0 \frac{\partial \varphi}{\partial y}(x_0) = 0}$ (VI)

Aus (V),(VI) folgt: $f'(x_0) + \lambda_0\varphi'(x_0) = 0 \folgtnach{(I)} H'(x_0,\lambda_0) = 0.$

\end{beweis}

\begin{beispiel}
$(n=3,p=2)\ f(x,y,z) = x+y+z,\ T:=\{(x,y,z) \in \MdR^3: x^2+y^2=2,\ x+z=1\},\ \varphi(x,y,z) = (x^2+y^2-2,x+z-1).$

Bestimme $\max f(T),\ \min f(T)$. Übung: $T$ ist beschränkt und abgeschlossen $\folgtnach{3.3} \exists a,b \in T: f(a) = \max f(T),\ f(b) = \min f(T)$.

$$\varphi'(x,y,z) = \left(\begin{array}{ccc}
2x & 2y & 0\\
1  & 0  & 1
\end{array}\right)$$

Rang $\varphi'(x,y,z) = 1 < p=2 \equizu x=y=0.\ a,b\in T \folgt$ Rang $\varphi'(a) =$ Rang $\varphi'(b) = 2$

\def\shouldbe{\overset{!}{=}}

$H(x,y,z,\lambda_1,\lambda_2) = x+y+z+\lambda_1(x^2+y^2-2) + \lambda_2(x+z-1)$\\
\begin{tabbing}
$H_x=1+2\lambda_1x+\lambda_2$ \= $\shouldbe 0$ (1)\\
$H_y=1+2\lambda_1y          $ \> $\shouldbe 0$ (2)\\
$H_z=1+\lambda_2            $ \> $\shouldbe 0$ (3)\\
$H_{\lambda_1}=x^2+y^2-2    $ \> $\shouldbe 0$ (4)\\
$H_{\lambda_2}=x+z-1        $ \> $\shouldbe 0$ (5)
\end{tabbing}

(3) $\folgt \lambda_2=-1 \folgtnach{(1)} 2\lambda_1x=0$; (2) $\folgt \lambda_1\ne0 \folgt x=0 \folgtnach{(5)} z=1$; (4) $\folgt y = \pm \sqrt{2}$

11.2 $\folgt a,b \in \{(0,\sqrt{2},1),(0,-\sqrt{2},1)\}$

$f(0,\sqrt{2},1) = 1+\sqrt{2} = \max f(T);\ f(0,-\sqrt{2},1) = 1-\sqrt{2} = \min f(T)$

\end{beispiel}

\paragraph{Anwendung}
Sei $A$ eine reelle, \emph{symmetrische} $(n\times n)$-Matrix. Beh: $A$ besitzt einen reellen EW.

\begin{beweis}
$f(x) := x\cdot(Ax) = Q_A(x)\ (x \in \MdR^n),\ T:= \{x \in \MdR^n: \|x\|=1\} = \partial U_1(0)$ ist beschränkt und abgeschlossen.

$\varphi(x) := \|x\|^2-1 = x\cdot x-1;\ \varphi'(x) = 2x,\ f'(x) = 2Ax$.

3.3 $\folgt \exists x_0 \in T: f(x_0) = \max f(T);\ \varphi'(x) = 2(x_1,\ldots,x_n);\ x_0 \in T \folgt$ Rang $\varphi'(x_0) = 1\ (=p)$

11.2 $\folgt \exists \lambda_0 \in \MdR: H'(x_0,\lambda_0) = 0;\ h(x,\lambda) = f(x)+\lambda\varphi(x);\ H'(x,\lambda) = 2Ax+2\lambda x$

$\folgt 0 = 2(Ax_0+\lambda_0 x_0) \folgt Ax_0 = (-\lambda_0) x_0,\ x_0 \ne 0 \folgt -\lambda_0$ ist ein EW von $A$.
\end{beweis}
\chapter{Wege im $\MdR^n$}

\indexlabel{Weg-}
\indexlabel{Bogen}
\indexlabel{Anfangspunkt}
\indexlabel{Endpunkt}
\indexlabel{Weg-!inverser}
\indexlabel{Inverser Weg}
\indexlabel{Parameter-!Intervall}
\begin{definition}
\begin{enumerate}
\item Sei $[a,b]\subseteq \MdR$ und $\gamma: [a,b] \to \MdR^n$ sei stetig. Dann heißt $\gamma$ ein \textbf{Weg} im $\MdR^n$
\item Sei $\gamma :[a,b] \to \MdR^n$ ein Weg. $\Gamma_\gamma := \gamma([a,b])$ heißt der zu $\gamma$ gehörende \textbf{Bogen}, $\Gamma_\gamma \subseteq \MdR^n$. 3.3 \folgt{} $\Gamma_\gamma$ ist beschränkt und abgeschlossen.
$\gamma(a)$ heißt der \textbf{Anfangspunkt} von $\gamma$, $\gamma(b)$ heißt der \textbf{Endpunkt} von $\gamma$. $[a,b]$ heißt \textbf{Parameterintervall} von $\gamma$. \\
$\gamma$ heißt \textbf{geschlossen} $:\equizu \gamma(a) = \gamma(b)$.
\item $\gamma^-:[a,b]\to \MdR^n$, definiert durch $\gamma^-(t):=\gamma(b+a-t)$ heißt der zu $\gamma$ \textbf{inverse Weg}. Beachte: $\gamma^- \ne \gamma$, aber $\Gamma_\gamma = \Gamma_{\gamma^-}$.
\end{enumerate}
\end{definition}

\begin{beispiele}
\item Sei $x_0, y_0\in\MdR^n$, $\gamma(t) := x_0 + t(y_0-x_0)$, $t\in[0,1]$. $\Gamma_\gamma=S[x_0,y_0]$
\item Sei $r>0$ und $\gamma(t) := (r \cos t, r \sin t)$, $t\in[0,2\pi]$\\ $\Gamma_\gamma=\{(x,y)\in\MdR^2: x^2+y^2=r^2\} = \partial U_r(0)$ \\
$\tilde\gamma(t) := (r \cos t, r \sin t)$, $t\in[0,4\pi]$. $\tilde\gamma \ne \gamma$, aber $\Gamma_{\tilde\gamma} = \Gamma_\gamma$.
\item Sei $f:[a,b] \to \MdR$ stetig und $\gamma(t) := (t, f(t)) \quad (t \in [a,b])$. Dann: $\Gamma_\gamma = $ Graph von $f$.
\end{beispiele}

\begin{erinnerung}
$\Z$ ist die Menge aller Zerlegungen von $[a,b]$
\end{erinnerung}

\begin{definition}
Sei $\gamma:[a,b]\to \MdR^n$ ein Weg. Sei $Z=\{t_0, \ldots, t_m\} \in \Z$.\\
$$L(\gamma,Z):= \sum_{j=1}^m\|\gamma(t_j) - \gamma(t_{j-1})\|$$
Übung: Sind $Z_1,Z_2\in\Z$ und gilt $Z_1 \subseteq Z_2 \folgt L(\gamma,Z_1) \le L(\gamma,Z_2)$

$\gamma$ heißt \textbf{rektifizierbar} (rb) \indexlabel{Rektifizierbarkeit} $:\equizu$ $\exists M\ge0: L(\gamma,Z)\le M\ \forall Z\in\Z$. In diesem Fall heißt $L(\gamma) := \sup\{L(\gamma,Z): Z\in\Z\}$ die \textbf{Länge} von $\gamma$\indexlabel{Länge}.

Ist $n=1$, so gilt: $\gamma$ ist rektifizierbar $\equizu$ $\gamma\in \BV[a,b]$. In diesem Fall: $L(\gamma) = V_\gamma([a,b])$.
\end{definition}

\begin{satz}[Rektifizierbarkeit und Beschränkte Variation]
Sei $\gamma = (\eta_1, \ldots, \eta_n):[a,b]\to\MdR^n$ ein Weg. $\gamma$ ist rektifizierbar $\equizu$ \mbox{$\eta_1,\ldots,\eta_n\in \BV[a,b]$}.
\end{satz}

\begin{beweis}
Sei $Z = \{t_0,\ldots,t_n\} \in \Z$ und $J=\{1,\ldots,n\}$.\\
$|\eta_j(t_k)-\eta_j(t_{k-1})| \stackrel{\text{1.7}}{\le} \|\gamma(t_k) - \gamma(t_{k-1})\| \stackrel{\text{1.7}}\le \sum_{\nu=1}^n |\eta_\nu(t_k) - \eta_\nu(t_{k-1})\|$. Summation über $k$ $\folgt$ $V_{\eta_j} \le L(\gamma,Z) \le \sum_{k=1}^m\sum_{\nu=1}^n |\eta_\nu(t_k)-\eta_\nu(t_{k-1})| = \sum_{\nu=1}^n V_{\eta_\nu}(Z) \folgt$ Behauptung
\end{beweis}
\paragraph{Übung:} $\gamma$ ist rektifizierbar $\equizu$ $\gamma^-$ ist rektifizierbar. In diesem Fall: \mbox{$L(\gamma) = L(\gamma^-)$}

\paragraph{Summe von Wegen:}Gegeben: $a_0, a_1,\ldots a_l \in \MdR$, $a_0<a_1<a_2<\ldots<a_l$ und Wege $\gamma_k:[a_{k-1},a_k] \to \MdR^n$ $(k=1,\ldots,l)$ mit : $\gamma_k(a_k) = \gamma_{k+1}(a_k)$ $(k=1,\ldots,l-1)$.
Definiere $\gamma:[a_0,a_l]\to\MdR^n$ durch $\gamma(t):=\gamma_k(t)$, falls $t\in[a_{k-1},a_k]$. $\gamma$ ist ein Weg im $\MdR^n$, $\Gamma_\gamma = \Gamma_{\gamma_1} \cup \Gamma_{\gamma_2} \cup \cdots \cup \Gamma_{\gamma_l}$. $\gamma$ heißt die Summe der Wege $\gamma_1,\ldots,\gamma_l$ und wird mit. $\gamma = \gamma_1 \oplus \gamma_2 \oplus \cdots \oplus \gamma_l$ bezeichnet. \indexlabel{Summe von Wegen}

\begin{bemerkung}
Ist $\gamma:[a,b]\to\MdR^n$ ein Weg und $Z=\{t_0,\ldots,t_m\}\in\Z$ und $\gamma_k:=\gamma_{|_{[t_{k-1},t_k]}}$ $(k=1,\ldots,m)$ $\folgt$ $\gamma = \gamma_1 \oplus \cdots \oplus \gamma_m$. Aus Analysis I, 25.1(7) und 12.1 folgt:
\end{bemerkung}

\begin{satz}[Summe von Wegen]
Ist $\gamma = \gamma_1 \oplus \cdots \oplus \gamma_m$, so gilt: $\gamma$ ist rektifizierbar $\equizu$ $\gamma_1,\ldots,\gamma_m$ sind rektifizierbar. In diesem Fall: $L(\gamma)=L(\gamma_1) + \cdots + L(\gamma_m)$
\end{satz}

\begin{definition}
\index{Weg-!Längenfunktion}
Sei $\gamma:[a,b]\to\MdR^n$ ein rektifizierbarer Weg. Sei $t\in(a,b]$. Dann: $\gamma_{|_{[a,t]}}$ ist rektifizierbar (12.2).
$$s(t):= \begin{cases}L(\gamma_{|_{[a,t]}}),&\text{falls }t\in(a,b] \\0, &\text{falls }t=a\end{cases}$$ heißt die zu $\gamma$ gehörende \textbf{Weglängenfunktion}.
\end{definition}

\begin{satz}[Eigenschaften der Weglängenfunktion]
Sei $\gamma:[a,b]\to\MdR^n$ ein rektifizierbarer Weg. Dann:
\begin{enumerate}
\item $s\in C[a,b]$
\item $s$ ist wachsend.
\end{enumerate}
\end{satz}

\begin{beweis}
\begin{enumerate}
\item \textbf{\color{red}In der großen Übung}
\item Sei $t_1, t_2 \in [a,b]$ und $t_1<t_2$. $\gamma_1:=\gamma_{|_{[a,t_1]}}$, $\gamma_2:=\gamma_{|_{[t_1,t_2]}}$, $\gamma_3:=\gamma_{|_{[a,t_2]}}$. Dann $\gamma_3 = \gamma_1 \oplus \gamma_2$. 12.2 $\folgt$ $\gamma_1,\gamma_2,\gamma_3$ sind rektifizierbar und $\underbrace{L(\gamma_3)}_{=s(t_2)} = \underbrace{L(\gamma_1)}_{s(t_1)} + \underbrace{L(\gamma_2)}_{\ge 0} \folgt s(t_2) \ge s(t_1)$.
\end{enumerate}
\end{beweis}

\begin{satz}[Rechenregeln für Wegintegrale]
Sei $f=(f_1,\ldots,f_n):[a,b]\to\MdR^n$ und $f_1,\ldots,f_n\in R[a,b]$.
$$\int_a^bf(t)dt := \left(\int_a^bf_1(t)dt, \int_a^bf_2(t)dt,\ldots, \int_a^bf_n(t)dt\right) \quad (\in\MdR^n)$$
Dann: \begin{enumerate}
\item $$x\cdot \int_a^bf(t)dt = \int_a^b(x\cdot f(t))dt \ \forall x\in\MdR^n$$
\item $$\left\|\int_a^bf(t)dt\right\| \le \int_a^b\|f(t)\|dt$$
\end{enumerate}
\end{satz}

\begin{beweis}
\begin{enumerate}
\item Sei $x=(x_1,\ldots,x_n) \folgt$\\ $x\cdot\int_a^b f(t)dt = \sum_{j=1}^n x_j\int_a^bf_j(t) dt = \int_a^b\left(\sum_{j=1}^n x_j f_j(t)dt\right) = \int_a^b \left(x\cdot f(t)\right) dt$
\item $y:=\int_a^bf(t)dt$. O.B.d.A: $y\ne 0$. $x:= \frac{1}{\|y\|} y \folgt \|x\|=1, y=\|y\|x$. $\|y\|^2 = y\cdot y = \|y\|(x\cdot y) = \|y\|\left(x\cdot \int_a^bf(t)dt \right) = \|y\|\int_a^b\left(x\cdot f(t)\right) dt \le \|y\|\int_a^b\underbrace{|x \cdot f(t)|}_{\le\|x\|\|f(t)\| = \|f(t)\|} dt \le \|y\| \int_a^b\|f(t)\|dt$
\end{enumerate}
\end{beweis}


\begin{satz}[Eigenschaften stetig differenzierbarer Wege]
$\gamma:[a,b]\to\MdR^n$ sei ein stetig differenzierbarer Weg. Dann:
\begin{enumerate}
\item $\gamma$ ist rektifizierbar
\item Ist $s$ die zu $\gamma$ gehörende Weglängenfunktion, so ist $s\in C^1[a,b]$ und $s'(t)=\|\gamma'(t)\|\ \forall t\in[a,b]$
\item $L(\gamma)=\int_a^b\|\gamma'(t)\|dt$
\end{enumerate}
\end{satz}

\begin{beweise}
\item $\gamma=(\eta_1,\ldots,\eta_n)$, $\eta_j\in C^1[a,b]\folgtnach{A1,25.1}\eta_j\in \text{BV}[a,b]\folgtnach{12.1}\gamma$ ist rektifizierbar.
\item Sei $t_0\in[a,b)$. Wir zeigen:
$$\frac{s(t)-s(t_0)}{t-t_0}\to\|\gamma'(t_0)\|\ (t\to t_0+0)\text{. (analog zeigt man :}\frac{s(t)-s(t_0)}{t-t_0}\to\|\gamma'(t_0)\|\ (t\to t_0-0)\text{).}$$
Sei $t\in (t_0, b];\ \gamma_1:=\gamma_{|_{[a,t_0]}}, \gamma_2:=\gamma_{|_{[t_0,t]}}, \gamma_3:=\gamma_{|_{[a,t]}}$. Dann: $\gamma_3=\gamma_1 \oplus \gamma_2$ und $\underbrace{L(\gamma_3)}_{=s(t)}=\underbrace{L(\gamma_1)}_{=s(t_0)}+L(\gamma_2)\folgt s(t)-s(t_0)=L(\gamma_2)\ (I).$\\
$\tilde{Z}:=\{t_0, t\}$ ist eine Zerlegung von $[t_0,t]\folgt \|\gamma(t)-\gamma(t_0)\|=L(\gamma_2,\tilde{Z})\le L(\gamma_2)$\\
\textbf{Definition}: $F:[a,b]\to\MdR$ durch $F(t)=\ds\int_a^t\|\gamma'(\tau)\|\text{d}\tau$. 2.Hauptsatz der Differential- und Integralrechnung $\folgt F$ ist differenzierbar und $F'(t)=\|\gamma'(t)\|\ \forall t\in[a,b]$. Sei $Z=\{\tau_0,\ldots,\tau_m\}$ eine beliebige Zerlegung von $[t_0, t]$.
$$\ds\int_{\tau_{j-1}}^{\tau_j}\gamma'(\tau)\text{d}\tau=\left(\cdots, \ds\int_{\tau_{j-1}}^{\tau_j}\eta_k'(\tau)\text{d}\tau,\cdots\right)\gleichnach{A1}\left(\cdots, \eta_k(\tau_j)-\eta_k(\tau_{j-1}),\cdots\right)=\gamma(\tau_j)-\gamma(\tau_{j-1})$$
$\folgt \|\gamma(\tau_j)-\gamma(\tau_{j-1})\|\overset{12.4}{\le}\ds\int_{\tau_{j-1}}^{\tau_j}\|\gamma'(\tau)\|\text{d}\tau$. Summation $\folgt L(\gamma_2,Z)\le\ds\int_{t_0}^t\|\gamma'(\tau)\|\text{d}\tau=F(t)-F(t_0)\folgt L(\gamma_2)\le F(t)-F(t_0)\ (III)$.\\
$(I), (II), (III)\folgt\|\gamma(t)-\gamma(t_0)\|\overset{(II)}{\le}L(\gamma_2)\overset{(I)}{=}s(t)-s(t_0)\overset{(III)}{\le}F(t)-F(t_0)$
$$\folgt\underbrace{\frac{\|\gamma(t)-\gamma(t_0)\|}{t-t_0}}_{\overset{t\to t_0}{\to}\|\gamma'(t_0)\|}\le\frac{s(t)-s(t_0)}{t-t_0}\le\underbrace{\frac{F(t)-F(t_0)}{t-t_0}}_{\overset{t\to t_0}{\to}F'(t_0)=\|\gamma'(t_0)\|}$$
$(3)\ L(\gamma)=s(b)=s(b)-s(a)\overset{AI}{=}\ds\int_a^b s'(t)\text{d}t\gleichnach{(2)}\ds\int_a^b\|\gamma'(t)\|\text{d}t$
\end{beweise}

\begin{beispiele}
\item $x_0, y_0\in\MdR^n, \gamma(t):=x_0+t(y_0-x_0)\ (t\in [0,1])$. $\gamma'(t)=y_0-x_0\folgt L(\gamma)=\ds\int_0^1\|y_0-x_0\|\text{d}t=\|y_0-x_0\|$.
\item Sei $f:[a,b]\to\MdR$ stetig und $\gamma(t):=(t, f(t)), t\in[a,b]$. $\gamma$ ist ein Weg im $\MdR^2$. $\gamma$ ist rektifizierbar $\equizu f \in \text{BV}[a,b]$. $\Gamma_\gamma=$Graph von $f$. Jetzt sei $f\in C^1[a,b] \folgtnach{12.5} L(\gamma)=\ds\int_a^b\|\gamma'(t)\|\text{d}t=\ds\int_a^b (1+f'(t)^2)^{\frac{1}{2}}\text{d}t$.
\item $\gamma(t):=(\cos t, \sin t)\ (t\in [0,2\pi])$. $\gamma'(t)=(-\sin t, \cos t)$. $\|\gamma'(t)\|=1\ \forall t\in [0,2\pi]\folgtnach{12.5}s'(t)=1\ \forall t\in[0,2\pi]\folgt s(t)=t\ \forall t\in[0,2\pi]$ (\begriff{Bogenmaß}). \begriff{Winkelmaß}: $\varphi:=\frac{180}{\pi}t$. $L(\gamma)=2\pi$.
\end{beispiele}

\begin{definition*}
$\gamma:[a,b]\to\MdR^n$ sei ein Weg.
\begin{enumerate}
\index{stückweise!stetige Differenzierbarkeit}\index{Differenzierbarkeit!stückweise stetige}
\index{Glattheit}
\index{stückweise!Glattheit}\index{Glattheit!stückweise}
\item $\gamma$ heißt \textbf{stückweise stetig differenzierbar} $:\equizu\exists z=\{t_0,\ldots,t_m\}\in\Z$ mit: $\gamma_{|_{[t_{k-1},t_k]}}$ sind stetig differenzierbar $(k=1,\ldots,m)\equizu\exists$ stetig differenzierbare Wege $\gamma_1,\ldots,\gamma_l: \gamma=\gamma_1\oplus\cdots\oplus\gamma_l$.
\item $\gamma$ heißt \textbf{glatt} $:\equizu \gamma$ ist stetig differenzierbar und $\|\gamma'(t)\|>0\ \forall t\in[a,b]$.
\item $\gamma$ heißt \textbf{stückweise glatt} $:\equizu\exists$ glatte Wege $\gamma=\gamma_1\oplus\cdots\oplus\gamma_l$
\end{enumerate}
\end{definition*}

Aus 12.2 und 12.5 folgt:

\begin{satz}[Rektifizierbarkeit von Wegsummen]
Ist $\gamma=\gamma_1\oplus\cdots\oplus\gamma_l$ stückweise stetig differenzierbar, mit stetig differenzierbaren Wegen $\gamma_1,\ldots,\gamma_l\folgt \gamma$ ist rektifizierbar und $L(\gamma)=L(\gamma_1)+\cdots+L(\gamma_l)$.
\end{satz}

\begin{definition*}
\index{Parameter-!Darstellung}
Sei $\gamma:[a,b]\to\MdR^n$ ein Weg. $\gamma$ heißt eine \textbf{Parameterdarstellung} von $\Gamma_\gamma$.
\end{definition*}

\begin{beispiele}
\item $x_0, y_0\in\MdR^n, \gamma_1(t):=x_0+t(y_0-x_0)\ t\in[0,1],\ \gamma_2(t):=\gamma_1^-(t)\ t\in[0,1],\ \gamma_3(t):=x_0+7t(y_0-x_0)\ t\in[0,\frac{1}{7}].\ \gamma_1,\gamma_2,\gamma_3$ sind Parameterdarstellungen von $S[x_0, y_0]$.
\item $\gamma_1(t)=(\cos t, \sin t),\ (t\in [0,2\pi]), \gamma_2(t):=(\cos t, \sin t), (t\in[0,4\pi])$. $\gamma_1, \gamma_2$ sind Parameterdarstellungen von $K=\{(x,y)\in\MdR^2: x^2+y^2=1\}.$
\end{beispiele}

\begin{definition*}
$\gamma_1:[a,b]\to\MdR^n$ und $\gamma_2:[\alpha,\beta]\to\MdR^n$ seien Wege.

\index{Äquivalenz}
\index{Parameter-!Transformation}
$\gamma_1$ und $\gamma_2$ heißen \textbf{äquivalent}, in Zeichen $\gamma_1\sim\gamma_2:\equizu\exists h:[a,b]\to[\alpha, \beta]$ stetig und streng wachsend, $h(a)=\alpha, h(b)=\beta$ und $\gamma_1(t)=\gamma_2(h(t))\ \forall t\in[a,b]$ (also $\gamma_1=\gamma_2\circ h)$. $h$ heißt eine \textbf{Parametertransformation} (PTF). Analysis 1 $\folgt h([a,b])=[\alpha,\beta]\folgt \Gamma_{\gamma_1}=\Gamma_{\gamma_2}$.
Es gilt: $\gamma_2=\gamma_1\circ h^{-1}\folgt \gamma_2\sim\gamma_1$. "`$\sim$"' ist eine Äquivalenzrelation.
\end{definition*}

\begin{beispiele}
\item $\gamma_1, \gamma_2, \gamma_3$ seien wie in obigem Beispiel (1). $\gamma_1\sim\gamma_3, \gamma_1\sim\gamma_2$.
\item $\gamma_1, \gamma_2$ seien wie in obigem Beispiel (2). $\gamma_1\nsim\gamma_2$, denn $L(\gamma_1)=2\pi\ne 4\pi=L(\gamma_2)$
\end{beispiele}

\begin{satz}[Eigenschaften der Parametertransformation]
$\gamma_1:[a,b]\to\MdR^n$ und $\gamma_2:[\alpha,\beta]\to\MdR^n$ seien äquivalente Wege und $h:[a,b]\to[\alpha,\beta]$ eine Parametertransformation.
\begin{enumerate}
\item $\gamma_1$ ist rektifizierbar $\equizu \gamma_2$ ist rektifizierbar. In diesem Falle: $L(\gamma_1)=L(\gamma_2)$
\item Sind $\gamma_1$ und $\gamma_2$ glatt $\folgt h\in C^1[a,b]$ und $h'>0$.
\end{enumerate}
\end{satz}

\begin{beweise}
\item[(2)] \textbf{\color{red}In den großen Übungen.}
\item[(1)] Es genügt zu zeigen: Aus $\gamma_2$ rektifizierbar folgt: $\gamma_1$ ist rektifizierbar und $L(\gamma_1)\le L(\gamma_2)$. Sei $Z=\{t_0,\ldots,t_m\}\in\Z\folgt\tilde{Z}:=\{h(t_0),\ldots, h(t_m)\}$ ist eine Zerlegung von $[\alpha,\beta]$.
$$L(\gamma_1, Z)=\ds\sum_{j=1}^m\|\gamma_1(t_j)-\gamma_1(t_{j-1})\|=\ds\sum_{j=1}^m\|\gamma_2(h(t_j))-\gamma_2(h(t_{j-1}))\|=L(\gamma_2, \tilde{Z})\le L(\gamma_2)$$
$\folgt \gamma_1$ ist rektifizierbar und $L(\gamma_1)\le L(\gamma_2)$.
\end{beweise}

\paragraph{Weglänge als Parameter}
Es sei $\gamma:[a,b]\to\MdR^n$ ein \emph{glatter} Weg. 12.5 $\folgt \gamma$ ist rb. $L:=L(\gamma)$. 12.5 $\folgt s \in C^1[a,b]$ und $s'(t) = \|\gamma'(t)\| > 0\ \forall t\in[a,b].\ s$ ist also \emph{streng wachsend}. Dann gilt: $s([a,b]) = [0,L],\ s^{-1}:[0,L]\to[a,b]$ ist streng wachsend und stetig db. $(s^{-1})'(\sigma) = \frac{1}{s'(t)}$ für $\sigma \in [0,L],\ s(t) = \sigma.$

\begin{definition}
$\tilde{\gamma}[0,L] \to \MdR^n$ durch $\tilde{\gamma}(\sigma) := \gamma(s^{-1}(\sigma)),$ also $\tilde{\gamma} = \gamma\circ s^{-1};\ \tilde{\gamma}$ ist ein Weg im $\MdR^n$ und $\tilde{\gamma} \sim \gamma;\ \Gamma_\gamma = \Gamma_{\tilde{\gamma}}.$

12.7 $\folgt \tilde{\gamma}$ ist rb, $L(\tilde{\gamma})=L(\gamma)=L,\ \tilde{\gamma}$ ist stetig db. $\tilde{\gamma}$ heißt Parameterdarstellung von $\Gamma_\gamma$ mit der Weglänge als Parameter. Warum?
\end{definition}

Darum: Sei $\tilde{s}$ die zu $\tilde{\gamma}$ gehörende Weglängenfunktion. $\forall \sigma\in[0,L]: \tilde{\gamma}(\sigma) = \gamma(s^{-1}(\sigma)).$ Sei $\sigma\in[0,L],\ t:= s^{-1}(\sigma) \in [a,b],\ s(t) = \sigma.$

$\tilde{\gamma}(\sigma) = (s^{-1})'(\sigma)\cdot\gamma'(s^{-1}(\sigma)) = \frac{1}{s'(t)}\gamma'(t) \gleichnach{12.5} \frac{1}{\|\gamma'(t)\|}\gamma'(t) \folgt \|\gamma'(\sigma)\|=1$ ($\folgt \tilde{\gamma}$ ist glatt).

$\tilde{s}'(\gamma) \gleichnach{12.5} \|\gamma'(\sigma)\| = 1 \folgtwegen{\tilde{s}(0)=0} \tilde{s}(\sigma)=\sigma.$

Also: $\|\tilde{\gamma}'(\sigma)\| = 1,\ \tilde{s}(\sigma)=\sigma\ \forall \sigma\in[0,L].$

\begin{beispiel}
$\gamma(t) = \frac{e^t}{\sqrt{2}}(\cos t,\sin t),\ t \in [0,1];\ \gamma$ ist stetig db; Nachrechnen: $\|\gamma'(t)\|=e^t\ \forall t \in [0,1] \folgt \gamma$ ist glatt.

$s'(t) \gleichnach{12.5} \|\gamma'(t)\| = e^t \folgt s(t) = e^t+c \folgt 0=s(0) = 1+c \folgt c=-1,\ s(t) = e^t-1\ (t\in[0,1]) \folgt L=L(\gamma)=s(1)=e-1.\ e^t=1+s(t),\ t=\log (1+s(t)).$

$\tilde{\gamma}(\sigma) = \gamma(s^{-1}(\sigma)) = \gamma(\log (1+\sigma)) = \frac{1+\sigma}{\sqrt{2}}(\cos (\log(1+\sigma)),\sin (\log(1+\sigma))),\ \sigma\in[0,e-1].$
\end{beispiel}

\chapter{Wegintegrale}

In diesem Paragraphen seien alle vorkommenden Wege stets stückweise stetig differenzierbar.

\begin{definition}
Sei $\gamma:[a,b]\to\mdr^n$ ein Weg, $\gamma=(\eta_1,\ldots,\eta_n), \Gamma :=\Gamma_\gamma.$
$g:\Gamma\to\mdr$ stetig und $f=(f_1,\ldots,f_n):\Gamma\to\mdr^n$ stetig.
\begin{enumerate}
\item Für $j\in \{1,\ldots,n\}: \int_\gamma g(x) \text{ d}x_j:=\int_a^b g(\gamma(t))\cdot\eta_j'(t)\text{ d}t$
\item \begin{align*}
\int_\gamma f(x)\cdot\text{d}x &:= \int_\gamma f_1(x)\text{ d}x_1+\cdots+f_n(x)\text{ d}x_n\\
&:=\sum_{j=1}^n \int_\gamma f_j(x)\text{ d}x_j
\end{align*}
Es ist $\int_\gamma f(x)\cdot\text{d}x=\int_a^b f(\gamma(t))\cdot\gamma'(t)\text{ d}t$ und heißt das
\textbf{Wegintegral von $f$ längs $\gamma$}.
\end{enumerate}
\end{definition}

\begin{beispiel}
$f(x,y,z) := (z,y,x),\ \gamma(t) = (t,t^2,3t),\ t\in[0,1].\ f(\gamma(t)) = (3t,t^2,t),\ \gamma'(t)=(1,2t,3),\ f(\gamma(t))\cdot\gamma'(t) = 3t+2t^3+3t = 6t+2t^3$.

$\int_\gamma f(x,y,z)\cdot d(x,y,z) = \int_0^1 (6t+2t^3) dt = \frac{7}{2}.$
\end{beispiel}

\begin{satz}[Rechnen mit Wegintegralen]
$\gamma,\Gamma,f$ seien wie oben, $g:\Gamma\to\MdR^n$ sei stetig, $\hat\gamma = (\hat{\gamma}_1,\ldots,\hat{\gamma}_n): [\alpha,\beta] \to \MdR^n$ sei rektifizierbar und $\xi,\eta \in \MdR$.
\begin{enumerate}
\item $\ds{\int_\gamma(\xi f(x)+\eta g(x))\cdot dx = \xi \int_\gamma f(x)\cdot dx+\eta \int_\gamma g(x)\cdot dx}$
\item Ist $\gamma = \gamma^{(1)} \oplus \gamma^{(2)} \folgt \ds{\int_\gamma f(x)\cdot dx = \int_{\gamma^{(1)}} f(x)\cdot dx + \int_{\gamma^{(2)}} f(x)\cdot dx}$
\item $\ds{\int_{\gamma^-} f(x)\cdot dx = -\int_\gamma f(x)\cdot dx}$
\item $\ds{\left| \int_\gamma f(x)\cdot dx\right| \le L(\gamma)\cdot \max\{\|f(x)\|:x \in \Gamma\}}$
\item Ist $\ds{\hat{\gamma} \sim \gamma \folgt \int_\gamma f(x)\cdot dx = \int_{\hat{\gamma}} f(x)\cdot dx}$.
\end{enumerate}
\end{satz}

\begin{beweise}
\item klar
\item Ana I, 26.1(3)
\item nur für $\gamma$ stetig differenzierbar. $\gamma^-(t) = \gamma(b+a-t),\ t\in[a,b].$

$\int_{\gamma^-} f(x)\cdot dx = \int_a^b f(\gamma(b+a-t))\cdot \gamma'(b+a-t) (-1) dt =$ (subst. $\tau=b+a-t,\ d\tau = dt$) $= \int_b^a f(\gamma(\tau))\cdot\gamma'(\tau) d\tau = -\int_a^b f(\gamma(\tau))\cdot\gamma'(\tau) d\tau = -\int_\gamma f(x)\cdot dx.$
\item Übung
\item Sei $\hat{\gamma} = \gamma\circ h,\ h:[\alpha,\beta]\to[a,b]$ stetig und streng wachsend. $h(\alpha) = a,\ h(\beta) = b$. Nur für $\gamma$ und $h$ stetig db. Dann ist $\hat{\gamma}$ stetig db.

$\int_{\hat{\gamma}} f(x)\cdot dx = \int_\alpha^\beta f(\gamma(h(t)))\cdot \gamma'(h(t))\cdot h'(t) dt =$ (subst. $\tau = h(t),\ d\tau = h'(t)dt$) $= \int_a^b f(\gamma(\tau))\cdot \gamma'(\tau)d\tau = \int_\gamma f(x)\cdot dx.$
\end{beweise}

\begin{definition}
$\gamma,\Gamma$ seien wie immer in diesem Paragraphen. $s$ sei die zu $\gamma$ gehörende Weglängenfunktion und $g:\Gamma \to \MdR$ stetig. 12.4 $\folgt s$ ist wachsend $\folgtnach{Ana I} s \in BV[a,b];\ g\circ\gamma$ stetig $\folgtnach{Ana I, 26.6} g\circ\gamma \in R_s[a,b]$.

$$\int_\gamma g(x) ds := \int_a^b g(\gamma(t))ds(t)$$

\textbf{Integral bzgl. der Weglänge}.
\end{definition}

\begin{satz}[Rechnen mit Integralen bzgl. der Weglänge]
Seien $\gamma,g$ wie oben.
\begin{enumerate}
\item $\ds{\int_{\gamma^-} g(x) ds = \int_\gamma g(x) ds}$
\item Ist $\ds{\gamma = \gamma^{(1)} \oplus \gamma^{(2)} \folgt \int_\gamma g(x)ds = \int_{\gamma^{(1)}} g(x)ds + \int_{\gamma^{(2)}} g(x)ds}$.
\item Ist $\gamma$ stetig db $\folgt \importantbox{\ds{\int_\gamma g(x)ds = \int_a^b g(\gamma(t))\|\gamma'(t)\|dt}}$
\end{enumerate}
\end{satz}

\begin{beispiel}
$g(x,y) = (1+x^2+3y)^{1/2},\ \gamma(t) = (t,t^2),\ t\in[0,1].$

$g(\gamma(t)) = (1+t^2+3t^2)^{1/2} = (1+4t^2)^{1/2},\ \gamma'(t) = (1, 2t),\ \|\gamma'(t)\| = (1+4t^2)^{1/2} \folgt \int_\gamma g(x,y)ds = \int_0^1 (1+4t^2) dt = \frac{7}{3}$
\end{beispiel}

\paragraph{Gegeben:} $\gamma_1,\gamma_2,\ldots,\gamma_m$ rektifizierbare Wege, $\gamma_k:[a_k,b_k]\to\MdR^n$ mit $\gamma_1(b_1) = \gamma_2(a_2), \gamma_2(b_2) = \gamma_3(a_3),\ldots , \gamma_{m-1}(b_{m-1}) = \gamma_m(a_m)$. $\Gamma := \Gamma_{\gamma_1} \cup \ldots \cup \Gamma_{\gamma_m}$.

$\text{AH}(\gamma_1,\ldots,\gamma_m) := \{\gamma:\gamma$ ist ein rektifizierbarer Weg im $\MdR^n$ mit: $\Gamma_\gamma=\Gamma$, $L(\gamma)=L(\gamma_1)+\cdots+L(\gamma_m)$ und $\int_\gamma f(x)\cdot dx = \int_{\gamma_1}f(x)\cdot dx+ \cdots + \int_{\gamma_m}f(x)\cdot dx$ für \emph{jedes} stetige $f:\Gamma\to\MdR^n\}.$

Ist $\gamma\in \text{AH}(\gamma_1,\ldots,\gamma_m)$, so sagt man $\gamma$ entsteht durch \indexlabel{Aneinanderhängung}\textbf{Aneinanderhängen} der Wege $\gamma_1,\ldots,\gamma_m$.

\begin{satz}[Stetige Differenzierbarekeit der Aneinanderhängung]
$\gamma_1,\ldots,\gamma_m$ seien wie oben. Dann: $\text{AH}(\gamma_1,\ldots,\gamma_m) \ne \emptyset$. \\
Sind $\gamma_1,\ldots,\gamma_m$ stetig differenzierbar, so existiert ein stückweise stetig differenzierbarer Weg $\gamma\in \text{AH}(\gamma_1,\ldots,\gamma_m)$.
\end{satz}

\begin{beweis}
O.B.d.A: $m=2$.

Def. $h:[b_1,c] \to [a_2,b_2]$ linear wie folgt: $h(x)=px+q$, $h(b_1)=a_2$, $h(c)=b_2$. $\hat\gamma_2 := \gamma_2\circ h$. Dann: $\gamma_2\sim \hat\gamma_2$. $\gamma := \gamma_1\oplus\hat\gamma_2$. 12.2, 12.7, 13.2 $\folgt$ $\gamma\in \text{AH}(\gamma_1,\gamma_2)$.
\end{beweis}

\begin{beispiel}
In allen Beispielen sei $f(x,y)=(y,x-y)$ und $t\in[0,1]$.
\begin{enumerate}
\item $\gamma_1(t)=(t,0)$, $\gamma_2(t)=(1,t)$.

Sei $\gamma \in \text{AH}(\gamma_1,\gamma_2)$. Anfangspunkt von $\gamma$ ist (0,0), Endpunkt von $\gamma$ ist (1,1). Nachrechnen: $\int_{\gamma_1}f(x,y)\cdot d(x,y) = 0$, $\int_{\gamma_2}f(x,y)\cdot d(x,y) = \frac{1}{2}$. Also: $\int_\gamma f(x,y) \cdot d(x,y) = \frac{1}{2}$

\item $\gamma_1(t) = (0,t)$, $\gamma_2(t)=(t,1)$.

Sei $\gamma\in \text{AH}(\gamma_1,\gamma_2)$, Anfangspunkt von $\gamma$ ist (0,0), Endpunkt von $\gamma$ ist (1,1). Nachrechnen: $\int_{\gamma}f(x,y)\cdot d(x,y) = \frac{1}{2}$

\item $\gamma(t)=(t,t^3)$. Anfangspunkt von $\gamma$ ist (0,0), Endpunkt von $\gamma$ ist (1,1). Nachrechnen: $\int_\gamma f(x,y)\cdot d(x,y) = \frac{1}{2}$
\end{enumerate}
\end{beispiel}

\chapter{Stammfunktionen}

In diesem Paragraphen sei stets: $\emptyset \ne G \subseteq \MdR^n$, $G$ ein \emph{Gebiet} und $f=(f_1,\ldots,f_n): G\to\MdR^n$ stetig.

\begin{definition}
Eine Funktion $\varphi:G\to\MdR$ heißt eine \textbf{Stammfunktion (SF) von $f$ auf $G$}\indexlabel{Stammfunktion} $:\equizu$ $\varphi$ ist auf $G$ partiell differenzierbar und $\grad\varphi = f$ auf $G$. Also: $\varphi_{x_j} = f_j$ auf $G$ ($j=1,\ldots,n$).
\end{definition}

\begin{bemerkung}
\
\vspace{-1.5em}
\begin{enumerate}
\item Ist $\varphi$ eine Stammfunktion von $f$ auf $G$ $\folgt$ $\grad\varphi = f \folgt \varphi \in C^1(G,\MdR) \folgtnach{5.3} \varphi$ ist auf $G$ differenzierbar und $\varphi' = f$ auf $G$.
\item Sind $\varphi_1$, $\varphi_2$ Stammfunktionen von $f$ auf $G$ $\folgtnach{(1)}$ $\varphi_1'=\varphi_2'$ auf $G$ $\folgtnach{6.2} \exists c\in\MdR: \varphi_1=\varphi_2+c$ auf $G$
\item Ist $n=1$ $\folgt$ $G$ ist ein offenes Intervall. AI, 23.14 $\folgt$ \emph{jedes} stetige $f:G\to\MdR$ besitzt auf $G$ eine Stammfunktion! Im Falle $n\ge 2$ ist dies \emph{nicht} so.
\end{enumerate}
\end{bemerkung}

\begin{beispiele}
\item $G=\MdR^2$, $f(x,y) = (y,-x)$.

Annahme: $f$ besitzt auf $\MdR^2$ die Stammfunktion $\varphi$. Dann: $\varphi_x = y$, $\varphi_y = -x$ auf $G$ $\folgt$ $\varphi\in C^2(\MdR^2,\MdR)$ und $\varphi_{xy} = 1 \ne -1 = \varphi_{yx}$. Widerspruch zu 4.1. Also: $f$ besitzt auf $\MdR^2$ \emph{keine} Stammfunktion.
\item $G=\MdR^2$, $f(x,y) = (y,x-y)$.

Ansatz für eine Stammfunktion $\varphi$ von $f$: $\varphi_x=y \folgt \varphi=xy+c(y)$, $c$ differenzierbar, $\folgt$ $\varphi_y\stackrel{!}{=}x+c'(y) = x-y \folgt c'(y) = -y$, etwa $c(y)=-\frac{1}{2}y^2$. Also: $\varphi(x,y) = xy - \frac{1}{2}y^2$. Probe: $\varphi_x=y$, $\varphi_y=x-y$, also: $\grad\varphi=f$. $\varphi$ ist also eine Stammfunktion von $f$ auf $\MdR^2$.
%Weiß wer warum man da ne Probe braucht?
%Braucht man nicht, war nur um uns zu überzeugen
\end{beispiele}
\vspace{2em} % ntheorembugumgehung
\begin{satz}[Hauptsatz der mehrdimensionalen Integralrechnung]
$f$ besitzt auf $G$ die Stammfunktion $\varphi$; $\gamma:[a,b]\to\MdR^n$ ein ein stückweise stetig differenzierbarer Weg mit $\Gamma_\gamma\subseteq G$. Dann:
$$ \int_\gamma f(x)\cdot dx = \varphi\left(\gamma(b)\right) - \varphi\left(\gamma(a)\right) $$
Das heißt: $\int_\gamma f(x)\cdot dx$ hängt nur vom Anfangs- und Endpunkt von $\gamma$ ab.

Ist $\gamma$ \emph{geschlossen}, das heißt $\gamma(a) = \gamma(b)$, dann gilt $\int_\gamma f(x)\cdot dx = 0$.
\end{satz}

\begin{beweis}
O.B.d.A.: $\gamma$ ist stetig differenzierbar. $\Phi(t):= \varphi (\gamma(t))$, $t\in[a,b]$. $\Phi$ ist stetig differenzierbar und $\Phi'(t) = \varphi'(\gamma(t))\cdot \gamma'(t) = f(\gamma(t))\cdot\gamma(t)$ Dann: $\int_\gamma f(x)\cdot dx \gleichnach{13.1} \int_a^bf(\gamma(t))\cdot\gamma'(t)dt = \int_a^b\Phi'(t)dt \gleichnach{AI} \Phi(b)-\Phi(a) = \varphi(\gamma(b))-\varphi(\gamma(a))$.
\end{beweis}

\begin{wichtigerhilfssatz}
Es seien $x_0,y_0\in G$. Dann existiert ein stückweise stetig differenzierbarer Weg $\gamma$ mit: $\Gamma_\gamma\subseteq G$ und Anfangspunkt von $\gamma = x_0$ und Endpunkt von $\gamma=y_0$.
\end{wichtigerhilfssatz}

\begin{beweis}
$G$ Gebiet $\folgt \exists z_0,z_1,\ldots,z_m \in G: S[z_0,\ldots,z_m]\subseteq G, z_0=x_0, z_m = y_0$.

$\gamma_j(t) := z_{j-1} + t(z_j - z_{j-1})$, $(t\in[0,1])$, ($j=1,\ldots,n$). Dann:$\Gamma_{\gamma_j} = S[z_{j-1},z_{j}] \folgt \Gamma_{\gamma_1}\cup\ldots\cup\Gamma_{\gamma_m} = S[z_0,\ldots,z_m] \subseteq G$. 13.4 $\folgt \exists \gamma \in \text{AH}(\gamma_1,\ldots,\gamma_m)$ stückweise stetig differenzierbar $\folgt \Gamma_{\gamma} = S[z_0,\ldots,z_m] \subseteq G$.
\end{beweis}

\begin{definition*}
\indexlabel{Weg-!unabhängig}
$\int f(x)\cdot \text{d}x$ heißt \textbf{in G wegunabhängig} (wu) $:\equizu$ für je zwei Punkte $x_0, y_0\in G$ gilt: für jeden stückweise stetig differenzierbaren Weg $\gamma:[a,b]\to\MdR^n$ mit $\Gamma_\gamma\subseteq G$, $\gamma(a)=x_0$ und $\gamma(b)=y_0$ hat das Integral $\ds\int_\gamma f(x)\cdot\text{d}x$ stets denselben Wert. In diesem Fall: $\ds\int_{x_0}^{y_0}f(x)\cdot\text{d}x:=\ds\int_\gamma f(x)\cdot\text{d}x$.
\end{definition*}

\textbf{14.1 lautet dann}: besitzt f auf G die Stammfunktion $\varphi\folgt \ds\int f(x)\cdot\text{d}x$ ist in $G$ wegunabhängig und $\int_{x_0}^{y_0}=\varphi(y_0)-\varphi(x_0)$ (Verallgemeinerung von Analysis 1, 23.5).

\begin{satz}[Wegunabhängigkeit, Existenz von Stammfunktionen]
$f$ besitzt auf $G$ eine Stammfunktion $\equizu\int f(x)\cdot\text{d}x$ ist in G wegunabhängig. \\
In diesem Fall: ist $x_0\in G$ und $\varphi:G\to\MdR$ definiert durch:
\begin{align*}
\varphi(z)=\ds\int_{x_0}^z f(x)\cdot\text{d}x\ (z\in G)\ \tag{$*$}
\end{align*}
Dann ist $\varphi$ eine Stammfunktion von $f$ auf $G$.
\end{satz}

\begin{beweis}
"`$\folgt$"': 14.1\quad "`$\impliedby$"': Sei $x_0\in G$ und $\varphi$ wie in $(*)$. Zu zeigen: $\varphi$ ist auf $G$ differenzierbar und $\varphi'=f$ auf G. Sei $z_0\in G, h\in\MdR^n,h\ne 0$ und $\|h\|$ so klein, dass $z_0+th\in G\ \forall t\in[0,1].\ \gamma(t):=z_0+th\ (t\in[0,1]), \Gamma_\gamma=s[z_0, z_0+h]\subseteq G$. $\rho(h):=\frac{1}{\|h\|}(\varphi(z_0+h)-\varphi(z_0)-f(z_0)\cdot h)$. Zu zeigen: $\rho(h)\to 0\ (h\to 0)$. 14.2 $\folgt$ es existieren stückweise stetig differenzierbare Wege $\gamma_1, \gamma_2$ mit: $\Gamma_{\gamma_1},\Gamma_{\gamma_2}\subseteq G$. Anfangspunkt von $\gamma_1=x_0=$Anfangspunkt von $\gamma_2$. Endpunkt von $\gamma_1=z_0$, Endpunkt von $\gamma_2=z_0+h$. Sei $\gamma_3\in \text{AH}(\gamma_1,\gamma)$ stückweise stetig differenzierbar (13.4!). Dann:
$$\underbrace{\ds\int_{\gamma_3}f(x)\cdot\text{d}x}_{=\varphi(z_0+h)}=\underbrace{\ds\int_{\gamma_1}f(x)\cdot\text{d}x}_{=\varphi(z_0)}+\ds\int_{\gamma}f(x)\cdot\text{d}x$$
$\ds\int f(x)\cdot\text{d}x$ ist wegunabhängig in $G\folgt$\\
$$\ds\int_{\gamma_3}f(x)\cdot\text{d}x=\ds\int_{\gamma_2}f(x)\cdot\text{d}x=\varphi(z_0+h)\folgt\varphi(z_0+h)-\varphi(z_0)=\ds\int_{\gamma}f(x)\cdot\text{d}x$$
Es ist:
\begin{eqnarray*}
&&\ds\int_{\gamma}f(z_0)\cdot\text{d}x=\ds\int_0^1 f(z_0)\cdot\underbrace{\gamma'(t)}_{=h}\text{d}t=f(z_0)\cdot h\\
&&\folgt \rho(h)=\frac{1}{\|h\|}\ds\int_{\gamma}(f(x)-f(z_0))\text{d}x\\
&&\folgt |\rho(h)|=\frac{1}{\|h\|}\left|\ds\int_{\gamma}f(x)-f(z_0)\text{d}x\right|\\
&&\le\frac{1}{\|h\|}\underbrace{L(\gamma)}_{=\|h\|}\underbrace{\max\{\|f(x)-f(z_0)\|: x\in\Gamma_\gamma\}}_{=\|f(x_n)-f(z_0)\|}
\end{eqnarray*}
wobei $x_n\in\Gamma_\gamma=S[z_0,z_0+h]\folgt |\rho(h)|\le\|f(x_n)-f(z_0)\|$. Für $h\to 0: x_n\to z_0\folgtnach{f stetig}\|f(x_n)-f(z_0)\|\to 0\folgt\rho(h)\to 0$.
\end{beweis}

\begin{satz}[Integrabilitätsbedingungen]
Sei $f=(f_1,\ldots, f_n)\in C^1(G,\MdR^n)$. Besitzt $f$ auf $G$ die Stammfunktion $\varphi\folgt$

$$\frac{\partial f_j}{\partial x_k}=\frac{\partial f_k}{\partial x_j}\text{ auf }G\ (j,k=1,\ldots,n)$$
(\begriff{Integrabilitätsbedingungen} (IB)). Warnung: Die Umkehrung von 14.4 gilt im Allgemeinen \textbf{nicht} ($\to$ Übungen!).
\end{satz}

\begin{beweis}
Sei $\varphi$ eine Stammfunktion von $f$ auf $G\folgt\varphi$ ist differenzierbar auf $G$ und $\varphi_{x_j}=f_j$ auf $G\ (j=1,\ldots,n)$. $f\in C^1(G,\MdR^n)\folgt\varphi\in C^2(G,\MdR)$
$$\folgt \frac{\partial f_j}{\partial x_k}=\varphi_{x_jx_k}\gleichnach{4.7}\varphi_{x_kx_j}=\frac{\partial f_k}{\partial x_j}\text{ auf G.}$$ $ $
\end{beweis}

\begin{definition*}
\index{Sternförmigkeit}
Sei $\emptyset\ne M\subseteq\MdR^n$. $M$ heißt \textbf{sternförmig} $:\equizu\exists x_0\in M: S[x_0,x]\subseteq M\ \forall x\in M$.\\
\textbf{Beachte:}
\begin{enumerate}
\item Ist $M$ konvex$\folgt M$ ist sternförmig
\item Ist $M$ offen und sternförmig$\folgt M$ ist ein Gebiet
\end{enumerate}
\end{definition*}

\begin{satz}[Kriterium zur Existenz von Stammfunktionen]
Sei $G$ sternförmig und $f\in C^1(G,\MdR^n)$. Dann: $f$ besitzt auf $G$ eine Stammfunktion $:\equizu f$ erfüllt auf $G$ die Integrabilitätsbedingungen
\end{satz}

\begin{beweis}
"`$\folgt$"': 14.1\quad "`$\impliedby$"': $G$ sternförmig $\folgt\exists x_0\in G:S[x_0,x]\subseteq G\ \forall x\in G$. OBdA: $x_0=0$.

Für $x=(x_1,\ldots,x_n)\in G$ sei $\gamma_x(t)=tx, t\in [0,1]$.
\begin{eqnarray*}
\varphi(x)&:=&\int_{\gamma_x}f(z)\cdot\text{d}z\ (x\in G)\\
&=&\ds\int_0^1 f(tx)\cdot x\text{d}t\\
&=&\ds\int_0^1(f_1(tx)\cdot x_1+f_2(tx)\cdot x_2+\ldots+f_n(tx)\cdot x_n)\text{d}t\\
\end{eqnarray*}
Zu zeigen: $\varphi$ ist auf $G$ partiell differenzierbar nach $x_j$ und $\varphi_{x_j}=f_j\ (j=1,\ldots,n)$.
OBdA: $j=1$. Später (in 21.3) zeigen wir: $\varphi$ ist partiell differenzierbar nach $x_1$ und:
$$\varphi_{x_1}(x)=\ds\int_0^1\frac{\partial}{\partial x_1}(f_1(tx)x_1+\ldots+f_n(tx)\cdot x_n)\text{d}t$$

Für $k=1,\ldots,n:\ g_k(x)=f_k(tx)\cdot x_k$.\\
$k=1:\ g_1(x)=f_1(tx)x_1\folgt\frac{\partial g_1}{\partial x_1}(x)=f_1(tx)+t\frac{\partial f_1}{\partial x_1}(tx)x_1$\\
$k\ge 2:\ g_k(x)=f_k(tx)x_k\folgt\frac{\partial g_k}{\partial x_1}(x)=t\frac{\partial f_k}{\partial x_1}(tx)x_k\folgt$

\begin{eqnarray*}
\varphi_{x_1}(x)&=&\int_0^1(f_1(tx)+t(\frac{\partial f_1}{\partial x_1}(tx)x_1+\ldots+\frac{\partial f_n}{\partial x_1}(tx)x_n))\text{d}t\\
&\gleichnach{IB}&\int_0^1(f_1(tx)+t(\frac{\partial f_1}{\partial x_1}(tx)x_1+\frac{\partial f_1}{\partial x_2}(tx)x_2+\ldots+\frac{\partial f_1}{\partial x_n}(tx)x_n))\text{d}t\\
&=&\ds\int_0^1(f_1(tx)+tf_1'(tx)\cdot x)\text{d}t
\end{eqnarray*}

Sei $x\in G$ (fest), $h(t):=t\cdot f_1(tx)\ (t\in [0,1])$. $h$ ist stetig differenzierbar und $h'(t)=f_1(tx)+tf_1'(tx)\cdot x\folgt \varphi_{x_1}(x)=\ds\int_0^1 h'(t)\text{d}t\gleichnach{A1}h(1)-h(0)=f_1(x)$.
\end{beweis}

\chapter{Vorgriff auf Analysis III}

In Analysis III werden wir für gewisse Mengen $A\subseteq\MdR^n$ und gewisse Funktionen
$f:A\to\MdR$ folgendes Integral definieren:
\[\int_A f(x)\text{d}x=\int_A f(x_1,\ldots,x_n)\text{ d}(x_1,\ldots,x_n)\]
In diesem Paragraphen geben wir "`Kochrezepte"' an, wie man solche Integrale
für spezielle Mengen $A \subseteq \MdR^2$ (bzw. $A \subseteq \MdR^3$)
und stetige Funktionen $f:A \to \MdR$ berechnen kann.

\renewcommand{\theenumi}{\Roman{enumi}}
\renewcommand{\labelenumi}{\theenumi}

\begin{enumerate}
\item \textbf{Der Fall $n=2$}:\\
\begin{definition*}
\indexlabel{Normalbereich}
Es sei $[a,b]\subset\MdR$, $h_1,h_2\in C[a,b]$ und $h_1\le h_2$ auf $[a,b]$.
\[A:=\Set{(x,y)\in\MdR^2 | x\in[a,b],h_1(x)\le y\le h_2(x)}\]
\[\left(A:=\Set{(x,y)\in\MdR^2 | y\in[a,b],h_1(y)\le x\le h_2(y)}\right)\]
heißt \textbf{Normalbereich} bezüglich der $x$-Achse ($y$-Achse).\\
\textbf{Übung:} $A$ ist kompakt.
\end{definition*}

\begin{satz}[Integral über Normalbereiche im $\MdR^2$]
Sei $A$ wie oben und $f:A\to\MdR$, dann gilt:
\[\int_A f(x,y)\text{d}(x,y)=\int_a^b\left(\int_{h_1(x)}^{h_2(x)} f(x,y)\text{d}y\right)\text{ d}x\]
\[\left(\int_A f(x,y)\text{d}(x,y)=\int_a^b\left(\int_{h_1(y)}^{h_2(y)} f(x,y)\text{ d}x\right)\text{ d}y\right)\]
\end{satz}

\textbf{Achtung:} $\int_A f(x)\text{d}x$ nicht mit dem Wegintegral $\int_\gamma f(x)\cdot\text{d}x$
verwechseln!

\begin{definition*}
\indexlabel{Flächeninhalt}
Sei $A$ ein Normalbereich bzgl. der $x$- oder $y$-Achse, so heißt:
\[|A|:=\int_A 1\text{ d}(x,y)\]
der \textbf{Flächeninhalt} von $A$.
\end{definition*}

\begin{beispiele}
\item[(1)] Sei $A$ ein Normalbereich bzgl. der x-Achse und seien $h_1,h_2$ wie oben. Dann gilt:
\begin{align*}
|A|&=\int_A 1\text{ d}(x,y)\\
&\stackrel{15.1}{=}\int_a^b\left(\int_{h_1(x)}^{h_2(x)} 1\text{ d}y\right)\text{ d}x\\
&=\int_a^b(h_2(x)-h_1(x))\text{ d}x
\end{align*}
Ist z.B. $h_1=0$, so folgt:
\[|A|=\int_a^b h_2(x)\text{ d}x\]
\item[(2)] Sei $A=[a,b]\times[c,d]$, dann ist $A$ Normalbereich bezüglich der $x$-
\textbf{und} der $y$-Achse. Sei $f:A\to\MdR$ stetig. Es folgt aus 15.1.:
\begin{align*}
\int_A f(x,y)\text{d}(x,y) &= \int_a^b\left(\int_c^d f(x,y)\text{ d}y\right)\text{ d}x\\
&= \int_c^d\left(\int_a^b f(x,y)\text{ d}x\right)\text{ d}y
\end{align*}
\item[(3)] Sei $r>0$ und $A:=\Set{(x,y)\in\MdR^2 | x^2+y^2\le r^2}$. Dann ist
$A$ ein Normalbereich der $x$-Achse mit $h_1(x):=\sqrt{r^2-x^2}$ und $h_2(x):=-\sqrt{r^2-x^2}$
(mit $x\in[-r,r]$), und es gilt:
\begin{align*}
|A|&=\int_{-r}^r h_2(x)-h_1(x)\text{ d}x\\
&=\int_{-r}^r 2\sqrt{r^2-x^2}\text{ d}x\\
&=\pi r^2
\end{align*}
\item[(4)] Sei $A:=\Set{(x,y)\in\MdR^2 | x\in[0,1],x\le y\le \sqrt x}$ und $f(x,y)=xy$.
Dann gilt für $h_1(x)=x$ und $h_2(x)=\sqrt x$:
\begin{align*}
\int_A xy\text{ d}(x,y) &= \int_0^1\left(\int_x^{\sqrt x} xy\text{ d}y\right)\text{ d}x\\
&= \int_0^1\left(\left[\frac12 xy^2\right]_x^{\sqrt x}\right)\text{ d}x\\
&= \int_0^1 \frac12x^2 -\frac12x^3\text{ d}x\\
&= \frac12\left[\frac13 x^3-\frac14 x^4\right]_0^1 = \frac1{24}
\end{align*}
Da $A=\Set{(x,y)\in\MdR^2 | y\in[0,1],y^2\le x\le y}$ außerdem Normalbereich bzgl. der
$y$-Achse ist, gilt:
\begin{align*}
\int_A f(x,y)\text{ d}y &= \int_0^1\left(\int_{y^2}^y xy \text{ d}x\right)\text{ d}y\\
&=\int_0^1 \left[ \frac12 x^2y\right]_{y^2}^y \text{ d}y\\
&=\int_0^1 \frac12y^3-\frac12y^5 \text{ d}y\\
&= \frac12 \left[ \frac14y^4-\frac16 y^5\right]_0^1 = \frac1{24}
\end{align*}
\end{beispiele}
\item \textbf{Der Fall $n=3$:}\\
\indexlabel{Normalbereich}
\begin{definition}
Sei $A\subseteq\MdR^2$ ein Normalbereich bzgl. der $x$- oder der $y$-Achse,
es seien $g_1,g_2:A\to\MdR$ stetig und $g_1\le g_2$ auf $A$.
\[B:=\Set{(x,y,z)\in\MdR^3 | (x,y)\in A, g_1(x,y)\le z\le g_2(x,y)}\]
heißt ein \textbf{Normalbereich} bezüglich der $x$-$y$-Ebene.
Normalbereiche bzgl der $x$-$z$- und $y$-$z$-Ebene werden analog definiert.
\end{definition}

\begin{satz}[Integral über Normalbereiche im $\MdR^3$]
Sei $B, g_1, g_2$ wie oben und $f:B\to\MdR$ stetig, dann gilt:
\[\int_B f(x,y,z) \text{ d}(x,y,z) = \int_A\left(\int_{g_1(x,y)}^{g_2(x,y)} f(x,y,z) \text{ d}z\right)\text{ d}(x,y)\]
\end{satz}

\begin{definition}
\indexlabel{Volumen}
$B$ sei wie in 15.2.
\begin{align*}
|B|&:=\int_B 1\text{ d}(x,y,z)\\
&\left( = \int_A g_2(x,y)-g_1(x,y)\text{ d}(x,y)\right)
\end{align*}
heißt \textbf{Volumen} von $B$.
\end{definition}

\begin{beispiele}
\item[(1)] Sei $B:=\overbrace{[a,b]\times[c,d]}^{:= A}\times[\alpha,\beta]$, dann gilt:
\begin{align*}
\int_B f(x,y,z) \text{ d}(x,y,z) &= \int_A\left(\int_\alpha^\beta f(x,y,z) \text{ d}z\right)\text{ d}(x,y)\\
&=\int_a^b\left(\int_c^d\left(\int_\alpha^\beta f(x,y,z) \text{ d}z\right)\text{ d}y\right)\text{ d}x
\end{align*}
Dabei darf die Integrationsreihenfolge beliebig vertauscht werden.
\item[(2)] Sei $B:=\Set{(x,y,z)\in\MdR^3 | x^2+y^2\le 1, 0\le z\le h}$ für ein
$h>0$. Dann setze $A:=\Set{(x,y)\in\MdR^2 | x^2+y^2\le 1}, g_1=0, g_2=h$. Es gilt:
\begin{align*}
|B|&= \int_A h\text{ d}(x,y)\\
&= h \int_A 1 \text{ d}(x,y)\\
&= h\cdot |A| = h\pi
\end{align*}
\end{beispiele}

\begin{satz}[Eigenschaften von Integralen über Normalbereiche]
Sei $B\subseteq\MdR^2$ oder $B\subseteq\MdR^3$ und $f,g:B\to\MdR$ stetig.
Je nach Definition von $B$ sei $X=(x,y)$ oder $X=(x,y,z)$.
\begin{enumerate}
\item[(1)] Für alle $\alpha,\beta\in\MdR$ gilt:
\[\int_B \alpha f(X)+\beta g(X)\text{ d}X=\alpha\int_B f(X)\text{ d}X+\beta\int_B g(X)\text{ d}X\]
\item[(2)] Es gilt die bekannte Dreiecksungleichung:
\[\left|\int_B f(X)\text{ d}X\right|\le \int_B |f(X)| \text{ d}X\le |B|\cdot \max\{|f(X)|: X\in B\}\]
\item[(3)] Ist $f\le g$ auf $B$, so gilt:
\[\int_B f(X)\text{ d}X \le \int_B g(X)\text{ d}X\]
\end{enumerate}
\end{satz}
\end{enumerate}

\chapter{Folgen, Reihen und Potenzreihen in $\MdC$}
\renewcommand{\labelenumi}{(\arabic{enumi})}

\index{komplex!Betrag}\index{Betrag!komplexer}
$\MdC$ und $\MdR^2$ sind Vektorräume \textbf{über $\MdR$} der Dimension zwei.
Sie unterscheiden sich als Vektorräume über $\mathbb{R}$ nur dadurch, dass ihre Elemente
mit:
\[z=x+\text{i} y\in \MdC \quad \text { bzw. }\quad (x,y)\in\MdR^2 \quad(x,y\in\MdR)\]
bezeichnet werden. Mit dem \textbf{komplexen Betrag} $|z|:=\sqrt{x^2+y^2}$ gilt:
\[|z|=\|(x,y)\|\]
Man sieht, dass alle aus der Addition, der Skalarmultiplikation und der Norm entwickelten Begriffe
und Sätze aus §1 und §2 auch in $\MdC$ gelten.\\

\begin{beispiel}[Konvergente Folgen]
Sei $(z_n)$ eine Folge in $\MdC$ und $z_0\in\MdC$. $(z_n)$ konvergiert genau dann gegen
$z_0$, wenn gilt:
\begin{align*}
&|z_n-z_0|\stackrel{n\to\infty}{\to}0\\
\stackrel{2.1}{\iff} &\Re(z_n)\stackrel{n\to\infty}{\to}\Re(z_0)\wedge
\Im(z_n)\stackrel{n\to\infty}{\to}\Im(z_0)
\end{align*}
Außerdem ist $(z_n)$ genau dann eine Cauchyfolge, wenn gilt:
\[\forall\ep>0\exists n_0\in\MdN\forall n,m\ge n_0: |z_n-z_m|<\ep\]
Also nach Cauchykriterium genau dann, wenn $(z_n)$ konvergent ist.
\end{beispiel}

\begin{satz}[Produkte und Quotienten von Folgen]
Seien $(z_n),(w_n)$ Folgen in $\MdC$ mit $z_n\stackrel{n\to\infty}{\to}z_0,
w_n\stackrel{n\to\infty}{\to}w_0$.
\begin{enumerate}
\item Es gilt:
\[z_nw_n\stackrel{n\to\infty}{\to}z_0w_0\]
\item Ist $z_0\ne 0$, so existiert ein $m\in\MdN:\forall n\ge m:z_n\ne 0$ und:
\[\frac1{z_n}\stackrel{n\to\infty}{\to}\frac1{z_0}\]
\end{enumerate}
\end{satz}

\begin{beweis}
Wie in Ana I.
\end{beweis}

\begin{definition}
\index{unendliche Reihe}\index{Reihe!unendliche}
\index{Konvergenz}\index{Divergenz}
Sei $(z_n)$ eine Folge in $\MdC$, $s_n:=z_1+\cdots+z_n (n\in\MdN)$. $(s_n)$ heißt
\textbf{unendliche Reihe} und wird mit $\sum_{n=1}^{\infty} z_n$ bezeichnet.\\
$\sum_{n=1}^{\infty} z_n$ heißt genau dann \textbf{konvergent} (\textbf{divergent}),
wenn $(s_n)$ konvergent (bzw. divergent) ist. Im Konvergenzfall gilt:
\[\sum_{n=1}^{\infty} z_n:=\limsup_{n\to\infty} s_n\]
\end{definition}

Die Definitionen und Sätze der Paragraphen 11, 12, 13 aus Ana I gelten wörtlich
auch in $\MdC$, bis auf diejenigen Definitionen und Sätze, in denen die Anordnung
auf $\MdR$ eine Rolle spielt (z.B. das Leibniz- und das Monotoniekriterium).
\\
\begin{beispiele}
\index{geometrische Reihe}\index{Reihe!geometrische}
\index{Exponentialfunktion}\index{komplex!Exponentialfunktion}
\index{Cosinus}\index{Kosinus}\index{komplex!Kosinus}
\index{Sinus}\index{komplex!Sinus}
\item Sei $z\in\MdC$. $\sum_{n=0}^{\infty} z^n$ heißt \textbf{geometrische Reihe}.\\
\textbf{Fall 1:} Ist $|z|< 1$, dann gilt:
\begin{align*}
&\sum_{n=0}^{\infty} |z|^n \text{ konvergiert}\\
\implies &\sum_{n=0}^{\infty} z^n\text{ konvergiert absolut}\\
\implies &\sum_{n=0}^{\infty} z^n\text{ konvergiert}
\end{align*}
\textbf{Fall 2:} Ist $|z|\ge 1$, dann gilt:
\begin{align*}
&|z|^n=|z^n|\stackrel{n\to\infty}{\not\to} 0\\
\implies &z^n\stackrel{n\to\infty}{\not\to} 0\\
\implies &\sum_{n=0}^{\infty} z^n \text{ divergiert}
\end{align*}
Ist $|z|<1$, so zeigt man wie in $\MdR$:
\[\sum_{n=0}^{\infty} z^n= \frac1{1-z}\]
\item Betrachte $\sum_{n=0}^\infty \frac{z^n}{n!}$. Für alle $z\in\MdC$ gilt:
\begin{align*}
&\sum_{n=0}^\infty \frac{|z|^n}{n!} \text{ konvergiert}\\
\implies &\sum_{n=0}^\infty \frac{z^n}{n!} \text{ konvergiert absolut}
\end{align*}
Für alle $z\in\MdC$ definiere die (komplexe) \textbf{Exponentialfunktion} wie folgt:
\[e^z:=\sum_{n=0}^\infty\frac{z^n}{n!}\]
\item Wie in Beispiel (2) sieht man, dass $\sum_{n=0}^\infty (-1)^n \frac{z^{2n}}{(2n)!}$ und
$\sum_{n=0}^\infty (-1)^n \frac{z^{2n+1}}{(2n+1)!}$ für alle $z\in\MdC$ absolut konvergieren.\\
Dadurch lassen sich auch \textbf{Cosinus} und \textbf{Sinus} auf ganz $\MdC$ definieren:
\[\cos{z}:=\sum_{n=0}^\infty (-1)^n\frac{z^{2n}}{(2n)!}\]
\[\sin{z}:=\sum_{n=0}^\infty (-1)^n\frac{z^{2n+1}}{(2n+1)!}\]
\end{beispiele}

\begin{satz}[Eigenschaften von Exponentialfunktion, Cosinus und Sinus]
Seien $z,w\in\MdC, z=x+iy$ mit $x,y\in\MdR$. Es gilt:
\begin{enumerate}
\item $e^{z+w}=e^z e^w$
\item $e^{iy}=\cos y+i\sin y$, insbesondere ist: $|e^{iy}|=1$
\item $e^z=e^x e^{iy}=e^x(\cos y+i\sin y)$
\item $ \cos (z)=\frac{e^{iz}+e^{-iz}}{2}, \sin (z)=\frac{e^{iz}-e^{-iz}}{2i}$\\
Insbesondere ist für alle $t\in\MdR: \cos(it)=\frac{e^{-t}+e^t}{2} \xrightarrow{t \rightarrow \infty} \infty,
\sin(it)=\frac{e^{-t}-e^t}{2i} \xrightarrow{t \rightarrow \infty} - \infty$\\
Also sind Cosinus und Sinus auf $\MdC$ \textbf{nicht} beschränkt.
\item $\forall k\in\MdZ:e^{z+2\pi i k}=e^z$
\item $e^z=1 \iff \exists k\in\MdZ:z=2k\pi i$
\item $e^{i\pi}+1=0$
\end{enumerate}
\end{satz}

\begin{beweise}
\item Wie in Ana I.
\item Nachrechnen!
\item Folgt aus (1) und (2).
\item Nachrechnen!
\item Es gilt:
\begin{align*}
e^{z+2k\pi i} &\stackrel{(1)}{=}e^ze^{2k\pi i}\\
&\stackrel{(2)}{=}e^z(\underbrace{\cos(2k\pi)}_{=1}+i\underbrace{\sin(2k\pi)}_{=0})\\
&= e^z
\end{align*}
$\Rightarrow e^z$ ist auf $\mathbb{C}$ nicht injektiv!
\item
Die Äquivalenz folgt aus Implikation in beiden Richtungen:
\begin{enumerate}
\item["`$\impliedby$"'] Folgt aus (5) mit $z=0$.
\item["`$\implies$"'] Sei $z=x+iy$ mit $x,y\in\MdR$. Es gilt:
\[1=e^z=e^x(\cos(y)+i\sin(y))=e^x\cos(y)+ie^x\sin(y)\]
Daraus folgt:
\[\sin(y)=0\implies \exists j\in\MdZ:y=j\pi\]
Und damit:
\begin{align*}
&1=e^x\cos(j\pi)=e^x(-1)^j\\
\implies &x=0\wedge\exists k\in\MdN: j=2k
\end{align*}
Also ist $z=i2k\pi$.
\end{enumerate}
\item Es gilt:
\[e^{i\pi}\stackrel{(2)}{=}\cos(\pi)+i\sin(\pi)=-1\]
\end{beweise}

\begin{beispiel}
Im Folgenden wollen wir alle $z\in\MdC$ bestimmen, für die $\sin(z)=0$ ist. Es gilt:
\begin{align*}
\sin(z)=0 &\stackrel{16.2(4)}{\iff}e^{iz}=e^{-iz}\\
&\stackrel{16.2(1)}{\iff}e^{2iz}=e^{-iz}e^{iz}=e^0=1\\
&\stackrel{16.2(6)}{\iff}\exists k\in\MdZ: 2iz=i2k\pi\\
&\iff z=k\pi
\end{align*}
Der Sinus hat also nur reelle Nullstellen.
\end{beispiel}

\begin{definition}
\index{Potenzreihe}
\index{Konvergenzradius}
Sei $(a_n)$ ein Folge in $\MdC$ und $z_0\in\MdC$. $\sum_{n=0}^\infty a_n(z-z_0)^n$
heißt eine \textbf{Potenzreihe} (PR). Sei nun:
\[\rho:=\limsup \sqrt[n]{|a_n|}\]
Dabei ist $\rho=\infty$, falls $\sqrt[n]{|a_n|}$ unbeschränkt ist. Dann heißt
\[r:=
\begin{cases}
0&\text{, falls }\rho=\infty\\
\infty&\text{, falls }\rho=0\\
\frac1\rho&\text{, falls }0<\rho<\infty
\end{cases}\]
der \textbf{Konvergenzradius} (KR) der PR.
\end{definition}

\begin{satz}[Konvergenz von Potenzreihen]
$\sum_{n=0}^\infty a_n(z-z_0)^n$ und $r$ seien wie oben.
\begin{enumerate}
\item Ist $r=0$, so konvergiert die PR \textbf{nur} für $z=z_0$.
\item Ist $r=\infty$, so konvergiert die PR absolut für alle $z\in\MdC$.
\item Sei $0<r<\infty$. Es gilt:
\begin{enumerate}
\item Ist $z\in\MdC$ und $|z-z_0|<r$, so konvergiert die PR absolut in $z$.
\item Ist $z\in\MdC$ und $|z-z_0|>r$, so divergiert die PR in $z$.
\item Ist $z\in\MdC$ und $|z-z_0|=r$, so ist keine allgemeine Aussage möglich.
\end{enumerate}
\end{enumerate}
\end{satz}

\begin{beweis}
Wie in Ana I.
\end{beweis}

\begin{beispiele}
\item Die PR $\sum_{n=0}^\infty z^n$ hat den KR $r=1$ und es gilt:
\[\sum_{n=0}^\infty z^n\text{ konvergiert }\iff |z|<1\]
\item Die PR $\sum_{n=0}^\infty \frac{z^n}{n^2}$ hat den KR $r=1$. Für $|z|=1$ gilt:
\[\sum_{n=0}^\infty \frac{|z|^n}{n^2}=\sum_{n=0}^\infty \frac1{n^2}\]
Also konvergiert $\sum_{n=0}^\infty \frac{z^n}{n^2}$ absolut. Insgesamt gilt also:
\[\sum_{n=0}^\infty \frac{z^n}{n^2}\text{ konvergiert }\iff |z|\le 1\]
\item Die PR $\sum_{n=0}^\infty \frac{z^n}{n}$ hat KR $r=1$, divergiert in $z=1$ und
konvergiert in $z=-1$.
\item Die PRen
\begin{align*}
&\sum_{n=0}^\infty \frac{z^n}{n!}&&\sum_{n=0}^\infty(-1)^n\frac{z^{2n}}{(2n)!}
&&\sum_{n=0}^\infty (-1)^n\frac{z^{2n+1}}{(2n+1)!}
\end{align*}
haben jeweils KR $r=\infty$ (siehe 16.3).
\end{beispiele}

\chapter{Normierte Räume, Banachräume, Fixpunktsatz}
In diesem Paragraphen sei $\mathbb{K}$ stets gleich $\mathbb{R}$ oder $\mathbb{C}$
und sei $X$ ein Vektorraum (VR) über $\mathbb{K}$.

\begin{definition}
\index{Norm}
\index{normierter Raum}\index{Raum!normierter}
Eine Abbildung $\|\cdot\|:X\to\mathbb{R}$ heißt genau dann eine \textbf{Norm} auf $X$,
wenn folgendes erfüllt ist:
\begin{enumerate}
\item $\forall x\in X: \|x\|\ge0 \text{ und } \|x\|=0\iff x=0$
\item $\forall x\in X,\alpha\in\mathbb{K}:\|\alpha x\|=|\alpha|\cdot\|x\|$
\item $\forall x,y\in X: \|x+y\|\le \|x\|+\|y\|$
\end{enumerate}
In diesem Fall heißt $(X,\|\cdot\|)$ ein \textbf{normierter Raum} (NR).
\end{definition}

\begin{beispiele}
\index{euklidische Norm}\index{Norm!euklidische}
\item Sei $n\in\mathbb{N},X=\mathbb{K}^n$ und für $x=(x_1,\ldots,x_n)\in X$
die \textbf{euklidische Norm} gegeben:
\[\|x\|_2:=\sqrt{\sum_{j=1}^n |x_j|^2}\]
Dann ist $(\mathbb{K}^n,\|\cdot\|_2)$ ein normierter Raum (vgl. §1).
\item Sei $X=C[a,b]$ und für $f\in X$ seien die folgenden Normen gegeben:
\begin{align*}
&\|f\|_1:=\int_a^b|f(x)|\text{ d}x\\
&\|f\|_2:=\sqrt{\int_a^b|f(x)|^2\text{ d}x}\\
&\|f\|_\infty:=\max\{|f(x)|:x\in[a,b]\}
\end{align*}
Leichte Übung: $(X,\|\cdot\|_1),(X,\|\cdot\|_2)$ und $(X,\|\cdot\|_\infty)$
sind NRe.
\item Sei $K\subseteq \mathbb{R}^n$ kompakt, $X:=C(K,\mathbb{R}^m)$ und sei für
$f\in X$ die Norm
\[\|f\|_\infty:=\max\{\|f(x)\|:x\in K\}\]
Leichte Übung: $(X,\|\cdot\|_\infty)$ ist ein NR.
\end{beispiele}

Für den Rest dieses Paragraphen sei $X$ stets ein NR mit Norm $\|\cdot\|$.

\begin{bemerkung}
Wie in §1 zeigt man die umgekehrte Dreiecksungleichung:
\[\forall x,y\in X:|\|x\|-\|y\||\le\|x-y\|\]
\end{bemerkung}

\begin{definition}
\index{Konvergenz}
\index{Divergenz}
\index{Grenzwert}
\index{Limes}
\index{Cauchy-!Folge}
\index{Offenheit}
\index{Abgeschlossenheit}
\begin{enumerate}
\item Sei $(x_n)$ eine Folge in $X$. $(x_n)$ heißt genau dann \textbf{konvergent},
wenn ein $x_0\in X$ existiert für das gilt:
\[\|x_n-x_0\|\stackrel{n\to\infty}\to 0\]
In diesem Fall ist $x_0$ eindeutig bestimmt und man schreibt $x_n\stackrel{n\to\infty}\to x_0$
oder $\lim_{n\to\infty} x_n=x_0$. $x_0$ heißt \textbf{Grenzwert} oder \textbf{Limes}
von $(x_n)$.
\item $(x_n)$ heißt genau dann \textbf{divergent}, wenn $(x_n)$ nicht konvergent ist.
\item $(x_n)$ heißt genau dann eine \textbf{Cauchyfolge} (CF), wenn gilt:
\[\forall\ep>0 \exists n_0=n_0(\ep)\in\MdN:\|x_n-x_m\|<\ep\quad \forall n,m\ge n_0\]
\item Sei $x_0\in X$ und $\delta>0$. Definiere:
\[U_\delta(x):=\Set{x\in X | \|x-x_0\|<\delta}\]
\item Sei $A\subseteq X$. $A$ heißt \textbf{offen}, genau dann wenn gilt:
\[\forall x\in A\exists \delta=\delta(x)>0: U_\delta(x)\subseteq A\]
A heißt \textbf{abgeschlossen}, genau dann wenn $X\setminus A$ offen ist.
\end{enumerate}
\end{definition}

\begin{satz}[Eigenschaften von Folgen in normierten Räumen]
\index{Beschränktheit}
\index{gleichmäßige Konvergenz}\index{Konvergenz!gleichmäßige}
Seien $(x_n),(y_n)$ Folgen in $X$, $(\alpha_n)$ Folge in $\mathbb{K}$, $x,y\in X$ und
$A\subseteq X$.
\begin{enumerate}
\item Gilt $x_n\to x,y_n\to y$ und $\alpha_n\to\alpha\in\mathbb{K}$, so folgt:
\begin{align*}
&x_n+y_n\to x+y& &\alpha_n x_n\to\alpha x& &\|x_n\|\to\|x\|
\end{align*}
D.h. die Addition und Skalarmultiplikation sind stetig.
\item Ist $(x_n)$ konvergent, so ist $(x_n)$ \textbf{beschränkt}, d.h.:
\[\exists c\ge 0:\forall n\in\MdN:\|x_n\|\le c\]
und $(x_n)$ ist eine CF.
\item Genau dann wenn $A$ abgeschlossen ist, gilt für jede konvergente Folge $(x_n)$ in $A$:
\[\lim_{n\to\infty}x_n\in A\]
\item Sei $(X,\|\cdot\|_\infty)$ wie in obigem Beispiel (3). Dann gilt für
$(f_n)$ in $X$ und $f\in X$, dass $(f_n)$ genau dann auf $K$ \textbf{gleichmäßig} gegen $f$
\textbf{konvergiert}, wenn gilt:
\begin{align*}
&\|f_n-f\|_\infty \stackrel{n\to\infty}{\to}0\\
:\iff &\forall\ep>0\exists n_0\in\MdN:\|f_n(x)-f(x)\|<\ep\quad \forall n\ge n_0\forall x\in K
\end{align*}
\end{enumerate}
\end{satz}

\begin{beweise}
\item Wie im $\MdR^n$.
\item Wie im $\MdR^n$.
\item Wie im $\MdR^n$.
\item \textbf{\color{red}In der großen Übung.}
\end{beweise}

\begin{beispiel}
Sei $X=C[-1,1]$ mit $\|f\|_2:=\sqrt{\int_{-1}^1 |f(x)|^2\text{ d}x}$. Definiere die
Folge $(f_n)$ wie folgt:
\[\forall n\in\MdN: f_n(x)=
\begin{cases}
-1&,-1\le x\le -\frac1n\\
nx&,-\frac1n\le x\le \frac1n\\
1&,\frac1n\le x\le 1
\end{cases}\]
Dann ist klar, dass $f_n\in X$ für alle $n\in\MdN$. In den \textbf{\color{red}großen Übungen} wird gezeigt:
\begin{enumerate}
\item $(f_n)$ ist eine CF in $X$.
\item Es existiert \textbf{kein} $f\in X$ mit $\|f_n-f\|_2\to 0$
\end{enumerate}
\end{beispiel}

\begin{definition}
\index{Banachraum}
\index{vollständiger Raum}\index{Raum!vollständiger}
$X$ heißt ein \textbf{Banachraum} oder \textbf{vollständig}, genau dann wenn jede
CF in $X$ einen Grenzwert in $X$ hat.
\end{definition}

\begin{beispiele}
\item Sei $X=\mathbb{K}^n, \|x\|_2=\sqrt{\sum_{j=1}^n |x_j|^2}$. Dann folgt aus §2,
dass $(X,\|\cdot\|_2)$ ein BR ist.
\item Sei $X=C[-1,1], \|f\|_2=\sqrt{\int_{-1}^1 |f(x)|^2 \text{ d}x}$. Dann ist
$(X,\|\cdot\|_2)$ \textbf{kein} BR.
\item Sei $(X,\|\cdot\|_\infty)$ wie in 17.1(4). In den \textbf{\color{red}großen Übungen} wird gezeigt,
dass $(X,\|\cdot\|_\infty)$ ein BR ist.
\end{beispiele}

\begin{satz}[Banachscher Fixpunktsatz]
    \index{Kontraktion}
    \index{Folge der sukzessiven Approximationen}\index{sukzessive Approximationen!Folge der}
    Sei $(X,\|\cdot\|)$ ein BR, $\emptyset\ne A\subseteq X$ sei abgeschlossen und es sei
    $F:A\to X$ eine Abbildung mit:
    \renewcommand{\labelenumi}{(\roman{enumi})}
    \begin{enumerate}
    \item $F(A)\subseteq A$
    \item $F$ ist eine \textbf{Kontraktion}, d.h.:
    \[\exists L\in[0,1):\forall x,y\in A:\|F(x)-F(y)\|\le L\cdot \|x-y\|\]
    \end{enumerate}
    \renewcommand{\labelenumi}{(\arabic{enumi})}
    Dann existiert genau ein $x^*\in A$ mit $F(x^*)=x^*$.\\
    Ist $x_0\in A$ beliebig und $(x_n)$ definiert durch $x_{n+1}:=F(x_n)\ (n\ge 0)$,
    so ist $x_n\in A$ für alle $n\in\MdN$ und es gilt:
    \[x_n\stackrel{n\to\infty}\to x^*\]
    Weiter gilt für alle $n\in\MdN$:
    \[\|x_n-x^*\|\le\frac{L^n}{1-L}\|x_1-x_0\|\]
    Diese Folge heißt Folge der \textbf{sukzessiven Approximationen}.
\end{satz}

\begin{beweis}
Sei $x_0\in A$ und $(x_n)$ wie oben definiert. Es gilt:
\[\|x_2-x_1\|=\|F(x_1)-F(x_2)\|\le L\cdot \|x_1-x_0\|\]
Induktiv lässt sich zeigen:
\[\forall k\in\MdN_0: \|x_{k+1}-x_k\|\le L^k\cdot \|x_1-x_0\|\]
Seien nun $m,n\in\MdN$ und $m>n$, dann gilt:
\begin{align*}
\|x_m-x_n\| &= \|(x_m-x_{m-1})+\cdots+(x_{n+1}-x_n)\|\\
&\le \|x_m-x_{m-1}\|+\cdots+\|x_{n+1}-x_n\|\\
&\le L^{m-1}\|x_1-x_0\|+\cdots+L^n\|x_1-x_0\|\\
&=(L^{m-1}+\cdots+L^n)\cdot \|x_1-x_0\|\\
&= L^n(1+\cdots+L^{m-n-1})\cdot \|x_1-x_0\|\\
&\le L^n(\sum_{j=0}^\infty L^j)\cdot \|x_1-x_0\|\\
&= \frac{L^n}{1-L}\|x_1-x_0\|
\end{align*}
Also ist $(x_n)$ eine CF. Da $X$ außerdem $BR$ ist, existiert ein $x^*\in X$ mit
$x_n\to x^*$. Wegen $(x_n)\subseteq A$ und $A$ abgeschlossen ist außerdem $x^*\in A$.\\
Festes $n$ und $m\to\infty$ liefert aus obiger Gleichung:
\[\forall n\in\MdN:\|x_n-x^*\|\le \frac{L^n}{1-L}\|x_1-x_0\|\]
Für $F(x^*)$ gilt also:
\begin{align*}
\|F(x^*)-x^*\| &= \|F(x^*)-x_{n+1}+x_{n+1}-x^*\|\\
&\le \|F(x^*)-x_{n+1}\|+\|x_{n+1}-x^*\|\\
&=\|F(x^*)-F(x_n)\|+\|x_{n+1}-x^*\|\\
&\le L\|x^*-x_n\|+\|x_{n+1}-x^*\|\stackrel{n\to\infty}\to 0
\end{align*}
Daraus folgt:
\[\|F(x^*)-x^*\| = 0\iff F(x^*)=x^*\]
Sei nun $z\in A$ und $F(z)=z$. Es gilt:
\begin{align*}
&\|x^*-z\|=\|F(x^*)-F(z)\| \le L\|x^*-z\|\\
\implies &(1-L)\|x^*-z\| \le 0\\
\implies &x^*=z
\end{align*}
Also ist $x^*$ eindeutig.
\end{beweis}

\chapter{Differentialgleichungen: Grundbegriffe}
In diesem Paragraphen seien $I,J,\ldots$ immer Intervalle in $\MdR$.

\begin{erinnerung}
Seien $p,k\in\MdN$. Eine Funktion $y=(y_1,\ldots,y_p):I\to\MdR^p$ heißt
$k$-mal (stetig) db, genau dann wenn $y_1,\ldots,y_p$ $k$-mal auf $I$ (stetig)
db sind.\\
In diesem Fall ist $y^{(j)}=(y_1^{(j)},\ldots,y_p^{(j)})\ (j=1,\ldots,k)$.
\end{erinnerung}

\begin{definition}
\index{gewöhnliche Differentialgleichung}\index{Differentialgleichung!gewöhnliche}
\index{Differentialgleichung!Lösung}\index{Lösung!einer Differentialgleichung}
Seien $n,p\in\MdN$, sei weiter
$D\subseteq\MdR\times\underbrace{\MdR^p\times\cdots\times\MdR^p}_{n+1 \text{ Faktoren}}$
und $F:D\to\MdR^p$ eine Funktion.\\
Eine Gleichung der Form:
\begin{align*}
F(x,y,\ldots,y^{(n)})=0 \tag{i}
\end{align*}
heißt eine \textbf{(gewöhnliche) Differentialgleichung} (Dgl) \textbf{$n$-ter Ordnung}.\\
Eine Funktion $y:I\to\MdR^p$ heißt eine \textbf{Lösung} (Lsg) von (i), genau dann wenn
$y$ auf $I$ $n$-mal db, für alle $x\in I, (x,y(x),\ldots,y^{(n)}(x))\in D$ ist und gilt:
\[\forall x\in I: F(x,y(x),\ldots,y^{(n)}(x))=0\]
\end{definition}

\begin{beispiele}
\item Sei $n=p=1$, $F(x,y,z)=z+\frac yx$ und $D=\Set{(x,y,z)\in\MdR^3 | x\ne 0}$, dann ist die
zugehörige Dgl:
\[y'+\frac yx =0\]
Z.B. ist $y:(0,\infty)\to\MdR, y(x)=\frac1x$ eine Lösung der Dgl.\\
Weitere Lösungen sind:
\begin{align*}
y:(0,1)\to\MdR,y(x):=0\\
y:(-\infty,0)\to\MdR,y(x):=\frac3x
\end{align*}
\item Sei $n=1,p=2$ und folgende Dgl gegeben:
\[y'=\begin{pmatrix}y_1'\\y_2'\end{pmatrix}=\begin{pmatrix}-y_2\\y_1\end{pmatrix}\]
Dann ist $y:\MdR\to\MdR^2, y(x):=(\cos x,\sin x)$ eine Lösung der Dgl.
\end{beispiele}

\begin{definition}
\index{explizite Differentialgleichung}\index{Differentialgleichung!explizite}
Seien $n,p\in\MdN, D\subseteq \MdR\times\underbrace{\MdR^p\times\cdots\times\MdR^p}_{n \text{ Faktoren}}$
und $f:D\to\MdR^p$.\\
Eine Gleichung der Form:
\begin{align*}
y^{(n)}=f(x,y,\ldots,y^{(n-1)})\tag{ii}
\end{align*}
heißt \textbf{explizite Differentialgleichung $n$-ter Ordnung}.\\
(hier gilt: $F(x,y,\ldots,y^{(n)})=y^{(n)}-f(x,y,\ldots,y^{(n-1)})$)
\end{definition}

\begin{definition}
\index{Anfangswertproblem}
\index{Anfangswertproblem!Lösung}\index{Lösung!eines Anfangswertproblems}
\index{eindeutige Lösung}\index{Lösung!eindeutige}
Seien $p,n,D$ und $f$ wie oben. Weiter sei $(x_0,y_0,\ldots,y_{n-1})\in D$ fest.\\
Dann heißt:
\begin{align*}
\begin{cases}y^{(n)}=f(x,y,\ldots,y^{(n-1)})\\
y(x_0)=y_0,\ldots,y^{(n-1)}(x_0)=y_{n-1}\end{cases}\tag{iii}
\end{align*}
ein \textbf{Anfangswertproblem} (AwP).\\
Eine Funktion $y:I\to\MdR^p$ heißt eine \textbf{Lösung} des AwP (iii), genau dann
wenn $y$ eine Lösung von (ii) ist und gilt:
\[y(x_0)=y_0,\ldots,y^{(n-1)}(x_0)=y_{n-1}\]
Das AwP heißt \textbf{eindeutig lösbar}, genau dann wenn (iii) eine Lösung hat und
für je zwei Lösungen $y:I\to\MdR^p,\tilde y:J\to\MdR^p$ von (iii) gilt:
\[\forall x\in I\cap J:y(x)=\tilde y(x)\]
(Beachte: $x_0\in I\cap J$)
\end{definition}

\begin{beispiele}
\item Sei $n=p=1$ und das folgende AwP gegeben:
\[\begin{cases}
y'=2\sqrt{|y|}\\
y(0)=0
\end{cases}\]
Dann sind:
\begin{align*}
&y:\MdR\to\MdR, x\mapsto x^2\\
&y:\MdR\to\MdR, x\mapsto 0
\end{align*}
Lösungen des AwP.
\item Sei $n=p=1$ und das folgende AwP gegeben:
\[\begin{cases}
y'=y\\
y(0)=1
\end{cases}\]
Dann ist:
\[y:\MdR\to\MdR, x\mapsto e^x\]
eine Lösung des AwP. In §19 werden wir sehen, dass dieses AwP eindeutig lösbar ist.
\end{beispiele}

\chapter{Lineare Differentialgleichungen 1. Ordnung}
In diesem Paragraphen sei $I\subseteq\MdR$ ein Intervall und $a,s:I\to\MdR$ \textbf{stetig}.
Weiter sei $J\subseteq I$ ein Teilintervall von $I$.

\begin{definition}
\index{linear!Differentialgleichung (1. Ordnung)}\index{Differentialgleichung!lineare (1.Ordnung)}
\index{homogen!Differentialgleichung}\index{Differentialgleichung!homogene}
\index{inhomogen!Differentialgleichung}\index{Differentialgleichung!inhomogene}
\index{Störfunktion}
Die Differentialgleichung
\begin{align*}
& y'=a(x)y+s(x)&\tag{$*$}
\end{align*}
heißt \textbf{lineare Differentialgleichung 1. Ordnung}. Sie heißt \textbf{homogen},
falls $s\equiv 0$, anderenfalls heißt sie \textbf{inhomogen}. $s$ heißt \textbf{Störfunktion}.
\end{definition}

Wir betrachten zunächst die zu ($*$) gehörende \textbf{homogene Gleichung}:
\begin{align*}
&y'=a(x)y&\tag{H}
\end{align*}
Aus Ana I 23.14 folgt, dass $a$ auf $I$ eine Stammfunktion $A$ besitzt.

\begin{satz}[Lösung einer homogenen linearen Dgl 1. Ordnung]
Sei $y:J\to\MdR$ eine Funktion. $y$ ist genau dann eine Lsg von (H), wenn ein $c\in\MdR$
existiert mit:
\[y(x)=c\cdot e^{A(x)}\]
\end{satz}

\begin{beweis}
\begin{enumerate}
\item["`$\impliedby$"'] Es existiere ein $c\in\MdR$, sodass $y(x)=ce^{A(x)}$ für $x\in J$. Dann gilt:
\[\forall x\in J: y'(x)=c\cdot e^{A(x)}\cdot A'(x)=a(x)\cdot c\cdot e^{A(x)}=a(x)y(x)\]
\item["`$\implies$"'] Sei $g(x):=\frac{y(x)}{e^{A(x)}}$. Nachrechnen: $\forall x\in J:g'(x)=0$\\
Aus Ana I folgt, dass ein $c\in\MdR$ existiert, sodass für alle $x\in J$ gilt $g(x)=c$.
\end{enumerate}
\end{beweis}

\begin{satz}[Eindeutige Lösung eines Anfangswertproblems]
Sei $x_0\in I,y_0\in\MdR$. Dann hat das
\[\text{AwP}
\begin{cases}
y'=a(x)y\\
y(x_0)=y_0
\end{cases}\]
auf $I$ genau eine Lösung.
\end{satz}

\begin{beweis}
Sei $c\in\MdR,y(x)=c\cdot e^{A(x)}$ für alle $x\in I$. Dann folgt aus 19.1, dass
$y$ eine Lösung von (H) ist. Außerdem gilt:
\begin{align*}
&y_0=y(x_0)\\
\iff &y_0=c\cdot e^{A(x_0)}\\
\iff &c=y_0\cdot e^{-A(x_0)}
\end{align*}
\end{beweis}

\begin{beispiel}
Sei das folgende AwP gegeben:
\[\begin{cases}
y'=\sin(x)y\\
y(0)=1
\end{cases}\]
Die allgemeine Lösung der homogenen Gleichung $y'=\sin(x)y$ ist für $c\in\MdR$:
\[\importantbox{y(x)=c\cdot e^{-\cos(x)}}\]
Außerdem gilt:
\[1=y(0)=c\cdot e^{-\cos(0)}=\frac ce\]
Also folgt $c=e$ und damit ist die Lösung des AwP $y(x)=e^{1-\cos(x)}$.
\end{beispiel}

\index{Variation der Konstanten}
Nun betrachten wir die \textbf{inhomogene Gleichung}
\begin{align*}
y'=a(x)y+s(x)\tag{IH}
\end{align*}
Für eine spezielle Lösung $y_s$ von (IH) macht man den Ansatz $y_s(x)=c(x)\cdot e^{A(x)}$
mit einer (unbekannten) db Funktion $c$. Dies heißt \textbf{Variation der Konstanten}.\\
Mit diesem Ansatz gilt:
\begin{align*}
y_s'(x)&=c'(x)\cdot e^{A(x)}+c(x)\cdot e^{A(x)}\cdot a(x)\\
&\stackrel{!}{=}a(x)y_s(x)+s(x)\\
&=a(x)c(x)\cdot e^{A(x)}+s(x)
\end{align*}
Dies ist äquivalent dazu, dass gilt:

\begin{align*}
&c'(x)\cdot e^{A(x)}=s(x)\\
\iff &c'(x)=s(x)\cdot e^{-A(x)}\\
\iff &\importantbox{c(x)=\int s(x)\cdot e^{-A(x)}\text{ d}x}
\end{align*}
Ist also $c$ eine Stammfunktion von $s\cdot e^{-A}$, so ist $y_s(x):=c(x)\cdot e^{A(x)}$
eine Lösung von (IH). Insbesondere besitzt (IH) auf $I$ Lösungen.

\begin{beispiel}
Sei folgende inhomogene Gleichung gegeben:
\begin{align*}
y'=\sin(x)y+\sin(x)\tag{$*$}
\end{align*}
Der Ansatz $y_s(x)=c(x)\cdot e^{-\cos(x)}$ für eine spezielle Lösung von ($*$) liefert wie oben:
\[c(x)=\int \sin(x)\cdot e^{\cos(x)}\text{ d}x = -e^{\cos(x)}\]
Dann ist $y_s(x)=-e^{\cos(x)}\cdot e^{-\cos(x)}=-1$.
\end{beispiel}

\begin{definition}
Definiere die Lösungsmengen:
\begin{align*}
    L_H   &:=\Set{y:I\to\MdR | y\text{ ist eine Lösung von (H)} }\\
    L_{IH}&:=\Set{y:I\to\MdR | y\text{ ist eine Lösung von (IH)}}
\end{align*}
16.1$\implies$ $L_H=\Set{c\cdot e^{A} | c\in \MdR}$. Bekannt: $L_{IH}\ne\emptyset$.
\end{definition}

\begin{satz}[Lösungen]
Sei $y_s\in L_{IH}, x_0\in I,y_0\in \MdR$.
\begin{enumerate}
\item $y\in L_{IH}\iff \exists y_h\in L_{H}: y=y_h+y_s$
\item Das
\[
\text{AwP}
\begin{cases}
y'=a(x)y+s(x)\\
y(x_0)=y_0
\end{cases}\]
hat auf $I$ genau eine Lösung
\end{enumerate}
\end{satz}

\begin{beweis}
Leichte Übung!
\end{beweis}

\begin{beispiele}
\item ($I=\MdR$) Bestimme die allg. Lösung von
\begin{align*}
y'=2xy+x\tag{$*$}
\end{align*}
1. Bestimme die allg. Lösung der Gleichung $y'=2xy$: $y(x)=ce^{x^2} (c\in\MdR)$.\\
2. Bestimme eine spezielle Lösung von $(*)$: $y_s(x)=ce^{x^2}$ mit $c(x)=\int xe^{-x^2}=-\frac 12 e^{-x^2}$
Also: $y_s(x)=-\frac12$\\
3. Die Allgemeine Lösung von $(*)$ lautet:
\[y(x)=ce^{x^2}-\frac 12 \quad(c\in\MdR)\]
\item Löse das AwP:
\[\begin{cases}
y'=2xy+x\\
y(1)=-1
\end{cases}\]
Allg. Lösung der Dgl: $y(x)=ce^{x^2}-\frac 12$\\
\[-1=y(1)=ce-\frac 12\implies c=-\frac 1{2e}\]
Lösung des AwPs: $y(x)=-\frac1{2e}e^{x^2}-\frac 12$.
\end{beispiele}

\chapter{Differentialgleichungen mit getrennten Veränderlichen}

In diesem §en seien $I,J\subseteq\MdR$ Intervalle, $f\in C(I),g\in C(J),x_0\in I$ und
$y_0\in J$.

\begin{definition}
\index{Differentialgleichung!mit getrennten Veränderlichen}
\index{getrennte Veränderliche!Differentialgleichung mit}
Die Differentialgleichung:
\begin{align*}
y'=f(x)g(y)\tag{i}
\end{align*}
heißt \textbf{Differentialgleichung mit getrennten Veränderlichen}.\\
Wir betrachten auch noch das
\begin{align*}
\text{AwP}
\begin{cases}
y'=f(x)g(y)\\
y(x_0)=y_0
\end{cases}
\tag{ii}
\end{align*}
\end{definition}

\begin{satz}[Lösungen]
Sei $y_0\in J^0$ (also ein innerer Punkt von $J$) \textbf{und} $g(y)\ne 0\ \forall y\in J$.\\
Dann existiert ein Intervall $I_{x_0}$ mit $x_0\in I_{x_0}\subseteq I$ und:
\begin{enumerate}
\item Das AwP (ii) hat eine Lösung $y:I_{x_0}\to\MdR$.
\item Die Lösung aus (1) erhält man durch Auflösen der folgenden Gleichung nach $y(x)$.
\begin{align*}
\importantbox{\int_{y_0}^{y(x)}\frac 1{g(t)}\text{ d}t=\int_{x_0}^x f(t)\text{ d}t\tag{$*$}}
\end{align*}
\item Sei $U\subseteq I$ ein Intervall und $u:U\to\MdR$ eine Lösung des AwPs (ii),
so ist $U\subseteq I_{x_0}$ und $u=y$ auf $U$ (wobei $y$ die Lösung aus (1) ist).\\
Insbesondere ist das AwP (ii) eindeutig lösbar.
\end{enumerate}
\end{satz}

\begin{beweis}
Definiere $G\in C^1(J)$ und $F\in C^1(I)$ durch:
\begin{align*}
&G(y):=\int_{y_0}^y \frac 1{g(t)}\text{ d}t &&F(x):=\int_{x_0}^x f(t)\text{ d}t
\end{align*}
Dann ist $G'=\frac 1g, F'=f$ und $F(x_0)=0=G(y_0)$.\\
Da für alle $y\in J$ gilt:
\[G'(y)=\frac 1{g(y)}\ne 0\]
ist entweder $G'>0$ auf $J$ oder $G'<0$ auf $J$.\\
Also existiert die Umkehrabbildung $G^{-1}:G(J)\to J$, $K:=G(J)$ ist ein Intervall und es gilt:
\begin{align*}
y_0\in J^0&\implies 0=G(y_0)\in K^0\\
&\implies \exists \ep>0:(-\ep,\ep)\subseteq K
\end{align*}
Da $F$ stetig in $x_0$ ist, existiert ein $\delta>0$ mit:
\[|F(x)|=|F(x)-F(x_0)|<\ep \quad \forall x\in U_\delta(x_0)\cap I=:M_0\]
$M_0$ ist ein Intervall, $x_0\in M_0\subseteq I$ und $F(M_0)\subseteq K$. Sei
\[\mathfrak{M}:=\Set{M\subseteq I | M \text{ ist Intervall},x_0\in M,F(M)\subseteq K}\]
Da $M_0\in\mathfrak{M}$ ist, ist $\mathfrak{M}\ne\emptyset$. Sei
\[I_{x_0}:=\bigcup_{M\in\mathfrak{M}} M\]
dann ist $I_{x_0}\in\mathfrak{M}$. Definiere nun $y:I_{x_0}\to\MdR$ durch:
\[y(x):=G^{-1}(F(x))\]
so ist $y$ auf $I_{x_0}$ differenzierbar und es gilt:
\[y(x_0)=G^{-1}(F(x_0))=G^{-1}(0)=y_0\]
Weiter gilt:
\begin{align*}\forall x\in I: G(y(x))=F(x)\tag{+}\end{align*}
also gilt $(*)$.
Differenzierung von (+) liefert:
\begin{align*}
&\forall x\in I_{x_0}: G'(y(x)y'(x)=F'(x)\\
\implies &\forall x\in I_{x_0}: \frac 1{g(y(x))}y'(x)=f(x)\\
\implies &\forall x\in I_{x_0}: y'(x)=f(x)g(y(x))
\end{align*}
\begin{itemize}
\item[(3)] Es ist $u'(t)=f(t)g(u(t))$ für alle $t\in U$ \textbf{und} $u(U)\subseteq J$.
Daraus folgt:
\begin{align*}
&f(t)=\frac{u'(t)}{g(u(t))}\\
\implies &F(x)=\int_{x_0}^x f(t)\text{ d}t=\int_{x_0}^x \frac{u'(t)}{g(u(t))}\text{ d}t\\
&\stackrel{Subst.}{=}
\begin{cases}
s=u(t)\\
\text{ d}s= u'(t)\text{ d}t\\
t=x_0\implies s=u(x_0)=y_0
\end{cases}=\int_{y_0}^{u(x)}\frac 1{g(s)}\text{ d}s=G(u(x))
\end{align*}
Also: $\forall x\in U:F(x)=G(u(x))$. Somit gilt:
\[F(U)=G(u(U))\subseteq G(J)=K\]
D.h. $U\in\mathfrak{M}$ und daher ist: $U\subseteq I_{x_0}$.\\
Weiter gilt:
\[\forall x\in U: u(x)=G^{-1}(F(x))=y(x)\]
\end{itemize}
\end{beweis}

\textbf{Für die Praxis: Trennung der Veränderlichen (TDV):}\\
\begin{align*}
&y'=f(x)g(y)\\
\to\ &\frac{\text{ d}y}{\text{ d}x}=f(x)g(y)\\
\to\ &\frac{\text{ d}y}{g(y)}=f(x)\text{ d}x\\
\to\ &\int{\frac{\text{ d}y}{g(y)}}=\int f(x)\text{ d}x+c\tag{iii}
\end{align*}
Die allgemeine Lösung von (i) erhält man durch Auflösen der Gleichung (iii) nach $y$.\\
Zur Lösung von (ii) passt man die Konstante $c$ der Anfangsbedingung $y(x_0)=y_0$ an.

\begin{beispiele}
\item Sei $y'=2xe^{-y}$. Dann gilt:
\begin{align*}
\frac{\text{ d}y}{\text{ d}x}=2xe^{-y}\\
\to\ &e^y\text{ d}y = 2x\text{ d}x\\
\to\ &\int e^y\text{ d}y=\int 2x\text{ d}x+c\\
\to\ & e^y=x^2+c\\
\to\ &y=\log(x^2+c)
\end{align*}
Ist z.B. $c=0$, so ist $y(x):=\log(x^2)$ eine Lösung auf $(0,\infty)$, oder
$y(x)=\log(x^2)$ ist eine auf $(-\infty,0)$.\\
$c=2: y(x)=\log(x^2+2)$ ist eine Lösung auf $\MdR$.\\
$c=-1: y(x)=\log(x^2-1)$ ist eine Lösung auf $(1,\infty)$.\\
Löse das
\begin{align*}
\text{AwP}
\begin{cases}
y'=2xe^{-y}\\
y(1)=1
\end{cases}
\end{align*}
Allg. Lösung der Dgl:
\begin{align*}
&y(x)=\log(x^2+c)\\
\implies &1=y(1)=\log(1+c)\\
\implies &e=1+c \iff c=e-1
\end{align*}
$y(x)=\log(x^2+e-1)$ ist Lösung des AwPs auf $\MdR$.
\item $y'=\frac{x^2}{1-x}\cdot\frac{1+y}{y^2}$. Trennung der Veränderlichen:
\begin{align*}
&\frac{\text{ d}y}{\text{ d}x}=\frac{x^2}{1-x}\cdot\frac{1+y}{y^2}\\
\to\ &\frac{y^2}{y+1}\text{ d}y=\frac{x^2}{x-1}\text{ d}x\\
\to\ &\frac{y^2}2-y+\log(1+y)=\frac{x^2}2+x+\log(x-1)+c
\end{align*}
(Lösungen in impliziter Form)
\end{beispiele}


\chapter{Systeme von Differentialgleichungen 1. Ordnung}

In diesem Paragraphen sei $D \subseteq \MdR^{n+1}$ und $f = (f_1, \ldots, f_n): D \to \MdR^n$. Für Punkte im $\MdR^{n+1}$ schreiben wir $(x,y)$, wobei $x \in \MdR$ und $y = (y_1,...,y_n) \in \MdR^n$.

\begin{definition}
\index{System von Differentialgleichungen}\index{Differentialgleichung!System von}
\index{Anfangswertproblem}
Ein \textbf{System von Differentialgleichungen 1. Ordnung} hat die Form:
\begin{align*}
\begin{cases}
y_1'=f_1(x, y_1, \ldots, y_n)\\
y_2'=f_2(x, y_1, \ldots, y_n)\\
\quad\ \vdots\\
y_n'=f_n(x, y_1, \ldots, y_n)
\end{cases}
\tag{i}
\quad\text{oder kurz: } y'=f(x,y)
\end{align*}
Wir betrachten auch noch das
\begin{align*}
\text{AwP}
\begin{cases}
y'=f(x,y)\\
y(x_0) = y_0\\
\end{cases}
\tag{ii}
\quad\text{(wobei } (x_0, y_0) \in D)
\end{align*}
\end{definition}

\begin{satz}[Integralgleichung zur Lösbarkeit eines Anfangswertproblems]
Sei $I \subseteq \MdR$ ein Intervall, $D := I \times \MdR^n, x_o\in I, y_0\in \MdR^n$ und $f:D\to\MdR^n$ sei stetig. Für $y\in C(I,\MdR^n)$ gilt:
\[\text{$y$ ist eine Lösung des AwP (ii)} \iff  \forall x\in I:y(x) = y_0 + \int_{x_0}^x f(t, y(t)) \text{d}t \]
In diesem Fall ist $y \in C^1(I, \MdR^n)$.
\end{satz}

\begin{beweis}
\begin{enumerate}
\item["`$\implies$"'] Es gilt: $y'(x) = f(x, y(x)) \forall x\in I$; da $y$ und $f$ stetig sind, folgt: $y' \in C(I,\MdR )$. Weiter:
\[\int_{x_0}^x f(t,y(t)) \text{d}t = \int_{x_0}^x y'(t)\text{d}t = y(x) - y(x_0) = y(x) - y_0 \quad \forall x\in I.\]
Bringt man $y_0$ auf die linke Seite, ergibt sich die Behauptung.
\item["`$\impliedby$"'] Es gelte für alle $x\in I$:
\[y(x) = y_0  + \int_{x_0}^x f(t, y(t))\text{d}t\]
Aus dem zweiten Hauptsatz der Differential- und Integralrechnung folgt: $y$ ist auf $I$ differenzierbar und
\[ y'(x) = \frac{\text{d}}{\text{d}x} \int_{x_0}^x f(t, y(t))\text{d}t = f(x, y(x)) \quad \forall x \in I. \]
Also erfüllt $y$ die Differentialgleichung.
Klar: $y(x_0) = y_0$. Also löst $y$ das AwP.
\end{enumerate}
\end{beweis}


\begin{definition}
\index{Lipschitz-Bedingung}
\index{Lipschitz-Bedingung!lokale}\index{lokal!Lipschitz-Bedingung}
Es sei weiterhin $D \subseteq \MdR^{n+1}$. Sei $f: D \to \MdR^n$ eine Funktion.
\begin{enumerate}
\item $f$ genügt auf $D$ einer \textbf{Lipschitz-Bedingung bezüglich \boldmath \(y\)}
\[
    :\iff
    \exists L \ge 0:
        \forall (x,y), (x,\bar y ) \in D:
            \|f(x,y)-f(x,\bar y)\| \le L \|y-\bar y \|
\]
\item $f$ genügt auf $D$ einer \textbf{lokalen Lipschitz-Bedingung bezüglich \boldmath \(y\)}
\[
    :\iff
    \forall a \in D \exists \text{Umgebung } U_a:
    f_{|_{D \cap U}} \text{ genügt einer Lipschitz-Bedingung bzgl. } y
\]
\end{enumerate}
\end{definition}

\begin{satz}[Satz über die $\alpha$-Norm]
Sei $I = [a,b] \subseteq \MdR, x_o \in I$ und für $y\in C(I, \MdR^n)$ sei $\|y\|_\infty := \max \{\|y(x)\| : x\in I \}$ wie in §17 (also ist $(C(I, \MdR^n), \|\cdot \|_\infty )$ ein Banachraum).

Sei $\alpha > 0$ mit $\varphi_\alpha (x) := e^{-\alpha |x-x_0|}\ (x \in I)$.

Für $y \in C(I, \MdR^n)$ sei $\|y\|_\alpha := \max \{\varphi_\alpha(x)\cdot \|y(x)\| : x\in I \}$.

Dann:
\begin{enumerate}
\item $\|\cdot\|_\alpha$ ist eine Norm auf $C(I,\MdR^n)$.
\item Seien $c_1 := \min \{ \varphi_\alpha(x) : x \in I \},\text{ } c_2 := \max \{ \varphi_\alpha(x) : x \in I \}$. Es gilt:
\[c_1\|y\|_\infty \leq \|y\|_\alpha \leq c_2 \|y\|_\infty \quad \forall y \in C(I, \MdR^n)\]

\item Sei $(g_k)$ eine Folge in $C(I,\MdR^n)$ und $g \in C(I, \MdR^n)$.
\begin{enumerate}
\item Es gilt:
\begin{align*}
\|g_k -g\|_\alpha \stackrel{k \to \infty}\to 0 &\iff \|g_k - g\|_\infty \stackrel{k \to \infty}\to 0\\
&\iff (g_k)\text{ konvergiert auf $I$ gleichmäßig gegen $g$}
\end{align*}
\item $(g_k)$ ist eine Cauchy-Folge in $(C(I,\MdR^n), \|\cdot \|_\alpha)$, genau dann
wenn $(g_k)$ eine Cauchy-Folge in $(C(I,\MdR^n), \|\cdot \|_\infty)$ ist.
\item $(C(I,\MdR^n), \|\cdot \|_\alpha)$ ist ein Banachraum.
\end{enumerate}
\end{enumerate}
\end{satz}

\begin{beweis}
(1), (2) \text{ Nachrechnen.}

(3) \text{ (i) und (ii) folgen aus (2); (iii) folgt aus (i) und (ii).}
\end{beweis}

\textbf{Bezeichnung:} EuE = Existenz und Eindeutigkeit.
\index{Existenz und Eindeutigkeit}
\begin{satz}[EuE-Satz von Picard-Lindelöf (Version I)]
Sei $I = [a,b], x_o \in I, y_0 \in \MdR^n, D:= I \times \MdR^n, f\in C(D, \MdR^n)$
und $f$ genüge auf $D$ einer Lipschitz-Bedingung bezüglich $y$.\\
\\
Dann ist das
\begin{align*}\text{AwP}
\begin{cases}
y'=f(x,y)\\
y(x_0) = y_0\\
\end{cases}
\tag{ii}
\end{align*}
auf $I$ eindeutig lösbar.

Ist $g_0 \in C(I, \MdR^n)$ und $(g_k)$ definiert durch
\[ g_{k+1}(x) := y_0 + \int_{x_0}^x f(t, g_k(t)) \text{d}t \quad (x \in I, k \geq 0), \]
dann konvergiert $(g_k)$ auf $I$ gleichmäßig gegen die Lösung des AwPs (ii).\\
$(g_n)$ heißt Folge der sukzessiven Approximationen.
\end{satz}

\begin{beweis}
Da $f$ auf $D$ einer Lipschitz-Bedingung genügt, gilt:
\[\exists L > 0: \|f(x,y) - f(x, \bar y )\| \leq L \|y- \bar y \| \quad \forall(x,y), (x, \bar y ) \in D.\]
Es sei $\alpha := 2L$; $\varphi_\alpha$ und $\|\cdot\|_\alpha$ seien wie in 21.2, $X := C(I, \MdR^n)$. Definiere $F: X \to X$ durch
\[(F(y))(x) := y_0 + \int_{x_0}^x f(t, y(t))\text{d}t\]

Für $y \in X$ gilt dann:
\begin{align*}
    y \text{ ist Lösung des AwP} &\iff F(y) = y\\
    F(y) = y &\iff y(x) = y_0 + \int_{x_0}^x f(t, y(t))\text{d}t \quad \forall x \in I \\
    &\stackrel{21.1}\iff y \text{ löst das AwP (ii)}
\end{align*}
Wir zeigen: $\|F(y)-F(z)\|_\alpha \leq \frac12 \|y-z\|_\alpha \quad \forall y,z \in X$. \textbf{Alle} Behauptungen folgen dann aus 17.2.

Seien $y,z \in X, x \in I$. Dann ist
\begin{align*}
\|(F(y))(x) - (F(z))(x)\|&= \left\|\int_{x_0}^x (f(t, y(t)) - f(t, z(t)))\text{ d}t \right\|\\
&\stackrel{12.4}\le \left| \int_{x_0}^x \|f(t, y(t)) - f(t,z(t))\| \text{ d}t \right|\\
&\le \left| \int_{x_0}^x L \|y(t)-z(t)\| \text{ d}t \right|\\
&= L \left| \int_{x_0}^x \|y(t)-z(t)\| \text{ d}t \right| \displaybreak[0]\\
&= L \left| \int_{x_0}^x \|y(t)-z(t)\| \varphi_\alpha (t) \cdot \frac1{\varphi_\alpha(t)}\text{ d}t \right| \displaybreak[0]\\
&\le L \left| \int_{x_0}^x \|y-z\|_\alpha \cdot \frac1{\varphi_\alpha(t)}\text{ d}t \right| \displaybreak[0]\\
&\le L \|y-z\|_\alpha \left| \int_{x_0}^x \frac1{\varphi_\alpha(t) }\text{ d}t\right|\\
&= \frac{L}{\alpha} \|y-z\|_\alpha \left(\frac1{\varphi_\alpha(x)} -1 \right)\\
&\le \frac12 \|y-z\|_\alpha \frac{1}{\varphi_\alpha(x)}
\end{align*}
Also gilt:
\begin{align*}
&\|(F(y))(x) - (F(z))(x)\| \leq \frac12 \|y-z\|_\alpha \frac{1}{\varphi_\alpha(x)} \quad \forall x \in I\\
\implies &\varphi_\alpha(x) \|(F(y))(x) - (F(z))(x)\| \leq \frac12 \|y-z\|_\alpha \quad \forall x \in I
\end{align*}
Fazit: $\|F(y)-F(z)\|_\alpha \leq \frac12 \|y-z\|_\alpha$.
\end{beweis}

\textbf{Frage: } Warum haben wir in obigem Beweis nicht die $\|\cdot \|_\infty$-Norm benutzt?

\begin{align*}
\|(F(y))(x) -(F(z))(x)\| &\stackrel{\text{wie oben}}\leq L \left| \int_{x_0}^x \|y(t)-z(t)\| dt \right| \\
&\leq L \left| \int_{x_0}^x \|y-z\|_\infty dt \right| \\
&\leq L \|y-z\|_\infty \left| \int_{x_0}^x 1 \text{d}t \right|\\
&= L \|y-z\|_\infty |x-x_0|\\
&\leq L(b-a) \|y-z\|_\infty \quad \forall x \in I
\end{align*}
Dann:  $\|F(y)-F(z)\|_\infty \leq L(b-a)\|y-z\|_\infty$
I.A. wird $L(b-a)$ \textbf{nicht} kleiner 1 sein!


\begin{beispiel}[zu 21.3]
Zeige, dass das
\begin{align*}
\text{AwP}
\begin{cases}
y' = 2x(1+y)\\
y(0) =  0
\end{cases} \end{align*}
\textbf{auf $\mathbb{R}$} genau eine Lösung hat.\\
Sei $a > 0$ und $I := [-a, a]; f(x,y) = 2x(1+y).$ Dann gilt $\forall x \in I, \forall y, \bar y \in \mathbb{R}:$
\begin{align*}
|f(x,y)-f(x, \bar y )|&= |2xy-2x\bar y | \\
&= 2|x||y-\bar y | \\
&\leq 2a|y-\bar y |.
\end{align*}

Aus 21.3 folgt dann: das Anfangswertproblem hat auf $I$ genau eine Lösung $y: [-a, a] \to \mathbb{R}$.
Setze nun $g_0(x) :=0$ und $(g_k)$  sei definiert wie in 21.3.
Induktiv sieht man (Übung!):
\[g_k(x) = x^2 + \frac{x^4}{2!} + \frac{x^6}{3!} + \cdots + \frac{x^{2k}}{k!} \]
Aus 21.3 folgt: $(g_k)$ konvergiert auf $I$ gleichmäßig gegen $y$.\\
Aus Analysis I folgt: $(g_k)$ konvergiert auf $I$ gleichmäßig gegen $e^{x^2} - 1$.\\
Also: Lösung des AwPs auf $[-a,a]$: $y(x) = e^{x^2} -1$.\\
Es war $a > 0$ beliebig, also ist $y(x) = e^{x^2} -1$ \textbf{die} Lösung des AwPs \textbf{auf $\mathbb{R}$}.
\end{beispiel}

\textbf{Ohne} Beweis:
\begin{satz}[EuE-Satz von Picard-Lindelöf (Version II)]
Sei $I=[a,b] \subseteq \MdR, x_0 \in I, y_0 \in \MdR^n, s > 0$, es sei
\[D := \Set{(x,y)\in\MdR^{n+1} | x \in I, \|y-y_0\| \leq s}\]
und $f \in C(D,\MdR^n)$. Weiter sei
\[M := \max\{\|f(x,y)\| : (x,y) \in D \} > 0\]
und $f$ genüge auf $D$ einer Lipschitz-Bedingung bezüglich $y$.
Außerdem sei
\[J := I \cap \left [x_0 - \frac{s}{M}, x_0 + \frac{s}{M} \right ]\]
Dann hat das
\begin{align*}
\text{AwP}
\begin{cases}
y' = f(x,y)\\
y(x_0) = y_0
\end{cases}
\label{(ii)}
\end{align*}
auf \(J\) genau eine Lösung.
\end{satz}

\textbf{Ohne} Beweis:
\begin{satz}[EuE-Satz von Picard-Lindelöf (Version III)]
Es sei $D \subseteq \MdR^{n+1}$ \textbf{offen}, $(x_0, y_0) \in D, f\in C(D, \MdR^n)$ und $f$ genüge auf $D$ einer \textbf{lokalen} Lipschitz-Bedingung bezüglich $y$.

Dann hat das
\begin{align*}
\text{AwP}
\begin{cases}
y' = f(x,y)\\
y(x_0) = y_0
\end{cases}
\label{(ii)}
\end{align*}
genau eine Lösung.
\\
\\
(Nochmals, das heißt: Das AwP (ii) hat eine Lösung $y: J \to \MdR^n\quad(J \subseteq \MdR$
Intervall) und für je zwei Lösungen $\hat y : \hat J \to \MdR, \tilde y : \tilde J \to \MdR$
von (ii) gilt: $\hat y = \tilde y$ auf $\hat J \cap \tilde J \quad (\hat J, \tilde J \text{ Intervalle in } \MdR$))
\end{satz}

\begin{definition}
\index{Fortsetzbarkeit}
Sei $y: J \to \MdR^n$ ($J \subseteq \MdR$ ein Intervall) eine Lösung des AwPs (ii).\\
\(y\) heißt \textbf{nicht fortsetzbar}, genau dann wenn aus $\hat y : \hat J \to \MdR^n
(\hat J$ ein Intervall in $\MdR$) ist Lösung von (ii) stets folgt, dass $\hat J \subseteq J$
und auf $\hat J$ $\hat y = y$ ist.
\end{definition}

\begin{satz}[Eindeutigkeit einer nicht fortsetzbaren Lösung]
Es seien $D, (x_0, y_0)$ und $f$ wie in 21.5. Dann besitzt das AwP (ii) eine eindeutig bestimmte, nicht fortsetzbare Lösung.
\end{satz}

\begin{beweis}
Es sei
\[\mathfrak{M} := \{ (y,I_y) : I_y \subseteq \MdR \text{ Intervall, }
x_0 \in I_y, y: I_y \to \MdR^n \text{ ist Lösung von (ii)} \}\]
Aus 21.5 folgt, dass $\mathfrak{M} \ne \emptyset$ ist und für
$(y_1, I_{y_1}), (y_2, I_{y_2}) \in \mathfrak{M}$ gilt: $y_1 = y_2$ auf
$I_{y_1} \cap I_{y_2}$.
\begin{align*}
I := \bigcup_{(y, I_y) \in \mathfrak{M}} I_y
\end{align*}
ist ein Intervall. Definiere $y: I \to \MdR^n$ wie folgt: zu $x \in I$ existiert
ein $(y_1, I_{y_1}) \in \mathfrak{M}$, sodass für $x \in I_{y_1}$ gilt: $y(x) := y_1(x)$.

\textbf{Übung:} $y: I \to \MdR^n$ leistet das Gewünschte.
\end{beweis}

\chapter{Lineare Systeme}
In diesem Paragraphen sei $I\subseteq\mdr$ ein Intervall, $x_0\in I, y_0\in\mdr^n,
D:=I\times\mdr^n,b:I\to\mdr^n$ stetig und $A:I\to\mdr^{n\times n}$ ebenfalls stetig
(d.h. für $A(x)=(a_{jk}(x))$ sind alle $a_{jk}:I\to\mdr$ stetig).\\
Hier ist für alle $x\in I$ und $y=(y_1,\ldots,y_n)\in\mdr^n$:
\[f(x,y):=A(x)y+b(x)\]

\begin{definition}
\index{linear!System von Differentialgleichungen}
\index{System von Differentialgleichungen!lineares}
\index{homogen!System von Differentialgleichungen}
\index{System von Differentialgleichungen!homogenes}
\index{inhomogen!System von Differentialgleichungen}
\index{System von Differentialgleichungen!inhomogenes}
\index{Anfangswertproblem}
Das System von Differentialgleichungen:
\begin{align*}
y'=A(x)y+b(x)\tag{S}
\end{align*}
heißt ein \textbf{lineares System}. (Fall $n=1$ siehe §19.)\\
Ist $b\equiv 0$, so heißt (S) \textbf{homogen}, anderenfalls \textbf{inhomogen}.\\
Neben (S) betrachten wir auch noch das zu (S) gehörige \textbf{homogene System}
\begin{align*}
y'=A(x)y\tag{H}
\end{align*}
und das
\begin{align*}
\text{AwP}
\tag{A}
\begin{cases}
y'=A(x)y+b(x)\\
y(x_0)=y_0
\end{cases}
\end{align*}
\end{definition}

\begin{satz}[Lösungen]
\begin{enumerate}
\item (A) hat auf $I$ genau eine Lösung.
\item Das System (S) hat Lösungen auf $I$.
\item Ist $J\subseteq I$ ein Intervall und $\hat y:J\to\mdr^n$ eine Lösung von (S),
so gibt es eine Lösung $y:I\to\mdr^n$ von (S) mit $\hat y=y$ auf $J$.
\item Sei $y_s:I\to\mdr^n$ eine spezielle Lösung von (S), dann ist $y:I\to\mdr^n$ genau dann eine
Lösung von (S) auf $I$, wenn eine Lösung $y_h:I\to\mdr^n$ von (H) existiert mit:
\[y=y_h+y_s\]
\end{enumerate}
\end{satz}

\begin{wichtigebemerkung}
Wegen 22.1(3) gehen wir immer davon aus, dass Lösungen von (S) auf ganz $I$ definiert sind.
\end{wichtigebemerkung}

\begin{beweise}
\item \textbf{Fall 1:} $I=[a,b]$\\
Es ist $f(x,y)=A(x)y+b(x)$. Sei $L:=\max\{\|A(x)\|:x\in I\}$. Für alle $(x,y),(x,\overline y)\in D$ gilt:
\begin{align*}
\|f(x,y)-f(x,\overline y) \| &=\|A(x)(y-\overline y)\|\\
&\stackrel{\text{§1}}{\le} \|A(x)\|\cdot\|y-\overline y\|\\
&\le L\|y-\overline y\|
\end{align*}
Die Behauptung folgt aus 21.3.

\textbf{Fall 2:} $I$ beliebig.\\
Sei $\mathfrak{M}:=\Set{K\subseteq I | K\text{ ist kompaktes Intervall, } x_0\in K}$.
Dann ist $I=\bigcup_{K\in\mathfrak{M}} K$.\\
Ist $x\in I$, so existiert ein $K\in\mathfrak{M}$ mit $x\in K$. Nach Fall 1. hat das
AwP auf $K$ genau eine Lösung $y_K:K\to\mdr^n$. Definiere nun $y:I\to\mdr^n$ wie folgt:
\begin{align}
y(x):=y_K(x)\tag{$*$}
\end{align}
Sei $\tilde K\in\mathfrak{M}$ mit $x\in\tilde K$ und sei $y_{\tilde K}$ die eindeutig
bestimmte Lösung von (A) auf $\tilde K$. Dann ist $y_K=y_{\tilde K}$ auf $K\cap\tilde K$, also:
\[y_K(x)=y_{\tilde K}(x)\]
D.h. $y$ ist durch ($*$) wohldefiniert.\\
\textbf{Leichte Übung}: $y$ ist auf $I$ db und löst das AwP auf $I$.\\
Sei $\tilde y:I\to\mdr^n$ eine weitere Lösung von (A) auf $I$ und sei $x\in I$.
Dann existiert ein $K\in\mathfrak{M}$ mit $x\in K$ und nach Definition gilt $y(x)=y_K(x)$.
Da $\tilde y_K$ eine Lösung des AwPs (A) auf $K$ ist, gilt nach Fall 1.: $\tilde y\mid_K=y_K$
Dann gilt also:
\[\tilde y(x)=\tilde y\mid_K(x)=y_K(x)=y(x)\]
\item Folgt aus (1).
\item Sei $\xi \in J,\eta:=\hat y(\xi)$. Dann ist $\hat y$ eine Lösung auf $J$ des AwPs
\begin{align*}
\tag{+}
\begin{cases}
y'=A(x)+b(x)\\
y(\xi)=\eta
\end{cases}
\end{align*}
Aus (1) folgt, dass das AwP auf $I$ eine eindeutig bestimmte Lösung $y:I\to\mdr^n$ hat. Sei $x\in J$.\\
\textbf{Fall $x=\xi$}:\\
In diesem Fall gilt:
\[\hat y(x)=\hat y(\xi)=\eta=y(\xi)=y(x)\]
\textbf{Fall $x>\xi$}:\\
Sei $K:=[\xi,x]$. Da $\hat y$ und $y$ Lösungen des AwPs (+) auf $[\xi,x]$ sind folgt aus
(1), dass $y=\hat y$ auf $K$, also:
\[\hat y(x)=y(x)\]
\textbf{Fall $x<\xi$}:\\
Sei $K:=[x,\xi]$. Da $\hat y$ und $y$ Lösungen des AwPs (+) auf $[x,\xi]$ sind folgt aus
(1), dass $y=\hat y$ auf $K$, also:
\[\hat y(x)=y(x)\]
\item Leichte Übung!
\end{beweise}

\begin{definition}
Setze $\mathbb{L} := \{ y: I\to \MdR^n : y $ ist eine Lösung von (H) auf $I$ $\}$\\
($y \equiv 0$ liegt in $\mathbb{L}$)
\end{definition}

\begin{satz}[Lösungsmenge als Vektorraum]
\begin{enumerate}
\item Sind $y^{(1)}, y^{(2)} \in \mathbb{L}$ und $\alpha \in \MdR$, so
sind $y^{(1)} + y^{(2)} \in \mathbb{L}$ und $\alpha y^{(1)} \in \mathbb
{L}$. $\mathbb{L}$ ist also ein reeller Vektorraum.

\item Seien $y^{(1)}, ..., y^{(k)} \in \mathbb{L}$. Dann sind
äquivalent:
 \begin{enumerate}
  \item $y^{(1)}, ... , y^{(k)}$ sind in $\mathbb{L}$ linear unabhängig.
  \item $\forall x \in I$ sind $y^{(1)}(x), ..., y^{(k)}(x)$ linear
unabhängig im $\MdR^n$.
  \item $\exists \xi \in I: y^{(1)}(\xi ), ..., y^{(k)}(\xi )
$ sind linear unabhängig im $\MdR^n$.
 \end{enumerate}

\item $\dim \mathbb{L} = n$.
\end{enumerate}
\end{satz}

\begin{beweise}
\item Nachrechnen

\item Der Beweis erfolgt durch Ringschluss:
\begin{enumerate}
\item[(i)$\implies$ (ii)] Sei $x_1 \in I$. Seien $\alpha_1, ...,
\alpha_k \in \MdR$ und
\begin{align*}
0 &= \alpha_1 y^{(1)}(x_1) + \cdots + \alpha_k y^ {(k)}(x_1)\\
\tilde y :&= \alpha_1 y^{(1)} + \cdots + \alpha_k y^{(k)}
\end{align*}
Aus (1) folgt: $\tilde y \in \mathbb{L}$. Weiter ist $\tilde y$ eine
Lösung des AwPs
\begin{align*} \begin{cases}
y' = A(x) y\\
y(x_1) = 0
\end{cases} \end{align*}
Da $y \equiv 0$ dieses AwP ebenfalls löst und aus 22.1 folgt, dass das AwP
eindeutig lösbar ist, muss gelten:
\[0 = \tilde y = \alpha_1 y^{(1)} + \cdots +  \alpha_k y^{(k)}\]
Aus der Voraussetzung folgt dann:
\[\alpha_1 = \alpha_2 = \cdots = \alpha_k = 0\]
Also sind $y^{(1)}(x_1), ..., y^{(k)} (x_1)$ sind linear unabhängig im $\MdR^n$.
\item[(ii) $\implies$ (iii)] Klar \checkmark
\item[(iii) $\implies$ (i)]Seien $\alpha_1, ..., \alpha_k \in \MdR$
und $0 = \alpha_1 y^{(1)} + \cdots + \alpha_k y^{(k)}$, dann folgt:
\[0 = \alpha_1 y^{(1)}(\xi ) + \cdots + \alpha_k y^{(k)}(\xi )\]
Aus der Voraussetzung folgt dann: $\alpha_1 = \alpha_2 = \cdots = \alpha_k = 0$
Also sind $y^{(1)}, ..., y^{(k)}$ linear unabhängig in $\mathbb{L}$.
\end{enumerate}

\item Aus (2) folgt, dass $\dim \mathbb{L} \le n$ ist.

Für $j = 1,..., n$ sei $y^{(j)}$ die eindeutig bestimmte Lösung des
AwPs
\begin{align*}
\begin{cases}
y' = A(x) y\\
y(x_0) = e_j
\end{cases}
(e_j = \text{ j-ter Einheitsvektor im }\MdR^n).
\end{align*}
Dann sind $y^{(1)}(x_0), ..., y^{(n)}(x_0)$ linear unabhängig im $
\MdR^n$. Aus (2) folgt, dass $y^{(1)}, ..., y^{(k)}$ linear unabhängig
in $\mathbb{L}$ sind, also ist $\dim \mathbb{L} \ge n$.
\end{beweise}

\begin{definition}
\index{Differenzierbarkeit!einer $n \times n$-Matrix}
Sei $B : I \to \MdR^{n \times n}, B(x) = \left( b_{jk}(x) \right)$ für alle $x\in I$.\\
$B$ heißt \textbf{differenzierbar} auf $I$, genau dann wenn $b_{jk} : I \to \MdR$
auf $I$ differenzierbar sind ($j,k = 1,\ldots, n$).\\
In diesem Fall ist
\[B'(x) := (b'_{jk}(x)) \quad (x\in I)\]
\end{definition}

\begin{definition}
\index{Lösungs-!System}\index{Lösungs-!Matrix}\index{Wronskideterminante}
\index{Fundamental-!Matrix}\index{Fundamental-!System}
\begin{enumerate}
\item Seien $y^{(1)}, ..., y^{(n)} \in \mathbb{L}$. $y^{(1)}, ..., y^{(n)}$
heißt ein \textbf{Lösungssystem} (LS) von (H).
\[Y(x) := (y^{(1)}(x), ..., y^{(n)}(x))\]
(j-te Spalte von $Y$  =  $y^{(j)}$) heißt \textbf{Lösungsmatrix} (LM) von (H).
\[W(x) := \det Y(x)\]
heißt \textbf{Wronskideterminante}.
\item Sei $y^{(1)}, ..., y^{(n)}$ ein Lösungssystem von (H). Sind
$y^{(1)}, ..., y^{(n)}$ linear unabhängig in $\mathbb{L}$, so heißt
$y^{(1)}, ..., y^{(n)}$ ein \textbf{Fundamentalsystem} (FS) und
$Y = (y^{(1)}, ..., y^{(n)})$ eine \textbf{Fundamentalmatrix} (FM).
\item Ist $y^{(1)}, ..., y^{(n)}$ ein FS von (H), so lautet die allgemeine Lösung von (H):
\[y(x) = c_1 y^{(1)}(x) + \cdots + c_n y^{(n)} (x) \quad (c_1, ..., c_n \in \MdR)\]
\end{enumerate}
\end{definition}

\begin{satz}[Zusammenhang FS, FM und Wronskideterminante]
$y^{(1)}, ..., y^{(n)}$ sei ein LS von (H). $Y$ und $W$ seien definiert wie oben. Dann:\begin{enumerate}
\item $Y'(x) = A(x)Y(x) \quad \forall x \in I$.
\item $y^{(1)}, ..., y^{(n)}$ ist ein Fundamentalsystem von (H)\\ $\iff Y(x) \text{ invertierbar } \forall x \in I$ \\ $\iff \exists \xi \in I: Y(\xi )$ ist invertierbar \\ $\iff \forall x \in I: W(x) \neq 0$ \\ $\iff \exists \xi \in I: W(\xi ) \neq 0$.
\end{enumerate}
\end{satz}

\begin{beweise}
\item Nachrechnen
\item folgt aus 22.3.
\end{beweise}


\textbf{Spezialfall:} $n=2$. $A(x) = \begin{pmatrix} a_1(x) & -a_2(x) \\ a_2(x) & a_1(x) \end{pmatrix}$; $a_1, a_2 : I \to \MdR$ stetig. Sei $y^{(1)} = (y_1, y_2)$ eine Lösung von
\begin{align*}
\tag{$*$} y' = A(x) y
\end{align*}
auf $I$ und $y^{(1)} \not\equiv 0$. Das heißt:
\begin{align*}
\begin{cases}
y_1' = a_1(x) y_1 - a_2(x) y_2 \\
y_2' = a_2(x) y_1 + a_1(x) y_2
\end{cases}.
\end{align*}

Setze $y^{(2)} := (-y_2, y_1)$. Dann ist:
\begin{align*}
A(x) y^{(2)} = \begin{pmatrix} -a_1(x) y_2 - a_2(x) y_1 \\ -a_2(x) y_2 + a_1(x) y_1 \end{pmatrix} = \begin{pmatrix} -y_2' \\ y_1' \end{pmatrix} = \left( y^{(2)} \right)'
\end{align*}
Das heißt: $y^{(2)}$ löst ebenfalls ($*$) auf $I$, oder: $y^{(1)}, y^{(2)}$ ist ein Lösungssystem von ($*$).
\begin{align*}
Y(x) = \begin{pmatrix} y_1(x) & -y_2(x) \\ y_2(x) & y_1(x) \end{pmatrix}, W(x) = \det Y(x) = y_1(x)^2 + y_2(x)^2 \neq 0
\end{align*}
Mit 22.4 folgt: $y^{(1)}, y^{(2)}$ ist ein Fundamentalsystem von ($*$).

\begin{beispiel} ($n=2$), $A = \begin{pmatrix} 0 & -1 \\ 1 & 0 \end{pmatrix}$;
\begin{align*}
\tag{$*$} y' = Ay
\end{align*}
und $y = (y_1, y_2)$. Also: $\begin{pmatrix} y_1' \\ y_2' \end{pmatrix} = \begin{pmatrix} -y_2 \\ y_1 \end{pmatrix}$.

$y^{(1)}(x) := \begin{pmatrix} \cos(x) \\ \sin(x) \end{pmatrix}$ ist eine Lösung von ($*$) auf $\MdR$.
$y^{(2)}(x) := \begin{pmatrix} -\sin(x) \\ \cos(x) \end{pmatrix}$ ist eine weitere Lösung von ($*$) auf $\MdR$.
$y^{(1)}, y^{(2)}$ ist ein Fundamentalsystem von ($*$).
Allgemeine Lösung von ($*$): $y(x) = \begin{pmatrix} c_1 \cos(x) - c_2 \sin(x) \\ c_1 \sin(x) + c_2 \cos(x) \end{pmatrix}\quad (c_1, c_2 \in \MdR)$.

\end{beispiel}

\textbf{Ohne} Beweis:

\begin{satz}[Spezielle Lösung]
Sei $y^{(1)}, ..., y^{(n)}$ ein Fundamentalsystem von (H), $Y(x)$ sei definiert wie oben. Setze
\begin{align*}
\importantbox{y_s(x) := Y(x) \int Y(x)^{-1} b(x) \text{d}x \quad (x \in I).}
\end{align*}
Dann ist $y_s$ eine spezielle Lösung von (S) auf $I$.
\begin{align*}
W_k(x) := \det \left( y^{(1)}(x), ..., y^{(k-1)}(x), b(x), y^{(k+1)}(x), ..., y^{(n)}(x) \right)\quad (k=1,...,n)
\end{align*}
Dann gilt: $y_s(x) = \sum_{k=1}^n \left( \int \frac{W_k(x)}{W(x)} \text{d}x\right) y^{(k)}(x)$.
\end{satz}

\begin{beispiel}
Bestimme die allgemeine Lösung von
\begin{align*}
\tag{+}
y' = Ay + \begin{pmatrix} -\sin(x) \\ \cos(x) \end{pmatrix},
\end{align*}
wobei
\begin{align*}
A = \begin{pmatrix} 0 & -1 \\ 1 & 0 \end{pmatrix}.
\end{align*}
Bekannt: Fundamentalsystem der homogenen Gleichung $y' = Ay$:
\begin{align*}
y^{(1)}(x) = \begin{pmatrix} \cos(x) \\ \sin(x) \end{pmatrix}, y^{(2)}(x) = \begin{pmatrix} -\sin(x) \\ \cos(x) \end{pmatrix}.
\end{align*}
\begin{align*}
&W(x) = \left| \begin{array}{cc} \cos(x) & -\sin(x) \\ \sin(x) & \cos(x) \end{array} \right| = \cos^2(x) + \sin^2(x) = 1. \\
&W_1(x) = \left| \begin{array}{cc} -\sin(x) & -\sin(x) \\ \cos(x) & \cos(x) \end{array} \right| = 0. \\
&W_2(x) = \left| \begin{array}{cc} \cos(x) & -\sin(x) \\ \sin(x) & \cos(x) \end{array} \right| = 1. \\
&y_s(x) := \left( \int 1 \text{d}x \right) y^{(2)}(x) = xy^{(2)}(x) = \begin{pmatrix} -x \sin(x) \\ x \cos(x) \end{pmatrix} \text{ ist eine spezielle Lösung von (+).}
\end{align*}
Allgemeine Lösung von (+):
\begin{align*}
y(x) &= \underbrace{c_1 \begin{pmatrix} \cos(x) \\ \sin(x) \end{pmatrix} + c_2 \begin{pmatrix} -\sin(x) \\ \cos(x) \end{pmatrix}}_{\text{allg. Lsg. der hom. Glg.}} + \underbrace{\begin{pmatrix} -x \sin(x) \\ x \cos(x) \end{pmatrix}}_{\text{spez. Lsg.}} \\
&= \begin{pmatrix} c_1 \cos(x) - c_2 \sin(x) - x \sin(x) \\ c_1 \sin(x) + c_2 \cos(x)  + x \cos(x) \end{pmatrix}\quad(c_1, c_2 \in \MdR)
\end{align*}
Löse das
$\text{AwP}
\begin{cases}
y' = \begin{pmatrix} 0 & -1 \\ 1 & 0 \end{pmatrix}y + \begin{pmatrix} -\sin(x) \\ \cos(x) \end{pmatrix} \\
y(0) = \begin{pmatrix} 0 \\ 0 \end{pmatrix}
\end{cases}$. \\
Es gilt:
\begin{align*}
\begin{pmatrix}0 \\ 0\end{pmatrix} = y(0) = \begin{pmatrix} c_1 \cos(0) - c_2 \sin(0) - 0\cdot\sin(0) \\ c_1 \sin(0) + c_2 \cos(0)  + 0\cdot\cos(0) \end{pmatrix} = \begin{pmatrix}c_1 \\ c_2\end{pmatrix}.
\end{align*}
Also: $c_1 = c_2 = 0$, d.h.: \textbf{die} Lösung des AwP ist: $y(x) = \begin{pmatrix} -x \sin(x) \\ x \cos(x) \end{pmatrix}$.

\end{beispiel}



\chapter{Homogene lineare Systeme mit konstanten Koeffizienten}


In diesem Paragraphen sei $A \in \mathbb{R}^{n \times n}$ eine
konstante Matrix. \\
Wir betrachten das homogene System
\begin{align*}
\tag H y'=Ay
\end{align*}

\textbf{Ohne} Beweise geben wir ein "`Kochrezept"' an, wie man zu einem Fundamentalsystem von (H) kommt.

\textbf{Vorbereitungen}:
\index{charakteristisches Polynom}\index{Polynom!charakteristisches}
\begin{enumerate}
\item Es sei stets $p(\lambda) := \det(A - \lambda I)$ das \textbf{charakteristische Polynom}
von $A$ ($I$ = Einheitsmatrix). \\
Sei $\lambda_0 \in \MdC$ ein Eigenwert (EW) von $A$, dann
ist $p(\lambda_0) = 0$. Die Koeffizienten von $p$ sind reell, also ist $p(\overline{\lambda_0}) = 0$
und damit $\overline{\lambda_0}$ ein Eigenwert von $A$.
\item Für $\lambda_0 \in \mdc$ gilt:
\[\kernn(A-\lambda_0 I) \subseteq \kernn((A-\lambda_0 I)^2) \subseteq \kernn((A-\lambda_0 I)^3) \subseteq \ldots\]
\end{enumerate}

\textbf{Kochrezept}:
\begin{enumerate}
\item Bestimme die \textbf{verschiedenen} Eigenwerte $\lambda_1, \ldots, \lambda_r (r \le n)$
von $A$ und deren algebraische Vielfachheiten $k_1, \ldots, k_r$, also:
\[p(\lambda) = (-1)^n (\lambda - \lambda_1)^{k_1} (\lambda - \lambda_2)^{k_2} \cdots (\lambda - \lambda_r)^{k_r}\]
Ordne diese Eigenwerte wie folgt an: $\lambda_1, \ldots, \lambda_m \in \mdr, \lambda_{m+1}, \ldots, \lambda_r \in \mdc \setminus \mdr$.\\
Aus der Liste $\lambda_{m+1}, \ldots, \lambda_r$ entferne jedes $\lambda_j$ mit
$\Im(\lambda_j) < 0$. Es bleibt:
\[M := \{\lambda_1, \ldots, \lambda_m\} \cup \{\lambda_j : m + 1 \le j \le r, \Im(\lambda_j) > 0 \}\]

\item Zu $\lambda_j \in M$ bestimme eine Basis von $V_j := $ Kern$((A-\lambda_j I)^{k_j})$
wie folgt: Bestimme eine Basis von Kern$(A-\lambda_j I)$, ergänze diese Basis zu
einer Basis von Kern$((A-\lambda_j I)^2)$, usw.

\item Sei $\lambda_j \in M$ und $v$ ein Basisvektor von $V_j$. \\
\[y(x) := e^{\lambda_j x} (v+\frac{x}{1!} (A-\lambda_j I)v + \frac{x^2}{2!} (A-\lambda_j I)^2 v + \cdots + \frac{x^{k_j - 1}}{ (k_j - 1)! } (A - \lambda_j I)^{k_j - 1} v )\]
Oder kürzer:
\[\importantbox{y(x) := e^{\lambda_j x} \cdot \left ( \sum_{i=0}^{k_j-1} \frac{x^i}{i!} (A - \lambda_j I)^i \cdot v \right )}\]
\textbf{Fall 1}: $\lambda_j \in \mdr$.\\
Dann ist $y(x) \in \mdr^n \; \forall x \in \mdr$ und y ist eine Lösung von (H). \\
\textbf{Fall 2}: $\lambda_j \in \mdc \setminus \mdr$.\\
Zerlege $y$ komponentenweise in Real- und Imaginärteil:
\[y(x) := y^{(1)}(x) + i y^{(2)}(x)\]
mit $y^{(1)}(x),y^{(2)}(x)\in\mdr^n$. Dann sind $y^{(1)}, y^{(2)}$ linear unabhängige Lösungen von (H).

\item Führt man (3) für \textbf{jedes} $\lambda_j \in M$ und \textbf{jeden} Basisvektor von $V_j$ durch, so erhält man ein Fundamentalsystem von (H).
\end{enumerate}

\begin{definition}
\index{linear!Hülle}\index{Hülle!lineare}
$ [\ldots] $ bezeichne die \textbf{lineare Hülle}.
\end{definition}

\textbf{Beispiele}:

\begin{enumerate}
\item Bestimme ein Fundamentalsystem der Gleichung:
\begin{align*}
\tag{$\ast $} y' = Ay
\end{align*}
mit
\[A:=\begin{pmatrix} 1 & -4 \\ 1 & 1  \end{pmatrix}\]
Es gilt:
\[p(\lambda)=\det(A-\lambda I) = (\lambda - (1 + 2i))(\lambda-(1-2i))\]
\begin{align*}
\lambda_1 &= 1 + 2i &\lambda_2 &= 1-2i\\
k_1 &= 1 &k_2&=1\\
\end{align*}
Also ist $M := \{\lambda_1\}$. Aus $\kernn(A-\lambda_1 I) = \left[ \begin{pmatrix} 2i \\ 1 \end{pmatrix} \right]$
folgt:
\begin{align*}
y(x) &= e^{(1+2i)x} \begin{pmatrix} 2i \\ 1 \end{pmatrix} \\
&= e^x (\cos(2x) + i \sin(2x)) \begin{pmatrix} 2i \\ 1 \end{pmatrix}\\
&= e^x \begin{pmatrix} -2\sin(2x) \\ \cos(2x) \end{pmatrix}
+ ie^x \begin{pmatrix} 2\cos(2x) \\ \sin(2x) \end{pmatrix}
\end{align*}
Sei also:
\begin{align*}
y^{(1)}(x)&:=e^x \begin{pmatrix} -2\sin(2x) \\ \cos(2x) \end{pmatrix}&
y^{(2)}(x)&:=e^x \begin{pmatrix} 2\cos(2x) \\ \sin(2x) \end{pmatrix}
\end{align*}
Dann ist $y^{(1)}, y^{(2)}$ ein Fundamentalsystem von ($\ast$).

\item Bestimme ein Fundamentalsystem der Gleichung:
\begin{align*}
\tag{$\ast $} y' = Ay
\end{align*}
mit
\[A:=\begin{pmatrix} 0 & 1 & -1 \\ -2 & 3 & -1 \\ -1 & 1 & 1 \end{pmatrix}\]
Es gilt:
\[p(\lambda)=\det(A-\lambda I) = -(\lambda - 2)(\lambda - 1)^2\]
\begin{align*}
\lambda_1 &= 2 &\lambda_2&=1\\
k_1&=1 &k_2&=2
\end{align*}
Also ist $M := \{\lambda_1, \lambda_2\}$.\\
\boldmath $\lambda_1 = 2$\unboldmath: Aus $\kernn(A-2I) =
\left[ \begin{pmatrix} 0 \\ 1\\1 \end{pmatrix} \right]$ folgt:
\[y^{(1)}(x) := e^{2x}\begin{pmatrix} 0 \\ 1\\1 \end{pmatrix}\]
\boldmath $\lambda_2 = 1$\unboldmath: Aus $\kernn(A-I) =
\left[ \begin{pmatrix} 1 \\ 1\\0 \end{pmatrix} \right]
\subseteq \left[ \begin{pmatrix} 1\\1\\0 \end{pmatrix},
\begin{pmatrix} 0\\0\\1\end{pmatrix} \right] = $ Kern$((A -I)^2)$ folgt:
\begin{align*}
y^{(2)}(x) := e^x \begin{pmatrix} 1\\1\\0 \end{pmatrix} &&
y^{(3)}(x) := e^x\left( \begin{pmatrix} 0\\0\\1\end{pmatrix} + x(A-I) \begin{pmatrix} 0\\0\\1\end{pmatrix} \right) = e^x \begin{pmatrix} -x \\ -x \\ 1 \end{pmatrix}
\end{align*}
$y^{(1)}, y^{(2)}, y^{(3)}$ ist ein Fundamentalsystem von ($\ast$).

\item Sei $A$ wie in Beispiel (2). Löse das \[
\text{AwP}
\begin{cases}
y'=Ay\\
y(0) = \begin{pmatrix} 1 \\ 0 \\ 1 \end{pmatrix}
\end{cases}\]
Die allgemeine Lösung von $y' = Ay$ lautet:
\[ y(x) = c_1  e^{2x}\begin{pmatrix} 0 \\ 1\\1 \end{pmatrix}
+ c_2  e^x \begin{pmatrix} 1\\1\\0 \end{pmatrix}
+ c_3 e^x \begin{pmatrix} -x \\ -x \\ 1 \end{pmatrix}\quad c_1, c_2, c_3 \in \mdr \]
Es gilt:
\begin{align*}
\begin{pmatrix} 1 \\ 0 \\ 1 \end{pmatrix} \stackrel!= y(0)
= c_1  \begin{pmatrix} 0 \\ 1\\1 \end{pmatrix}
+ c_2   \begin{pmatrix} 1\\1\\0 \end{pmatrix}
+ c_3  \begin{pmatrix} 0 \\ 0 \\ 1 \end{pmatrix}
= \begin{pmatrix} c_2 \\ c_1+c_2\\c_1+c_3 \end{pmatrix}\\
\end{align*}
\begin{align*}
\implies c_1=-1 &&c_2 = 1 && c_3 = 2
\end{align*}
Lösung des AWPs:
\[y(x) = -e^{2x}\begin{pmatrix} 0 \\ 1\\1 \end{pmatrix}
+ e^x \begin{pmatrix} 1\\1\\0 \end{pmatrix}
+ 2e^x \begin{pmatrix} -x \\ -x \\ 1 \end{pmatrix}\]

\item Bestimme die allgemeine Lösung von
\begin{align*}
\tag{$\ast $} y' = Ay + \begin{pmatrix} e^x \\ e^x \end{pmatrix}
\end{align*}
Mit
\[A:=\begin{pmatrix} 1 & 0 \\ 0 & -1 \end{pmatrix}\]
Bestimme dazu zunächst die allgemeine Lösung von $y' = Ay$. Es gilt:
\[p(\lambda)=\det(A-\lambda I) = (1-\lambda)(1+\lambda)\]
\begin{align*}
\lambda_1 &= 1 &\lambda_2 &= -1\\
k_1&=1&k_2&=1
\end{align*}
Da $\kernn(A-I) = \left[ \begin{pmatrix} 1 \\ 0 \end{pmatrix} \right]$ und
$\kernn(A+I) = \left[ \begin{pmatrix} 0 \\ 1 \end{pmatrix} \right]$ ist, ist
\begin{align*}
y^{(1)}(x) &= e^x \begin{pmatrix} 1 \\ 0 \end{pmatrix}
&y^{(2)}(x) &= e^{-x} \begin{pmatrix} 0 \\ 1 \end{pmatrix}
\end{align*}
ein Fundamentalsystem von $y' = Ay$.\\
Sei nun $Y(x) := \begin{pmatrix} e^x & 0 \\ 0 & e^{-x} \end{pmatrix}$ \\
Dann ist
\begin{align*}
y_s(x) &= Y(x) \int Y(x)^{-1} \begin{pmatrix} e^x \\ e^x \end{pmatrix} \text{ d}x\\
&= Y(x) \int \begin{pmatrix} e^{-x} & 0 \\ 0 & e^x \end{pmatrix} \begin{pmatrix} e^x \\ e^x \end{pmatrix} \text{ d}x\\
&= Y(x) \int \begin{pmatrix} 1 \\ e^{2x} \end{pmatrix}\text{ d}x\\
&=  \begin{pmatrix} e^x & 0 \\ 0 & e^{-x} \end{pmatrix} \begin{pmatrix} x \\ \frac12e^{2x} \end{pmatrix} = \begin{pmatrix} xe^x \\ \frac12e^x \end{pmatrix}
\end{align*}
eine spezielle Lösung von ($\ast$).

Die allgemeine Lösung von ($\ast$) lautet also:
\[y(x) = c_1 e^x \begin{pmatrix} 1 \\ 0 \end{pmatrix}
+ c_2 e^{-x} \begin{pmatrix} 0 \\ 1 \end{pmatrix}
+ \begin{pmatrix} xe^x \\ \frac12e^x \end{pmatrix}\quad c_1, c_2 \in \mdr\]
\end{enumerate}


\chapter{Lineare Differentialgleichungen n-ter Ordnung}
\index{linear!Differentialgleichung n-ter Ordnung}\index{Differentialgleichung!lineare (n-ter Ordnung)}
\index{homogen!Differentialgleichung n-ter Ordnung}\index{Differentialgleichung!homogene (n-ter Ordnung)}
\index{inhomogen!Differentialgleichung n-ter Ordnung}\index{Differentialgleichung!inhomogene (n-ter Ordnung)}
\index{Anfangswertproblem}
In diesem Paragraphen sei $n\in\mdn, I\subseteq\mdr$ ein Intervall und $a_0,\ldots,a_{n-1},b:I\to\mdr$
stetig. Für $y\in C^n(I,\mdr)$ setze $Ly:=y^{(n)}+a_{n-1}(x)y^{(n-1)}+\cdots+a_0(x)y$.
Die Differenzialgleichung
\begin{align*}
\tag D Ly=b
\end{align*}
heißt eine \textbf{lineare Dgl $n$-ter Ordnung}. Sie heißt \textbf{homogen}, falls $b\equiv 0$,
anderenfalls \textbf{inhomogen}.\\
Setze $b_0(x):=(0,\ldots,0,b(x))^T (\in\mdr^n)$ und
\begin{align*}
A(x):=
\begin{pmatrix}
0&1&0&\cdots&0\\
\vdots&\ddots&\ddots&\ddots&\vdots\\
\vdots&&\ddots&\ddots&0\\
0&\ldots&\ldots&0&1\\
-a_0(x)&\ldots&\ldots&\ldots&-a_{n-1}(x)
\end{pmatrix}
\end{align*}
Damit erhalten wir das System:
\begin{align*}
\tag S z'=A(x)z+b_0(x)
\end{align*}

\begin{satz}[Lösungen]
\begin{enumerate}
\item Ist $y:I\to\mdr$ eine Lösung von (D) auf $I$, so ist $z:=(y,y',\ldots,y^{(n-1)})$
eine Lösung von (S) auf $I$.
\item Ist $z:=(z_1,\ldots,z_n)$ eine Lösung von (S) auf $I$, so ist $y:=z_1$ eine Lösung von (D) auf $I$.
\end{enumerate}
\end{satz}

\begin{beweis}
Nachrechnen!
\end{beweis}

Wir betrachten auch noch die zu (D) gehörende \textbf{homogene} Gleichung
\begin{align*}
\tag H Ly=0
\end{align*}
Sind $y_0,\ldots,y_{n-1}\in\mdr$ und $x_0\in I$, so heißt
\begin{align*}
\tag A \begin{cases}
Ly=b\\
y(x_0)=y_0,\ldots,y^{(n-1)}(x_0)=y_{n-1}
\end{cases}
\end{align*}
ein \textbf{Anfangswertproblem} (AwP).

Die folgenden Sätze 24.2 und 24.3 folgen aus 24.1 und den Sätzen aus §21.

\begin{satz}[Lösungsmenge als Vektorraum]
\begin{enumerate}
\item Das AwP (A) hat auf $I$ genau eine Lösung.
\item (D) hat Lösungen auf $I$.
\item Sei $y_s$ eine spezielle Lösung von (D) auf $I$. Für $y:I\to\mdr$ gilt:\\
$y$ ist eine Lsg von (D) auf $I$, genau dann wenn eine Lösung $y_h$ von (H) existiert:
\[y=y_h+y_s\]
\item Ist $J\subseteq I$ ein Intervall, $\hat y:J\to\mdr$ eine Lsg von (D) auf $J$,
so existiert eine Lsg $y:I\to\mdr$ mit $\hat y=y|_J$.
\item Sei $\mathbb{L}$ die Menge aller Lösungen von (H) auf $I$. Dann ist $\mathbb{L}$
ein reeller Vektorraum und $\dim\mathbb{L}=n$.\\
Für $y_1,\ldots,y_k\in\mathbb{L}$ sind äquivalent:
\begin{enumerate}
\item $y_1,\ldots,y_k$ sind linear unabhängig in $\mathbb{L}$.
\item Für alle $x\in I$ sind die Vektoren $(y_j(x),y_j'(x),\ldots,y_j^{(n-1)}(x)) (j=1,\ldots,k)$
linear unabhängig im $\mdr^n$.
\item Es existiert ein $\xi\in I$ sodass die Vektoren $(y_j(\xi),\ldots,y_j^{(n-1)}(\xi)) (j=1,\ldots,k)$
linear unabhängig sind im $\mdr^n$.
\end{enumerate}
\end{enumerate}
\end{satz}

\begin{definition}
\index{Wronskideterminante}
\index{Fundamental-!System}
Seien $y_1,\ldots,y_n\in\mathbb{L}$.
\begin{align*}
W(x):= \det\begin{pmatrix}
y_1(x)&\cdots&y_n(x)\\
\vdots& &\vdots\\
y_1^{(n-1)}(x)&\cdots&y_n^{(n-1)}(x)
\end{pmatrix}
\end{align*}
heißt \textbf{Wronskideterminante}. Sind $y_1,\ldots,y_n$ linear unabhängig in $\mathbb{L}$,
so heißt $y_1,\ldots,y_n$ ein \textbf{Fundamentalsystem} (FS) von (H). I.d. Fall
lautet die allgemeine Lösung von (H):
\[y=c_1y_1+\cdots+c_ny_n \quad (c_1,\ldots,c_n\in\mdr)\]
Aus 24.2 folgt für $y_1,\ldots,y_n\in\mathbb{L}$:\\
$y_1,\ldots,y_n\in\mathbb{L}$ ist genau dann ein FS von (H), wenn gilt:
\begin{align*}
\forall x\in I: W(x)\ne 0 \iff \exists\xi\in I:W(\xi)\ne 0
\end{align*}
\end{definition}

\begin{satz}[Spezielle Lösung]
Sei $y_1,\ldots,y_n$ ein FS von (H) und $W$ wie oben. Für $k=1,\ldots,n$ sei
$W_k(x)$ die Determinante die entsteht, wenn man die $k$-te Spalte von $W(x)$
ersetzt durch $(0,\ldots,0,b(x))^T$. Setze
\[y_s(x):=\sum_{k=1}^n\left(y_k(x)\cdot \int \frac{W_k(x)}{W(x)}\text{ d}x\right)\]
Dann ist $y_s$ eine spezielle Lösung von (D).
\end{satz}

\begin{beispiel}[Spezialfall $n=2$]
Die homogene Gleichung hat die Form
\begin{align*}
\tag H y''+a_1(x)y'+a_0(x)y=0
\end{align*}
Sei $y_1$ eine Lsg von (H) mit $y_1\ne 0\forall x\in I$. Sei $z\not\equiv$ eine Lsg von
\[z'=-\left(a_1(x)+\frac{2y_1'(x)}{y_1(x)}\right)z, \text{ \quad so ist}\]
\[y_2(x):=y_1(x)\cdot\int z(x)\text{ d}x\]
eine weitere Lsg von (H) und $y_1,y_2$ ist ein FS von (H).
\end{beispiel}

\begin{beweis}
Nachrechnen: $y_2$ ist Lsg von (H).\\
Aus $y_2'=y_1'\cdot\int z(x)\text{ d}x+y_1z(x)$ folgt:
\begin{align*}
W(x)&=\det\begin{pmatrix}
y_1(x)&y_1(x)\cdot\int z(x)\text{ d}x\\
y_1'(x)&y_1'(x)\int z(x)\text{ d}x+y_1z(x)
\end{pmatrix}\\
&= y_1y_1'\cdot\int z(x)\text{ d}x+y_1^2z(x)-y_1y_1'\cdot\int z(x)\text{ d}x\\
&= y_1^2z(x)
\end{align*}
Da $z\not\equiv 0$ ist, existiert ein $\xi\in I$ mit $z(\xi)\ne 0$, also $W(\xi)\ne 0$.
D.h. $y_1,y_2$ sind linear unabhängig in $\mathbb{L}$.
\end{beweis}

\begin{beispiele}
\item Bestimme die allg. Lösung der Gleichung (mit $I=(1,\infty)$)
\begin{align*}
\tag{$*$} y''+\frac{2x}{1-x^2}y'-\frac{2}{1-x^2}y=0
\end{align*}
Offensichtlich ist $y_1(x)=x$ eine Lsg von ($*$) auf $I$. Die Gleichung erster Ornung lautet:
\begin{align*}
\tag{$**$} z'=-\left(\frac{2x}{1-x^2}+\frac 2x\right)z=\frac 2{x(x^2-1)} z
\end{align*}
Es ist $\int\frac2{x(x^2-1)}\text{ d}x=\log(1-\frac1{x^2})$, daraus ergibt sich die allgemeine
Lösung von ($**$):
\[z(x)=ce^{\log(1-\frac1{x^2})}=c(1-\frac1{x^2})\quad (c\in\mdr)\]
Sei also:
\[y_2(x):=y_1(x)\cdot\int 1-\frac1{x^2}\text{ d}x=1+x^2\]
Damit ist $y_1,y_2$ ein Fundamentalsystem von ($*$) und die allgemeine Lösung lautet:
\[y(x)=c_1x+c_2(1+x^2)\quad (c_1,c_2\in\mdr)\]
\item Bestimme die allg. Lösung der Gleichung
\begin{align*}
\tag{+} y''+\frac{2x}{1-x^2}y'-\frac{2}{1-x^2}y=x^2-1
\end{align*}
Die allg. Lösung der homogenen Gleichung lautet
\[y(x)=c_1x+c_2(1+x^2)\]
Es ist also $y_1(x)=x$ und $y_2(x)=1+x^2$. Damit gilt:
\begin{align*}
W(x)&=\det\begin{pmatrix}
x&1+x^2\\
1&2x
\end{pmatrix}=2x^2-(1+x^2)=x^2-1\\
W_1(x)&=\det\begin{pmatrix}
0&1+x^2\\
x^2-1&2x
\end{pmatrix}=-(1+x^2)(x^2-1)\\
W_2(x)&=\det\begin{pmatrix}
x&0\\
1&x^2-1
\end{pmatrix}=x^3-x
\end{align*}
Es folgt:
\begin{align*}
\frac{W_1(x)}{W(x)}=-1-x^2 &&\frac{W_2(x)}{W(x)}=x
\end{align*}
Daraus ergibt sich nun eine spezielle Lösung von (+):
\[y_s(x)=y_1(x)\cdot\int(-1-x^2)\text{ d}x+y_2(x)\cdot\int x\text{ d}x=\frac16 x^4-\frac12 x^2\]
Die allgemeine Lösung von (+) lautet:
\[y(x)=c_1x+c_2(1+x^2)+\frac16 x^4-\frac12 x^2\quad (c_1,c_2\in\mdr)\]
\item Löse das
\begin{align*}
\text{AwP}
\begin{cases}
y''+\frac{2x}{1-x^2}y'-\frac{2}{1-x^2}y=x^2-1\\
y(0)=0, y'(0)=1
\end{cases}
\end{align*}
Die allgemeine Lösung der Dgl lautet:
\[y(x)=c_1x+c_2(1+x^2)+\frac16 x^4-\frac12 x^2\quad (c_1,c_2\in\mdr)\]
Also ist:
\[y'(x)=c_1+2c_2x+\frac23x^3-x\]
Außerdem gilt:
\begin{align*}
0\stackrel!= y(0)=c_2&&1\stackrel!=y'(0)=c_1
\end{align*}
Daraus folgt für die Lösung des AwPs:
\[y(x)=x+\frac16x^4-\frac12x^2\]
\end{beispiele}

\chapter{Lineare Differentialgleichungen n-ter Ordnung mit konstanten Koeffizienten}
In diesem Paragraphen sei $n\in\mdn, a_0,\ldots,a_{n-1}\in\mdr, I\subseteq\mdr$ ein Intervall
und $b:I\to\mdr$ stetig.

\index{homogen!Differentialgleichung}\index{Differentialgleichung!homogene}
\index{charakteristisch!Polynom}\index{Polynom!charakteristisches}
Wir betrachten zunächst die \textbf{homogene Gleichung}
\begin{align*}
\tag H y^{(n)}+a_{n-1}y^{(n-1)}+\cdots+a_0y=0
\end{align*}
und geben \textbf{ohne} Beweis ein "`Kochrezept"' an, wie man zu einem FS von (H) kommt.
\[p(\lambda):=\lambda^n+a_{n-1}\lambda^{n-1}+\cdots+a_1\lambda+a_0\]
heißt das \textbf{charakteristische Polynom} von (H).

\textbf{Übung:}\\
Ist
\[A:=\begin{pmatrix}
0&1&0&\cdots&0\\
\vdots&\ddots&\ddots&\ddots&\vdots\\
\vdots&&\ddots&\ddots&0\\
0&\cdots&\cdots&0&1\\
-a_0&\cdots&\cdots&\cdots&-a_{n-1}
\end{pmatrix}\]
so ist $\det(\lambda I-A)=p(\lambda)$.

\textbf{Kochrezept:}
\begin{enumerate}
\item Bestimme die verschiedenen Nullstellen $\lambda_1,\ldots,\lambda_r (r\le n)$ von $p$
und deren Vielfachheiten $k_1,\ldots,k_r$, also:
\[p(\lambda)=(\lambda-\lambda_1)^{k_1}\cdots(\lambda-\lambda_r)^{k_r}\]
Es seien $\lambda_1,\ldots,\lambda_m\in\mdr$ und $\lambda_{m+1},\ldots,\lambda_r\in\mdc\setminus\mdr$.
\[M:=\Set{\lambda_1,\ldots,\lambda_m}\cup\Set{\lambda_j | m+1\le j\le r,\Im(\lambda_j)>0}\]
\item Sei $\lambda_j\in M$.\\
\textbf{Fall 1:} $\lambda_j\in\mdr$\\
Dann sind
\[e^{\lambda_jx},xe^{\lambda_jx},\ldots,x^{k_j-1}e^{\lambda_jx}\]
$k_j$ linear unabhängige Lösungen von (H).\\
\textbf{Fall 2:} $\lambda_j\in\mdc\setminus\mdr$, etwa $\lambda_j=\alpha+i\beta$ $(\alpha,\beta\in\mdr,\beta>0)$\\
Dann sind
\begin{align*}
e^{\alpha x}\cos(\beta x), xe^{\alpha x}\cos(\beta x),\ldots,x^{k_j-1}e^{\alpha x}\cos(\beta x)\\
e^{\alpha x}\sin(\beta x), xe^{\alpha x}\sin(\beta x),\ldots,x^{k_j-1}e^{\alpha x}\sin(\beta x)
\end{align*}
$2k_j$ linear unabhängige Lösungen von (H).
\item Führt man (2) für jedes $\lambda_j\in M$ durch, so erhält man ein FS von (H).
\end{enumerate}

\begin{beispiele}
\item Bestimme die allg. Lösung der Gleichung
\begin{align*}
\tag{$*$} y^{(6)}-6y^{(5)}+9y^{(4)}=0
\end{align*}
Es gilt:
\[p(\lambda)=\lambda^6-6\lambda^5+9\lambda^4=\lambda^4(\lambda^2-6\lambda+9)=\lambda^4(\lambda-3)^2\]
Sei also:
\begin{align*}
\lambda_1&:=0&\lambda_2&:=3\\
k_1&:=4&k_2&:=2
\end{align*}
Ein FS von ($*$) lautet: $1,x,x^2,x^3,e^{3x},xe^{3x}$. Das bedeutet für die allgemeine Lösung von ($*$):
\[y(x)=c_1+c_2x+c_3x^2+c_4x^3+c_5e^{3x}+c_6xe^{3x}\quad(c_1,\ldots,c_6\in\mdr)\]
\item Bestimme die allgemeine Lösung der Gleichung:
\begin{align*}
\tag{$*$} y'''-2y''+y'-2y=0
\end{align*}
Es gilt:
\[p(\lambda)=\lambda^3-2\lambda^2+\lambda-2=(\lambda^2+1)(\lambda-2)=(\lambda-2)(\lambda+i)(\lambda-i)\]
Sei also:
\begin{align*}
\lambda_1&:=2&\lambda_2&:=i&\lambda_3&:=-i\\
k_1&:=1&k_2&:=1&k_3&:=1
\end{align*}
Dann ist $M:=\{2,i\}$ und ein FS von ($*$) lautet: $e^{2x},\cos(x),\sin(x)$. Das bedeutet für
die allgemeine Lösung von ($*$):
\[y(x)=c_1e^{2x}+c_2\cos(x)+c_3\sin(x)\quad (c_1,c_2,c_3\in\mdr)\]
\item Löse das
\begin{align*}
\text{AwP}
\begin{cases}
y'''-2y''+y'-2y=0\\
y(0)=0,y'(0)=1,y''(0)=0
\end{cases}
\end{align*}
Die allgemeine Lösung der Dgl lautet:
\[y(x)=c_1e^{2x}+c_2\cos(x)+c_3\sin(x)\]
Es ist:
\begin{align*}
y'(x)&=2c_1e^{2x}-c_2\sin(x)+c_3\cos(x)\\
y''(x)&=4c_1e^{2x}-c_2\cos(x)-c_3\sin(x)
\end{align*}
Außerdem gilt:
\begin{align*}
0&\stackrel!=y(0)=c_1+c_2&1&\stackrel!=y'(0)=2c_1+c_3&0&\stackrel!=4c_1-c_2
\end{align*}
Daraus folgt:
\begin{align*}
c_1&=0&c_2&=0&c_3&=1
\end{align*}
Also lautet die Lösung des AwPs:
\[y(x)=\sin(x)\]
\end{beispiele}

\index{inhomogen!Differentialgleichung}\index{Differentialgleichung!inhomogene}
Wir betrachten auch noch die \textbf{inhomogene Gleichung}
\begin{align*}
\tag{IH} y^{(n)}+a_{n-1}y^{(n-1)}+\cdots+a_0y=b(x)
\end{align*}

\begin{definition}
\index{nullfache Nullstelle}\index{Nullstelle!nullfache}
$\mu\in\mdc$ heißt eine \textbf{nullfache Nullstelle} von $p$, genau dann wenn
$p(\mu)\ne 0$ ist.
\end{definition}

\textbf{Regel} (ohne Beweis):\\
Seien $\alpha,\beta\in\mdr,m,q\in\mdn_0$ und $b$ von der Form:
\begin{align*}
&b(x)=(b_0+b_1x+\cdots+b_mx^m)e^{\alpha x}\cos(\beta x)\quad\text{, oder}\\
&b(x)=(b_0+b_1x+\cdots+b_mx^m)e^{\alpha x}\sin(\beta x)
\end{align*}
Ist $\alpha+\beta i$ eine $q$-fache Nullstelle von $p$, so gibt es eine spezielle Lösung
$y_s$ von (IH) der Form:
\[y_s(x)=x^qe^{\alpha x}\left[(A_0+A_1x+\cdots+A_mx^m)\cos(\beta x)+(B_0+B_1x+\cdots+B_mx^m)\sin(\beta x)\right]\]

\begin{beispiel}
Bestimme die allgemeine Lösung der Gleichung
\begin{align*}
y'''-y'=x+1\tag{$*$}
\end{align*}
\begin{enumerate}
\item Bestimme die allgemeine Lösung der homogenen Gleichung
\begin{align*}
y'''-y'=0\tag{$**$}
\end{align*}
Es gilt:
\[p(\lambda)=\lambda^3-\lambda=\lambda(\lambda^2-1)=\lambda(\lambda+1)(\lambda-1)\]
Also ist ein FS von ($**$): $1,e^x,e^{-x}$. Damit lautet die allgemeine Lösung der homogenen
Gleichung:
\[y_h(x)=c_1+c_2e^x+c_3e^{-x}\quad(c_1,c_2,c_3\in\mdr)\]
\item Bestimme eine allgemeine Lösung der inhomogenen Gleichung ($*$).\\
Es ist $m=1,\alpha=\beta=0,q=1$. Ansatz:
\begin{align*}
&y_s(x)=x(A_0+A_1x)=A_0x+A_1x^2\\
&y_s'(x)=A_0+2A_1x\\
&y_s''(x)=2A_1\\
&y_s'''(x)=0
\end{align*}
Mit Einsetzen in ($*$) folgt:
\[0-(A_1+2A_1x)=x+1\]
Also ist:
\begin{align*}
A_0=-1&&A_1=-\frac12
\end{align*}
D.h. eine spezielle Lösung von (IH) lautet:
\[y_s(x)=-x-\frac12 x^2\]
\end{enumerate}
Damit lautet die allgemeine Lösung von (IH):
\[y(x)=c_1+c_2e^x+c_3e^{-x}-x-\frac12 x^2\quad(c_1,c_2,c_3\in\mdr)\]
\end{beispiel}

\appendix
\chapter{Satz um Satz (hüpft der Has)}
\listtheorems{satz,wichtigedefinition}

\renewcommand{\indexname}{Stichwortverzeichnis}
\addcontentsline{toc}{chapter}{Stichwortverzeichnis}
\printindex

\chapter{Credits für Analysis II}
Abgetippt haben die folgenden Paragraphen:\\
\textbf{§ 1: Der Raum $\MdR^n$}: Wenzel Jakob, Joachim Breitner\\
\textbf{§ 2: Konvergenz im $\MdR^n$}: Joachim Breitner und Wenzel Jakob\\
\textbf{§ 3: Grenzwerte bei Funktionen, Stetigkeit}: Wenzel Jakob, Pascal Maillard\\
\textbf{§ 4: Partielle Ableitungen}: Joachim Breitner und Wenzel Jakob\\
\textbf{§ 5: Differentiation}: Wenzel Jakob, Pascal Maillard, Jonathan Picht\\
\textbf{§ 6: Differenzierbarkeitseigenschaften reellwertiger Funktionen}: Jonathan Picht, Pascal Maillard, Wenzel Jakob\\
\textbf{§ 7: Quadratische Formen}: Wenzel Jakob\\
\textbf{§ 8: Extremwerte}: Wenzel Jakob\\
\textbf{§ 9: Der Umkehrsatz}: Wenzel Jakob und Joachim Breitner\\
\textbf{§ 10: Implizit definierte Funktionen}: Wenzel Jakob\\
\textbf{§ 11: Extremwerte unter Nebenbedingungen}: Pascal Maillard\\
\textbf{§ 12: Wege im $\MdR^n$}: Joachim Breitner, Wenzel Jakob und Pascal Maillard\\
\textbf{§ 13: Wegintegrale}: Pascal Maillard und Joachim Breitner\\
\textbf{§ 14: Stammfunktionen}: Joachim Breitner und Ines Türk\\
\textbf{§ 15: Vorgriff auf Analysis III}: Rebecca Schwerdt\\
\textbf{§ 16: Folgen, Reihen und Potenzreihen in $\MdC$}: Rebecca Schwerdt\\
\textbf{§ 17: Normierte Räume, Banachräume, Fixpunktsatz}: Rebecca Schwerdt\\
\textbf{§ 18: Differentialgleichungen: Grundbegriffe}: Rebecca Schwerdt\\
\textbf{§ 19: Lineare Differentialgleichungen 1. Ordnung}: Rebecca Schwerdt\\
\textbf{§ 20: Differentialgleichungen mit getrennten Veränderlichen}: Rebecca Schwerdt\\
\textbf{§ 21: Systeme von Differentialgleichungen 1. Ordnung}: Peter Pan\\
\textbf{§ 22: Lineare Systeme}: Rebecca Schwerdt, Peter Pan\\
\textbf{§ 23: Homogene lineare Systeme mit konstanten Koeffizienten}\\
\textbf{§ 24: Lineare Differentialgleichungen n-ter Ordnung}: Rebecca Schwerdt\\
\textbf{§ 25: Lineare Differentialgleichungen n-ter Ordnung mit konstanten Koeffizienten}: Rebecca Schwerdt\\

\end{document}
