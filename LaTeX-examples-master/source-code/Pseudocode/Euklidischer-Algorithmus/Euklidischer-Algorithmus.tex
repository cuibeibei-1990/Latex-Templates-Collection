\documentclass{article}
\usepackage[pdftex,active,tightpage]{preview}
\setlength\PreviewBorder{2mm}

\usepackage[utf8]{inputenc} % this is needed for umlauts
\usepackage[ngerman]{babel} % this is needed for umlauts
\usepackage[T1]{fontenc}    % this is needed for correct output of umlauts in pdf
\usepackage{amssymb,amsmath,amsfonts} % nice math rendering
\usepackage{braket} % needed for \Set
\usepackage{algorithm,algpseudocode}

\usepackage{tikz}
\usetikzlibrary{decorations.pathreplacing,calc}
\newcommand{\tikzmark}[1]{\tikz[overlay,remember picture] \node (#1) {};}
\newcommand*{\AddNote}[4]{%
    \begin{tikzpicture}[overlay, remember picture]
        \draw [decoration={brace,amplitude=0.5em},decorate,very thick]
            ($(#3)!(#1.north)!($(#3)-(0,1)$)$) --
            ($(#3)!(#2.south)!($(#3)-(0,1)$)$)
                node [align=center, text width=2.5cm, pos=0.5, anchor=west] {#4};
    \end{tikzpicture}
}%

\begin{document}
\begin{preview}
    \begin{algorithm}[H]
        \begin{algorithmic}
            \Require $Z \in \mathbb{R}_{\geq 0}, b \in \mathbb{N}_{\geq 2}$
            \State $p\gets 0$\tikzmark{top}
            \While{$b^p > Z$}\tikzmark{right}
                \State $p\gets p+1$
            \EndWhile
            \State $i\gets p-1$\tikzmark{bottom}
            \\
            \While{$Z \neq 0$}\tikzmark{top2}
                \State $y_i\gets Z\,\mbox{div}\, b^i$
                \State $R\gets Z\,\mbox{mod}\, b^i$
                \State $Z\gets R$
                \State $i\gets i-1$
            \EndWhile\tikzmark{bottom2}
            \\
            \State \textbf{Result:} $y_{p-1} y_{p-2} \dots y_0, y_{-1} \dots y_{i+1}$
        \end{algorithmic}
    \caption{Euclidean algorithm for changing base from base 10}
    \AddNote{top}{bottom}{right}{calclulate $p$ such that: $b^p \leq Z < b^{p+1}$}
    \AddNote{top2}{bottom2}{right}{Calculate one digit in every loop}
    \label{alg:euclidBaseTransformation}
    \end{algorithm}
\end{preview}
\end{document}
