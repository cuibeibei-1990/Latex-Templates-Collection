\documentclass{article}
\usepackage[pdftex,active,tightpage]{preview}
\setlength\PreviewBorder{2mm}
\usepackage{tikz}
\usepackage{tkz-euclide}
\usetkzobj{all}
\usetikzlibrary{shapes, calc, decorations}
\usepackage{amsmath,amssymb}
\usepackage{helvet}
\usepackage[eulergreek]{sansmath}

\begin{document}
\begin{preview}
\begin{tikzpicture}[very thick,font=\sansmath\sffamily]
    \tkzDefPoint(0,1){A}
    \tkzDefPoint(4,0){B}
    \tkzDefPoint(3,6){C}

    \tkzLabelPoints[below left,font=\sansmath\sffamily](A)
    \tkzLabelPoints[below right,font=\sansmath\sffamily](B)
    \tkzLabelPoints[above,font=\sansmath\sffamily](C)

    % Draw polygon
    \tkzDrawPolygon[rounded corners=0.1mm,thick,fill=gray!10](A,B,C)

    % Get mid points
    \tkzDefMidPoint(A,B) \tkzGetPoint{F}
    \tkzDefMidPoint(A,C) \tkzGetPoint{E}
    \tkzDefMidPoint(B,C) \tkzGetPoint{D}

    \tkzDefCircle[circum,/tikz/overlay](D,E,F)\tkzGetPoint{Icircle}\tkzGetLength{rIN}
    \tkzDrawCircle[R,color=red, very thick](Icircle,\rIN pt)

    % Get 'Hoehe'
    \tkzInterLC[/tikz/overlay](B,C)(A,B)\tkzGetPoints{hah1}{hah2}
    \tkzDefMidPoint(hah1,hah2) \tkzGetPoint{G}

    \tkzInterLC[/tikz/overlay](A,B)(C,A)\tkzGetPoints{hch1}{hch2}
    \tkzDefMidPoint(hch1,hch2) \tkzGetPoint{H}

    \tkzInterLC[/tikz/overlay](A,C)(B,A)\tkzGetPoints{hbh1}{hbh2}
    \tkzDefMidPoint(hbh1,hbh2) \tkzGetPoint{I}

    % 'Hoehenschnittpunkt'
    \tkzInterLL(A,G)(C,H)\tkzGetPoint{S}

    % More points
    \tkzInterLC[/tikz/overlay,R](A,G)(Icircle,\rIN pt)\tkzGetPoints{G}{J}
    \tkzInterLC[/tikz/overlay,R](B,I)(Icircle,\rIN pt)\tkzGetPoints{L}{I}
    \tkzInterLC[/tikz/overlay,R](C,H)(Icircle,\rIN pt)\tkzGetPoints{H}{K}

    % Draw polygon
    \tkzDefCircle[circum,/tikz/overlay](A,B,C) \tkzGetPoint{Ocircle}
    \tkzDrawCircle[color=blue](Ocircle,A)
    \tkzDrawPolygon[rounded corners=0.1mm,thick](A,B,C)
    \tkzCalcLength(Ocircle,A) \tkzGetLength{dOcircle}

    % TODO: define shiftedOcircOrigin which is
    % Ocircle, but shifted by dOcircle (x-axis)
    \tkzDefShiftPoint[Ocircle](\dOcircle pt,0){C'}
    \tkzDrawSegments[dashed](Ocircle,C')
    \tkzLabelSegment[below](Ocircle,C'){$r_1$}

    % TODO: define shiftedIcircOrigin which is
    % Icircle, but shifted by rIN (x-axis)
    \tkzDefShiftPoint[Icircle](\rIN pt,0){C''}
    \tkzDrawSegments[dashed](Icircle,C'')
    \tkzLabelSegment[below](Icircle,C''){$r_2$}

    \tkzDrawPoints[size=2,fill=black](D,E,F)
    \tkzDrawPoints[size=2,fill=blue,color=blue](Ocircle)
    \tkzDrawPoints[size=2,fill=red,color=red](Icircle)
\end{tikzpicture}
\end{preview}
\end{document}
