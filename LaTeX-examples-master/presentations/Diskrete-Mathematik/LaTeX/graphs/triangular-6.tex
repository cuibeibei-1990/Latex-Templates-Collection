\documentclass[varwidth=true, border=2pt]{standalone}
\usepackage{ifthen}
\usepackage{tikz}
\usetikzlibrary{calc}

\begin{document}
\tikzstyle{vertex}=[draw,red,fill=red,circle,
minimum size=10pt,inner sep=0pt]
\tikzstyle{edge}=[red, very thick]
\begin{tikzpicture}
    \newcommand{\n}{6}
    \foreach \y in {0, ..., \n}{
        \pgfmathsetmacro{\loopend}{{2*\n-\y}}
        \pgfmathsetmacro{\second}{{\y+2}}
        \foreach \x in {\y, \second,..., \loopend}{
            \ifthenelse{\n=\y}{\breakforeach}{}
            \node (n-\x\y)[vertex] at (\x,\y) {};

            \ifthenelse{\y=0}{}{\draw[edge] (\x,\y) -- (\x+1,\y-1);}
            \pgfmathtruncatemacro\X{\x}
            \ifthenelse{\X<\loopend}{\draw[edge] (\x,\y) -- (\x+2,\y);}{}
            \ifthenelse{\X=\loopend}{}{\draw[edge] (\x,\y) -- (\x+1,\y+1);}

        }
    }
\end{tikzpicture}
\end{document}
