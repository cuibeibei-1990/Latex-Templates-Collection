%!TEX root = interventions.tex
\section{Interventions}
\subsection{Definition}
\begin{frame}{Interventionen}
    \begin{block}{Interventionsverteilung}
        Sei $\mathbb{P}^\mathbf{X}$ die zu einer SEM
        $\mathcal{S} := (\mathcal{S}, \mathbb{P}^N)$ gehörende Verteilung. \onslide<2->{Dann
                kann eine (oder mehr) Strukturgleichungen aus $\mathcal{S}$ ersetzt
                werden ohne einen Zyklus im Graphen zu erzeugen.} \onslide<3->{Die Verteilung des
        neuen SEM $\tilde{\mathcal{S}}$ heißt dann
        \textit{Interventionsverteilung}.}

        \onslide<4->{Bei den Variablen, deren Strukturgleichungen ersetzt wurden, sagt man,
        wurde \textit{interveniert}.}

        \onslide<5->{Die neue Verteilung wird mit
        \[\mathbb{P}_{\tilde{\mathcal{S}}}^{\mathbf{X}} = \mathbb{P}_{\mathcal{S}, do(X_j:=\tilde{f}(\tilde{\mathbf{PA}}_j, \tilde{N}_j))}^{\mathbf{X}}\]
        beschrieben.}

        \onslide<6->{Die Menge der Rauschvariablen in $\mathcal{S}$ beinhaltet nun einige
        \enquote{neue} und einige \enquote{alte} $N$'s. $\mathcal{S}$ muss
        paarweise unabhängig sein.}
    \end{block}
\end{frame}

\begin{frame}{Nieren-Beispiel}
    \begin{table}
        \begin{tabular}{lrr}
        \toprule
        ~      & \multicolumn{2}{c}{\textbf{Behandlungserfolg}}  \\
        \cmidrule{2-3}
        ~                   & \multicolumn{1}{c}{\textbf{A}} & \multicolumn{1}{c}{\textbf{B}} \\ \midrule
        Kleine Nierensteine & \textbf{93\%} & 87\% \\
        Große Nierensteine  & \textbf{73\%} & 69\% \\
        \textbf{Gesamt}     &          78\% & \textbf{83\%} \\
        \bottomrule
        \end{tabular}
    \end{table}

    \begin{figure}[!h]
        \centering
        \begin{tikzpicture}[->,>=stealth',shorten >=1pt,auto,node distance=2.5cm,
      thick,main node/.style={circle,fill=blue!10,draw,font=\sffamily\Large\bfseries}]
          \node (Z) at (1,1) {Z};
          \node (T) at (0,0) {T};
          \node (R) at (2,0) {R};

          \foreach \from/\to in {Z/T,Z/R,T/R}
            \draw (\from) -> (\to);
        \end{tikzpicture}
    \end{figure}

    \begin{align*}
        Z &= N_Z, \;\;\;& N_Z &\sim Ber(\nicefrac{1}{4})\\
        T &= \lfloor 2 \cdot (1-Z+N_T) \rfloor \;\;\; & N_T &\sim \mathcal{N}(0, 1)\\
        R &= \lfloor 2 \cdot (0.6 \cdot (1-Z) + 0.4 \cdot (1-T) + N_R) \rfloor  \;\;\; & N_R &\sim \mathcal{N}(0, 1)
    \end{align*}
\end{frame}

% \begin{frame}{Interventionen: Spezialfälle}
%     \begin{block}{Interventionsverteilung}
%         Wenn $\tilde{f}(\tilde{\mathbf{PA}_j}, \tilde{N}_j)$ eine Punktmasse
%         auf ein $a \in \mathbb{R}$ legt schreibt man
%         \[\mathbb{P}_\mathcal{S, do(X_j := \tilde{f}(\tilde{\mathbf{PA}_j}, \tilde{N}_j))}^{\mathbf{X}}\]
%         und nennt die Intervention
%         \textbf{perfekt}.\\

%         Eine Intervention mit $\tilde{\mathbf{PA}_j} = \mathbf{PA}_j$ wird
%         \textbf{mangelhaft} genannt.
%     \end{block}
% \end{frame}

\begin{frame}[t]{Beispiel 2.2.2: Ursache und Effekt}
    Es sei $\mathcal{S}$ gegeben durch
    \begin{align}
        X &= N_X\\
        Y &= 4 \cdot X + N_Y
    \end{align}
    mit $N_X, N_Y \overset{\text{iid}}{\sim} \mathcal{N}(0, 1)$ und den
    Graphen $X \rightarrow Y$.
    \only<2-9>{
        Dann gilt:
        \begin{align}
            \mathbb{P}_\mathcal{S}^Y = \mathcal{N}(0, 4^2 + 1) &\onslide<3->{\neq \mathcal{N}(8, 1)} \onslide<4->{= \mathbb{P}_{\mathcal{S}, do(X:=2)}^{Y}} \onslide<5->{= \mathbb{P}_\mathcal{S}^{Y|X=2}}\\
            &\onslide<6->{\neq \mathcal{N}(12, 1)} \onslide<7->{= \mathbb{P}_{\mathcal{S}, do(X:=3)}^{Y}} \onslide<8->{= \mathbb{P}_\mathcal{S}^{Y|X=3}}
        \end{align}
        \onslide<9->{$\Rightarrow$ Intervention auf $X$ beeinflusst die Verteilung von $Y$.}
    }
    \only<10-13>{
        Aber:

        \begin{align}
            \mathbb{P}_{\mathcal{S}, do(Y:=2)}^{X} &= \mathcal{N}(0, 1)\\
            \onslide<11->{&= \mathbb{P}_\mathcal{S}^X}\\
            \onslide<12->{&= \mathbb{P}_{\mathcal{S}, do(Y:=3.14159)}^{X}}\\
            \onslide<13->{&\neq \mathbb{P}_\mathcal{S}^{X|Y=2}}
        \end{align}
    }
    \only<14->{\\
        Beispiel: $X$ (rauchen) $\rightarrow Y$ (weiße Zähne)
        \begin{itemize}
            \item<15-> Es besteht eine Asymmetrie zwischen Ursache ($X$) und Effekt ($Y$).
            \item<16-> $\mathbb{P}_{\mathcal{S}, do(Y:=\tilde{N}_Y)}^{X,Y} \Rightarrow X \perp\!\!\!\perp Y$
            \item<17-> $\mathbb{P}_{\mathcal{S}, do(X:=\tilde{N}_X)}^{X,Y} \text{ und } Var(\tilde{N}_X) > 0 \Rightarrow X \not\perp\!\!\!\perp Y$
        \end{itemize}
    }
\end{frame}

\section{Totaler kausaler Effekt}
\subsection{Totaler kausaler Effekt}
\begin{frame}{Totaler kausaler Effekt}
    \begin{block}{Totaler kausaler Effekt}
        Gegeben sei ein SEM $\mathcal{S}$. Dann gibt es einen
        (totalen) kausalen Effekt von $X$ nach $Y$ genau dann wenn
        \[\exists \tilde{N}_X : X \not\!\perp\!\!\!\perp Y \text{ in } \mathbb{P}_{\mathcal{S}, do(X:=\tilde{N}_X)}^{\mathbf{X}}\]
        gilt.
    \end{block}
\end{frame}

\begin{frame}[t]{Totaler kausaler Effekt: Äquivalenzen}
    Folgende Aussagen sind äquivalent:

    \begin{enumerate}[label=(\roman*)]
        \item $\exists \tilde{N}_{X_1} \hphantom{\text{ mit vollem Support }}: X_1 \not\!\perp\!\!\!\perp X_2 \text{ in } \mathbb{P}_{\mathcal{S}, do(X_1:=\tilde{N}_{X_1})}^{\mathbf{X}}$
        \item $\exists x^\triangle \exists x^\square: \mathbb{P}_{\mathcal{S}, do(X_1:=x^\triangle)}^{X_2} \neq \mathbb{P}_{\mathcal{S}, do(X_1:=x^\square)}^{X_2}$
        \item $\exists x^\triangle \hphantom{\exists x^\square}: \mathbb{P}_{\mathcal{S}, do(X_1:=x^\triangle)}^{X_2} \neq \mathbb{P}_\mathcal{S}^{X_2}$.
        \item $\forall \tilde{N}_{X_1} \text{ mit vollem Support }: X_1 \not\!\perp\!\!\!\perp X_2 \text{ in } \mathbb{P}_{\mathcal{S}, do(X_1:=\tilde{N}_{X_1})}^{\mathbf{X}}$
    \end{enumerate}

    \only<2>{
    \textbf{Beweisplan:}\\
    (i) $\Rightarrow$ (ii) $\Rightarrow$ (iv) $\Rightarrow$ (i)\\
    $\neg$(i) $\Rightarrow$ $\neg$ (iii) äquivalent zu (iii) $\Rightarrow$ (i)\\
    (ii) $\Rightarrow$ (iii)
    }
    \only<3-5>{
        \begin{align}
            p_{\mathcal{S}, do(X_1:=x_1)}^{X_2}(x_2) &= \int \prod_{j \neq 1} p_j(x_j|x_{pa(j)}) \mathrm{d}x_3 \dots \mathrm{d}x_p \nonumber
            \only<4->{\\&= \int \prod_{j \neq 1} p_j(x_j|x_{pa(j)}) \frac{\tilde{p}(x_1)}{\tilde{p}(x_1)}\mathrm{d}x_3 \dots \mathrm{d}x_p \nonumber}
            \only<5->{\\&= p_{\mathcal{S}, do(X_1:=\tilde{N}_1)}^{X_2 | X_1=x_1}(x_2)\tag{A.1}\label{eq:A.1}}
        \end{align}
        \only<5->{mit $\tilde{p}(x_1) > 0$.}
    }
    \only<6>{
        \begin{align}
            X_2 \not\perp\!\!\!\perp X_1 \text{ in } \mathbb{Q} \Leftrightarrow &\exists x_1^\triangle, x_1^\square \nonumber\\
            &\text{mit } q(x_1^\triangle), q(x_1^\square) > 0\nonumber\\
            &\text{und } \mathbb{Q}^{X_2|X_1=x_1^\triangle} \neq \mathbb{Q}^{X_2 | X_1=x_1^\square}\tag{A.2}\label{eq:A.2}
        \end{align}
    }

    \only<7>{
        \begin{align}
            X_2 \not\perp\!\!\!\perp X_1 \text{ in } \mathbb{Q} \Leftrightarrow &\exists x_1^\triangle \nonumber\\
            &\text{mit } q(x_1^\triangle) > 0\nonumber\\
            &\text{und } \mathbb{Q}^{X_2|X_1=x_1^\triangle} \neq \mathbb{Q}^{X_2}\tag{A.3}\label{eq:A.3}
        \end{align}
    }

    \only<8-10>{
        \textbf{Beweisplan:} (i) $\Rightarrow$ (ii)\\
        \onslide<9->{(i) $\overset{A.2}{\Rightarrow} \exists x_1^\triangle, x_1^\square$ mit
        pos. Dichte unter $\tilde{N_1}$ sodass $\mathbb{P}_{\mathcal{S}, do(X_1:=\tilde{N_1})}^{X_2|X_1=x_1^\triangle} \neq \mathbb{P}_{\mathcal{S}, do(X_1:=\tilde{N_1})}^{X_2 | X_1=x^\square}$\\}
        \onslide<10->{$\overset{A.1}{\Rightarrow} (ii)$}
    }
    \only<11-13>{
        \textbf{Beweisplan:} (ii) $\Rightarrow$ (iv)\\
        \onslide<12->{(ii) $\overset{A.1}{\Rightarrow} \exists x_1^\triangle, x_1^\square$ mit pos. Dichte unter $\hat{N_1}$ sodass $\mathbb{P}_{\mathcal{S}, do(X_1:=\hat{N_1})}^{X_2|X_1=x_1^\triangle} \neq \mathbb{P}_{\mathcal{S}, do(X_1 := \hat{N_1})}^{X_2 | X_1 = x_1^\square}$}
        \onslide<13->{$\overset{A.2}{\Rightarrow} (iv)$}
    }
    \only<14>{
        \textbf{Beweisplan:} (iv) $\Rightarrow$ (i)\\
        Trivial
    }
    \only<15-17>{
        \textbf{Beweisplan:} $\neg$(i) $\Rightarrow$ $\neg$ (iii)\\
        \onslide<16->{Es gilt: $\mathbb{P}_\mathcal{S}^{X_2} = \mathbb{P}_{\mathcal{S}, do(X_1 := N_1^*)}^{X_2}$, wobei $N_1^*$ wie $\mathbb{P}_\mathcal{S}^{X_2}$ verteilt ist.\\}
        \onslide<17->{
            \begin{align}
                \neg (i) &\Rightarrow X_2 \perp\!\!\!\perp X_1 \text{ in } \mathbb{P}_{\mathcal{S}, do(X_1 := N_1^*)}^{\textbf{X}}\\
                &\overset{A.3}{\Rightarrow} \mathbb{P}_{\mathcal{S}, do(X_1 :=N_1^*)}^{X_2| X_1=x^\triangle} = \mathbb{P}_{\mathcal{S}, do(X_1 := N_1^*)}^{X_2} \;\;\;\forall x^\triangle \text{ mit } p_1(x^\triangle) > 0\\
                &\overset{A.1}{\Rightarrow} \mathbb{P}_{\mathcal{S}, do(X_1:=x^\triangle)}^{X_2} = \mathbb{P}_\mathcal{S}^{X_2} \;\;\; \forall x^\triangle \text{ mit } p_1(x^\triangle) > 0\\
                &\overset{\neg (ii)}{\Rightarrow} \neg (iii)
            \end{align}
        }
    }
    \only<18>{
        \textbf{Beweisplan:} (ii) $\Rightarrow$ (iii)\\
        Trivial (TODO: wirklich?)
    }
\end{frame}

\begin{frame}{Beispiel 2.2.6: Randomisierte Studie}
    \begin{itemize}
        \item<1-> Weise eine Behandlung $T$ zufällig (nach $\tilde{N_T}$) einem
              Patienten zu. Das könnte auch ein Placebo sein.
        \item<2-> Im SEM: Daten aus $\mathbb{P}_{\mathcal{S}, do(T:=\tilde{N_T})}^{\mathbf{X}}$
        \item<3-> Falls immer noch Abhängigkeit zw. Behandlung und Erfolg
              vorliegt $\Rightarrow T$ hat einen totalen kausalen Effekt auf
              den Behandlungserfolg.
    \end{itemize}
\end{frame}

\begin{frame}{Beispiel 2.2.7: Nicolai's running-and-health Beispiel}
    Das zugrundeliegende (\enquote{wahre}) SEM $\mathcal{S}$, welches die Daten
    generierte, hat die Form:

    \begin{align}
        A &= N_A            &&\text{mit } N_A \sim Ber(\nicefrac{1}{2})\\
        H &= A + N_H \mod 2 &&\text{mit } N_H \sim Ber(\nicefrac{1}{3})\\
        B &= H + N_B \mod 2 &&\text{mit } N_B \sim Ber(\nicefrac{1}{20})
    \end{align}

    mit dem Graphen $A \rightarrow H \rightarrow B$ und\\
    $N_A, N_H, N_B$ unabhängig.

    \begin{itemize}
        \item<1->  $B$ ist hilfreicher für die Vorhersage von $H$ als $A$.
        \item<2->  Intervention von $A$ hat auf $H$ einen größeren Einfluss als Intervention von $B$.
    \end{itemize}
\end{frame}

\begin{frame}{Proposition 2.2.9}
    \begin{enumerate}[label=(\roman*)]
        \item<1-> Falls es keinen gerichteten Pfad von $X$ nach $Y$ gibt, dann
                  gibt es keinen kausalen Effekt.
        \item<2-> Manchmal gibt es einen gerichteten Pfad, aber keinen kausalen
                  Effekt.
    \end{enumerate}

    \onslide<3->{Beweis von (i): Folgt aus der Markov-Eigenschaft des
                 interventierten SEMs. }\onslide<4->{Nach dem Entfernen der
                 in $X$ eingehenden Kanten gilt: $X$ und $Y$ sind
                 $d$-separiert, falls es keinen direkten Pfad von $X$ nach
                 $Y$ gibt. \\}
    \onslide<5->{Beweis von (ii) durch Gegenbeispiel: Sei
    \begin{align}
        X &= N_X\\
        Z &= 2X + N_Z\\
        Y &= 4X - 2Z + N_Y
    \end{align}
    Dann gilt: $Y = - 2N_Z + N_Y$ und daher $X \perp\!\!\!\perp$ für alle $N_X$. $\square$
    }
\end{frame}

% \begin{frame}{Nierensteine}
% \begin{columns}
%     \begin{column}{0.45\textwidth}
%         \begin{center}\textbf{Modell A}\end{center}
%     \end{column}
%     \begin{column}{0.45\textwidth}
%         \begin{center}\textbf{Modell B}\end{center}
%     \end{column}
% \end{columns}
% \end{frame}