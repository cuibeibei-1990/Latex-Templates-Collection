\subsection{How can we use this massive amout of information?}
\begin{frame}{How can we use this massive amout of information?}
    \begin{itemize}[<+->]
        \item 625.3 million websites
        \item Wikipedia is one website and has several millions of pages
        \item[$\Rightarrow$] we need to rank websites!
    \end{itemize}
\end{frame}

\subsection{Idea}
\begin{frame}{Basics of PageRank}
    We all know that:
    \begin{itemize}[<+->]
        \item humans know what is good for them
        \item[\xmark] machines don't know what's good for humans
        \item humans create websites
        \item humans will only \href{http://en.wikipedia.org/wiki/Hyperlink}{link} to websites they like
        \item[$\Rightarrow$] hyperlinks are a quality indicator
    \end{itemize}
\end{frame}

\begin{frame}{How Could We Use That?}
    \begin{itemize}[<+->]
        \item simply count number of links to a website
        \item[\xmark] 10,000 links from only one page
        \item count number of websites that link to a website
        \item[\xmark] quality of the linking website matters
        \item[\xmark] total number of links on the source page matters
    \end{itemize}
\end{frame}

\framedgraphic{A Brilliant Idea}{../images/BrinPage.jpg}

\begin{frame}{Ideas of PageRank}
    \begin{itemize}[<+->]
        \item decisions of humans are complicated
        \item a lot of webpages get visited
        \item[$\Rightarrow$] modellize clicks on links as random behaviour
        \item links are important
        \begin{itemize}
            \item links of page A get less important, if A has many links
            \item links of page A get more important, if many link to A
        \end{itemize}
        \item[$\Rightarrow$] if B has a link from A, the rank of B increases by $\frac{Rank(A)}{Links(A)}$
    \end{itemize}

    \pause[\thebeamerpauses]

    \begin{algorithmic}
        \If{A links to B}
            \State $Rank(B)$ += $\frac{Rank(A)}{Links(A)}$
        \EndIf
    \end{algorithmic}
\end{frame}

\begin{frame}{What is PageRank?}
    The PageRank algorithm calculates the probability of a randomly
    clicking user ending up on a given page.
\end{frame}

\documentclass[aspectratio=169,hyperref={pdfpagelabels=false}]{beamer}
\usepackage{lmodern}

\usepackage[utf8]{inputenc} % this is needed for german umlauts
\usepackage[ngerman]{babel} % this is needed for german umlauts
\usepackage[T1]{fontenc}    % this is needed for correct output of umlauts in pdf

\usepackage{braket} % needed for \Set
\usepackage{algorithm,algpseudocode}

\usepackage{verbatim}
\usepackage{tikz}
\usetikzlibrary{arrows,shapes}

% Define some styles for graphs
\tikzstyle{vertex}=[circle,fill=black!25,minimum size=20pt,inner sep=0pt]
\tikzstyle{selected vertex} = [vertex, fill=red!24]
\tikzstyle{blue vertex} = [vertex, fill=blue!24]
\tikzstyle{edge} = [draw,thick,-]
\tikzstyle{weight} = [font=\small]
\tikzstyle{selected edge} = [draw,line width=5pt,-,red!50]
\tikzstyle{ignored edge} = [draw,line width=5pt,-,black!20]

% see http://deic.uab.es/~iblanes/beamer_gallery/index_by_theme.html
%\usetheme{Frankfurt}
\usefonttheme{professionalfonts}

% disables bottom navigation bar
\beamertemplatenavigationsymbolsempty

% http://tex.stackexchange.com/questions/23727/converting-beamer-slides-to-animated-images
\setbeamertemplate{navigation symbols}{}%

\newcommand{\alertline}{%
 \usebeamercolor[fg]{normal text}%
 \only{\usebeamercolor[fg]{alerted text}}}


\begin{document}
\pgfdeclarelayer{background}
\pgfsetlayers{background,main}
\newcommand\hlight[1]{\tikz[overlay, remember picture,baseline=-\the\dimexpr\fontdimen22\textfont2\relax]\node[rectangle,fill=blue!50,rounded corners,fill opacity = 0.2,draw,thick,text opacity =1] {$#1$};}
\newcommand\tocalculate[1]{\tikz[overlay, remember picture,baseline=-\the\dimexpr\fontdimen22\textfont2\relax]\node[rectangle,fill=green!50,rounded corners,fill opacity = 0.2,draw,thick,text opacity =1] {$#1$};}

\begin{frame}
    \begin{minipage}[b]{0.30\linewidth}
    \centering
    \begin{align*}
        A &= \begin{pmatrix}
            1 &  2 & 3\\
            2 &  8 & 14\\
            3 & 14 & 34
        \end{pmatrix}\\
       \alertline<1> L &= \alertline<1>\begin{pmatrix}
            0 & 0 & 0\\
            0 & 0 & 0\\
            0 & 0 & 0
        \end{pmatrix}\\
        tmp &= 0
    \end{align*}
    \end{minipage}
    \hspace{0.5cm}
    \begin{minipage}[b]{0.60\linewidth}
    \centering
    \begin{algorithm}[H]
        \begin{algorithmic}
            \Function{Cholesky}{$A \in \mathbb{R}^{n \times n}$}
                \alertline<1>\State $L = \Set{0} \in \mathbb{R}^{n \times n}$ \Comment{Initialisiere $L$}\\

                \alertline<2>\alertline<4>\For{($k=1$; $\;k \leq n$; $\;k$++)}
                    \alertline<3>\State $L_{k,k} = \sqrt{A_{k,k} - \sum_{i=1}^{k-1} L_{k,i}^2}$
                    \For{($i=k+1$; $\;i \leq n$; $\;i$++)}
                        \State $L_{i,k} = \frac{A_{i,k} - \sum_{j=1}^{k-1} L_{i,j} \cdot L_{k,j}}{L_{k,k}}$
                    \EndFor
                \EndFor
                \alertline<5>\State \Return $L$
            \EndFunction
        \end{algorithmic}
    \caption{Cholesky-Zerlegung}
    \label{alg:seq1}
    \end{algorithm}
    \end{minipage}
\end{frame}
\begin{frame}
    \begin{align*}
        A &= \begin{pmatrix}
            \hlight{1} &  2 & 3\\
            2 &  8 & 14\\
            3 & 14 & 34
        \end{pmatrix}\\
        L &= \begin{pmatrix}
            \tocalculate{0} & 0 & 0\\
            0 & 0 & 0\\
            0 & 0 & 0
        \end{pmatrix}\\
        tmp &= 0
    \end{align*}
\end{frame}
\begin{frame}
    \begin{align*}
        A &= \begin{pmatrix}
            1 &  2 & 3\\
            2 &  8 & 14\\
            3 & 14 & 34
        \end{pmatrix}\\
        L &= \begin{pmatrix}
            \tocalculate{1} & 0 & 0\\
            0 & 0 & 0\\
            0 & 0 & 0
        \end{pmatrix}\\
        tmp &= 0
    \end{align*}
\end{frame}
%%%%%%%%%%%%%%%%%%%%%%%%%%%%%%%%%%%%%%%%%%%%%%%%%%%%%%%%%%%%%%%%
% Calculate L_2,1
\begin{frame}
    \begin{align*}
        A &= \begin{pmatrix}
            1 &  2 & 3\\
            \hlight{2} &  8 & 14\\
            3 & 14 & 34
        \end{pmatrix}\\
        L &= \begin{pmatrix}
            1 & 0 & 0\\
            \tocalculate{0} & 0 & 0\\
            0 & 0 & 0
        \end{pmatrix}\\
        tmp &= 0
    \end{align*}
\end{frame}

\begin{frame}
    \begin{align*}
        A &= \begin{pmatrix}
            1 &  2 & 3\\
            2 &  8 & 14\\
            3 & 14 & 34
        \end{pmatrix}\\
        L &= \begin{pmatrix}
            \hlight{1} & 0 & 0\\
            \tocalculate{2} & 0 & 0\\
            0 & 0 & 0
        \end{pmatrix}\\
        tmp &= 0
    \end{align*}
\end{frame}

\begin{frame}
    \begin{align*}
        A &= \begin{pmatrix}
            1 &  2 & 3\\
            2 &  8 & 14\\
            3 & 14 & 34
        \end{pmatrix}\\
        L &= \begin{pmatrix}
            1 & 0 & 0\\
            \tocalculate{2} & 0 & 0\\
            0 & 0 & 0
        \end{pmatrix}\\
        tmp &= 0
    \end{align*}
\end{frame}
%%%%%%%%%%%%%%%%%%%%%%%%%%%%%%%%%%%%%%%%%%%%%%%%%%%%%%%%%%%%%%%%
% Calculate L_3,1
\begin{frame}
    \begin{align*}
        A &= \begin{pmatrix}
            1 &  2 & 3\\
            2 &  8 & 14\\
            \hlight{3} & 14 & 34
        \end{pmatrix}\\
        L &= \begin{pmatrix}
            1 & 0 & 0\\
            2 & 0 & 0\\
            \tocalculate{0} & 0 & 0
        \end{pmatrix}\\
        tmp &= 0
    \end{align*}
\end{frame}

\begin{frame}
    \begin{align*}
        A &= \begin{pmatrix}
            \hlight{1} &  2 & 3\\
            2 &  8 & 14\\
            3 & 14 & 34
        \end{pmatrix}\\
        L &= \begin{pmatrix}
            1 & 0 & 0\\
            2 & 0 & 0\\
            \tocalculate{3} & 0 & 0
        \end{pmatrix}\\
        tmp &= 0
    \end{align*}
\end{frame}

%%%%%%%%%%%%%%%%%%%%%%%%%%%%%%%%%%%%%%%%%%%%%%%%%%%%%%%%%%%%%%%%
% Calculate L_2,2
\begin{frame}
    \begin{align*}
        A &= \begin{pmatrix}
            1 &  2 & 3\\
            2 &  \hlight{8} & 14\\
            3 & 14 & 34
        \end{pmatrix}\\
        L &= \begin{pmatrix}
            1 & 0 & 0\\
            2 & \tocalculate{0} & 0\\
            3 & 0 & 0
        \end{pmatrix}\\
        tmp &= 0
    \end{align*}
\end{frame}

\begin{frame}
    \begin{align*}
        A &= \begin{pmatrix}
            1 &  2 & 3\\
            2 &  8 & 14\\
            3 & 14 & 34
        \end{pmatrix}\\
        L &= \begin{pmatrix}
            1 & 0 & 0\\
            \hlight{2} & \tocalculate{8} & 0\\
            3 & 0 & 0
        \end{pmatrix}\\
        tmp &= 0
    \end{align*}
\end{frame}

\begin{frame}
    \begin{align*}
        A &= \begin{pmatrix}
            1 &  2 & 3\\
            2 &  8 & 14\\
            3 & 14 & 34
        \end{pmatrix}\\
        L &= \begin{pmatrix}
            1 & 0 & 0\\
            2 & \tocalculate{8} & 0\\
            3 & 0 & 0
        \end{pmatrix}\\
        tmp &= \hlight{4}
    \end{align*}
\end{frame}

\begin{frame}
    \begin{align*}
        A &= \begin{pmatrix}
            1 &  2 & 3\\
            2 &  8 & 14\\
            3 & 14 & 34
        \end{pmatrix}\\
        L &= \begin{pmatrix}
            1 & 0 & 0\\
            2 & \tocalculate{4} & 0\\
            3 & 0 & 0
        \end{pmatrix}\\
        tmp &= 0
    \end{align*}
\end{frame}


\begin{frame}
    \begin{align*}
        A &= \begin{pmatrix}
            1 &  2 & 3\\
            2 &  8 & 14\\
            3 & 14 & 34
        \end{pmatrix}\\
        L &= \begin{pmatrix}
            1 & 0 & 0\\
            2 & \tocalculate{2} & 0\\
            3 & 0 & 0
        \end{pmatrix}\\
        tmp &= 0
    \end{align*}
\end{frame}
\end{document}


%\begin{frame}{Ants}
%    \begin{itemize}[<+->]
%        \item Websites = nodes = anthill
%        \item Links = edges = paths
%        \item You place ants on each node
%        \item They walk over the paths
%        \item[] (at random, they are ants!)
%        \item After some time, some anthills will have more ants than
%              others
%        \item Those hills are more attractive than others
%        \item \# ants is probability that a random user would end on
%              a website
%    \end{itemize}
%\end{frame}

\begin{frame}{Mathematics}
    Let $x$ be a web page. Then
    \begin{itemize}
        \item $L(x)$ is the set of websites that link to $x$
        \item $C(y)$ is the out-degree of page $y$
        \item $\alpha$ is probability of random jump
        \item $N$ is the total number of websites
    \end{itemize}

    \[\displaystyle PR(x) := \alpha \left ( \frac{1}{N} \right ) + (1-\alpha) \sum_{y\in L(x)} \frac{PR(y)}{C(y)}\]
\end{frame}

\begin{frame}{Pseudocode}
        \begin{algorithmic}
\alertline<1>             \Function{PageRank}{Graph $web$, double $q=0.15$, int $iterations$} %q is a damping factor
%\alertline<2>                 \ForAll{$page \in web$}
%\alertline<3>                     \State $page.pageRank = \frac{1}{|web|}$ \Comment{intial probability}
%\alertline<2>                 \EndFor

\alertline<2>                 \While{$iterations > 0$}
\alertline<3>                     \ForAll{$page \in web$} \Comment{calculate pageRank of $page$}
\alertline<4>                         \State $page.pageRank = q$
\alertline<5>                         \ForAll{$y \in L(page)$}
\alertline<6>                             \State $page.pageRank$ += $\frac{y.pageRank}{C(y)}$
\alertline<5>                         \EndFor
\alertline<3>                     \EndFor
\alertline<2>                     \State $iterations$ -= $1$
\alertline<2>                 \EndWhile
\alertline<1>             \EndFunction
        \end{algorithmic}
\end{frame}
