\section{Prinzip der asymmetrischen Verschlüsselung}
Bereits im 5. Jahrhundert v. Chr. war ein Verfahren zur geheimen Weitergabe an Informationen bekannt: Die Skytale\footnote{[Wrixon], S. 21f}. Man wickelte Papier spiralförmig um einen Stab, die Skytale,  und schrieb die Nachricht längs der Skytale auf das Papier. Dann wurde das Papier dem Empfänger gebracht, der mit einer gleichen Skytale diese Nachricht entschlüsseln konnte.

Verfahren, die den gleichen Schlüssel zur Ver- als auch zur Entschlüsselung verwenden nennt man symmetrisch\footnote{[Birthälmer], S. 4}. In dem Beispiel ist die Skytale der Schlüssel. Es ist unüblich die Skytale als symmetrische Verschlüsselung zu bezeichnen, normalerweise sind Block- oder Stromchiffren damit gemeint. Diese sind jedoch schwerer zu beschreiben.

Nach Kerckhoffs' Prinzip darf ein Verschlüsselungssystem keine Geheimhaltung erfordern\footnote{[Petitcolas]}, also muss der Schlüssel für die Sicherheit sorgen. Wollen allerdings zwei Personen miteinander geheim kommunizieren, so muss dieser Schlüssel übertragen werden. Bei der Übertragung könnte er abgefangen werden.
Asymmetrische Verschlüsselungsverfahren benutzen einen öffentlichen Schlüssel zum verschlüsseln und einen privaten Schlüssel zum entschlüsseln. Will Alice eine geheime Nachricht von Bob empfangen, so schickt sie Bob ihren öffentlichen Schlüssel. Bob verschlüsselt seine Nachricht mit diesem Schlüssel und schickt die Nachricht an Alice. Der private Schlüssel wird nicht übertragen. In dieser Hinsicht sind asymmetrische Verschlüsselungsverfahren sicherer als symmetrische.

Mithilfe von asymmetrischen Verschlüsselungen kann man auch digitale Signaturen erstellen und sich damit authentifizieren. Im RSA-Verfahren sind privater und öffentlicher Schlüssel austauschbar. Das heißt, wenn eine Nachricht mit dem privaten Schlüssel verschlüsselt wird, kann sie mit dem öffentlichen Schlüssel entschlüsselt werden. Da jedoch nur der Besitzer des privaten Schlüssels eine Nachricht erstellen kann, die man mit dem öffentlichen Schlüssel entschlüsseln kann, ist es so möglich den Absender einer Nachricht zu authentifizieren.
